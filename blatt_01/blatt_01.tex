\documentclass[a4paper]{scrartcl}

% font/encoding packages
\usepackage[utf8]{inputenc}
\usepackage[T1]{fontenc}
\usepackage{lmodern}
\usepackage[ngerman]{babel}

\usepackage{amsmath, amssymb, amsfonts, amsthm, mathtools}
\allowdisplaybreaks
\newtheorem*{behaupt}{Behauptung}

\title{Formale Grundlagen der Informatik II}
\subtitle{Aufgabenblatt 1}
\author{
    Jan-Hendrik Briese (TODO) \\
    Lennart Braun (6523742) \\
    Marc Strothmann (6537646) \\
    Maximilian Knapperzbusch (6535090)
}
\date{zum 20. Oktober 2014}

\begin{document}
\maketitle

\section*{Übungsaufgabe 1.3}
\begin{enumerate}
    \item
        \begin{equation}
            \label{equ:13r}
            \begin{split}
                R_n &= (a^0 \cdot d \cdot b^0) + (a^2 \cdot d \cdot b^2) + \cdots + (a^n \cdot d \cdot b^n) \\
                  &+ (a^1 \cdot c \cdot b^1) + (a^3 \cdot c \cdot b^2) + \cdots + (a^{n-1} \cdot c \cdot b^{n-1}) \\
                  &+ (a^n \cdot d)
            \end{split}
        \end{equation}
        
    \item
        \begin{equation}
            M_n = \underbrace{\left( \bigcup_{i=0}^{\frac{n}{2}} \left\{ a^{2i} \cdot d \cdot b^{2i} \right\} \right)}_A
            \cup \underbrace{\left( \bigcup_{i=0}^{\frac{n}{2}-1} \left\{ a^{2i+1} \cdot c \cdot b^{2i+1} \right\} \right)}_B
            \cup \underbrace{\left\{ a^n \cdot d \right\}}_C
        \end{equation}
        
    \item
        Es ist zu sehen, dass $R_n$ und $M_n$ die gleiche Sprache beschreiben.
        $A$ entspricht der ersten Zeile von \eqref{equ:13r}, $B$ der zweiten
        und $C$ der dritten.
        \begin{behaupt}
            \begin{equation}
                M_n = L(A_n)
            \end{equation}
        \end{behaupt}
        \begin{proof} \hfill \\
            \begin{enumerate}
                \item $M_n \subseteq L(A_n)$ \\
                    Sei $w \in M_n$.
                    Dann ist $w$ Element mindestens einer der Mengen $A,B,C$.
                    \begin{enumerate}
                        \item[Fall $A$] 
                            \begin{equation}
                                w \in \bigcup_{i=0}^{\frac{n}{2}} \left\{ a^{2i} \cdot d \cdot b^{2n} \right\} \Leftrightarrow w = xyz \text{ mit } x = a^{2i} \text{, } y = d \text{, } z = b^{2i}
                            \end{equation}
                            \begin{equation}
                                \label{equ:13aa}
                                (p_0, xyz) \vdash^* (p_{4i}, yz) \vdash (p_{4i+1}, z) \vdash^* (p_1, \epsilon)
                            \end{equation}
                            Da \eqref{equ:13aa} eine Erfolgsrechnung ist, gilt $w \in L(A_n)$.

                        \item[Fall $B$] 
                            \begin{equation}
                                w \in \bigcup_{i=0}^{\frac{n}{2}-1} \left\{ a^{2i+1} \cdot c \cdot b^{2n+1} \right\} \Leftrightarrow w = xyz \text{ mit } x = a^{2i+1} \text{, } y = c \text{, } z = b^{2i+1}
                            \end{equation}
                            \begin{equation}
                                \label{equ:13ab}
                                (p_0, xyz) \vdash^* (p_{4i+2}, yz) \vdash (p_{4i+3}, z) \vdash^* (p_1, \epsilon)
                            \end{equation}
                            Da \eqref{equ:13ab} eine Erfolgsrechnung ist, gilt $w \in L(A_n)$.

                        \item[Fall $C$] 
                            \begin{equation}
                                w \in \left\{ a^n \cdot d \right\} \Leftrightarrow w = xy \text{ mit } x = a^n \text{, } y = d
                            \end{equation}
                            \begin{equation}
                                \label{equ:13ac}
                                (p_0, xy) \vdash^* (p_{2n}, y) \vdash (p_{2n+1}, \epsilon)
                            \end{equation}
                            Da \eqref{equ:13ac} eine Erfolgsrechnung ist, gilt $w \in L(A_n)$.
                            
                    \end{enumerate}
                \item $L(A_n) \subseteq M_n$ \\
                    Sei $w \in L(A_n)$.
                    Es gibt drei grundsätzliche Möglichkeiten in einen
                    Endzustand von $A_n$ zu gelangen.
                    \begin{enumerate}
                        \item
                            Es wird eine gerade Anzahl $k = 2i \leq n$ an $a$s
                            eingelesen, dann ein $d$ und schließlich $k$ $b$s.
                            $A_n$ befindet sich im Endzustand $p_1$.
                            $w = a^{2i}db^{2i} \in M_n$ (Teilmenge $A$)
                        \item
                            Es wird eine ungerade Anzahl $k = 2i+1 \leq n$ an
                            $a$s eingelesen, dann ein $c$ und schließlich $k$
                            $b$s.
                            $A_n$ befindet sich im Endzustand $p_1$.
                            $w = a^{2i+1}db^{2i+1} \in M_n$ (Teilmenge $B$)
                        \item
                            Es werden $n$ $a$s und dann ein $d$ eingelesen.
                            $A_n$ befindet sich im Endzustand $p_{2n+1}$.
                            $w = a^nd \in M_n$ (Teilmenge $C$)
                    \end{enumerate}
            \end{enumerate}
            Aus $M_n \subseteq L(A_n)$ und $L(A_n) \subseteq M_n$ folgt
            $M_n = L(A_n)$.
        \end{proof}

    \item
        Da $n$ fest gewählt ist, können die Auslassungszeichen in $R_n$
        aufgelöst werden.
        Jede Sprache, die durch einen regulären Ausdruck dargestellt werden
        kann, wird auch von einem endlichen Automaten akzeptiert und ist daher
        regulär.

    \item TODO

\end{enumerate}

\section*{Übungsaufgabe 1.4}
\begin{enumerate}
    \item TODO
    \item TODO
    \item
        \begin{equation}
            R = (e + f)^* \cdot eef \cdot (e + f)^*
        \end{equation}
        
    \item TODO
    \item Begründen Sie die Korrektheit der Lösung. \\
        \begin{behaupt}
            \begin{equation*}
                L(A') = M
            \end{equation*}
        \end{behaupt}
        \begin{proof} \hfill \\
            \begin{enumerate}
                \item $M \subseteq L(A')$ \\
                    Sei $w \in M$ ein Wort, das vom Automaten $A'$ akzeptiert
                    werden soll.
                    So muss $w$ die gespiegelte Form eines Wortes sein, welches
                    vom Automaten $A$ akzeptiert wird.
                    Im Startzustand $\{p_3\}$ können zunächst beliebig viele
                    $e$s gelesen werden.
                    Damit $A'$ in den Folgezustand $\{p_3,p_2\}$ wechseln kann,
                    muss ein $f$ gelesen werden.
                    Sobald sich der Automat $A'$ in diesem Zustand ist, können
                    beliebig viele $f$ gelesen werden.
                    Beim Lesen eines $e$ wechselt $A'$ in den Zustand
                    $\{p_3,p_2,p_1\}$.
                    In diesem Zustand kann durch das Lesen eines $f$ in den
                    vorherigen Zustand wechseln, oder durch das Lesen eines
                    weiteren $e$ in den Endzustand springen.
                    Im Endzustand können beliebig $e$ und $f$ gelesen werden.
                    Anhand der Übergänge vom Startzustand bis zum Endzustand
                    kann man erkennen, dass der Endzustand nur erreicht werden
                    kann, wenn jede Zerlegung von $w$ die Kombination $fee$
                    enthält.
                    Spiegelt man das Wort $w$, erhält man in der Zerlegung die
                    Kombination $eef$.
                    Daraus folgt, dass der Automat $A'$ die gespiegelten Wörter
                    des Automaten $A$ akzeptiert.
                \item $L(A') \subseteq M$ \\
                    Sei $w \in L(A')$, d.\,h. ein Wort, das vom Automaten
                    akzeptiert wird.
                    Wir müssen nun zeigen, dass daraus folgt, dass $w \in M$
                    gilt.
                    Zunächst bemerken wir, dass im Startzustand beliebig viele
                    $e$s gelesen werden können und dass der Automat nur in den
                    Folgezustand überführt werden kann, wenn ein $f$ gelesen
                    wird.
                    Um den Automaten in den Endzustand überführt werden kann,
                    müssen hintereinander ein $f$ und zwei $e$s gelesen werden.
                    Ist die Kombination nicht Teil des Wortes, so kann der
                    Automat nicht in den Endzustand gelangen.
                    Da $w$ diese Kombination beinhaltet, ist gezeigt dass $w$
                    gespiegelt ein Eingabewort des Automaten $A$ ist.
            \end{enumerate}
            Aus a) und b) folgt $L(A') = M$.
            $A'$ akzeptiert die gespiegelten Wörter aus $L(A)$.
        \end{proof}

    	
\end{enumerate}


\end{document}
