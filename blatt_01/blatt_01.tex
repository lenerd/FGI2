\documentclass[a4paper]{scrartcl}

% font/encoding packages
\usepackage[utf8]{inputenc}
\usepackage[T1]{fontenc}
\usepackage{lmodern}
\usepackage[ngerman]{babel}

% math
\usepackage{amsmath, amssymb, amsfonts, amsthm, mathtools}
\allowdisplaybreaks
\newtheorem*{behaupt}{Behauptung}

% tikz
\usepackage{tikz}
\usetikzlibrary{arrows,automata}

% misc
\usepackage{enumerate}

\title{Formale Grundlagen der Informatik II}
\subtitle{Aufgabenblatt 1}
\author{
    Jan-Hendrik Briese (6523408) \\
    Lennart Braun (6523742) \\
    Marc Strothmann (6537646) \\
    Maximilian Knapperzbusch (6535090)
}
\date{zum 20. Oktober 2014}

\begin{document}
\maketitle

\section*{Übungsaufgabe 1.3}
\begin{enumerate}
    \item
        \begin{equation}
            \label{equ:13r}
            \begin{split}
                R_n &= (a^0 \cdot d \cdot b^0) + (a^2 \cdot d \cdot b^2) + \cdots + (a^n \cdot d \cdot b^n) \\
                  &+ (a^1 \cdot c \cdot b^1) + (a^3 \cdot c \cdot b^2) + \cdots + (a^{n-1} \cdot c \cdot b^{n-1}) \\
                  &+ (a^n \cdot d)
            \end{split}
        \end{equation}
        
    \item
        \begin{equation}
            M_n = \underbrace{\left( \bigcup_{i=0}^{\frac{n}{2}} \left\{ a^{2i} \cdot d \cdot b^{2i} \right\} \right)}_A
            \cup \underbrace{\left( \bigcup_{i=0}^{\frac{n}{2}-1} \left\{ a^{2i+1} \cdot c \cdot b^{2i+1} \right\} \right)}_B
            \cup \underbrace{\left\{ a^n \cdot d \right\}}_C
        \end{equation}
        
    \item
        Es ist zu sehen, dass $R_n$ und $M_n$ die gleiche Sprache beschreiben.
        $A$ entspricht der ersten Zeile von \eqref{equ:13r}, $B$ der zweiten
        und $C$ der dritten.
        \begin{behaupt}
            \begin{equation}
                M_n = L(A_n)
            \end{equation}
        \end{behaupt}
        \begin{proof} \hfill \\
            \begin{enumerate}
                \item $M_n \subseteq L(A_n)$ \\
                    Sei $w \in M_n$.
                    Dann ist $w$ Element mindestens einer der Mengen $A,B,C$.
                    \begin{enumerate}
                        \item[Fall $A$] 
                            \begin{equation}
                                w \in \bigcup_{i=0}^{\frac{n}{2}} \left\{ a^{2i} \cdot d \cdot b^{2n} \right\} \Leftrightarrow w = xyz \text{ mit } x = a^{2i} \text{, } y = d \text{, } z = b^{2i}
                            \end{equation}
                            \begin{equation}
                                \label{equ:13aa}
                                (p_0, xyz) \vdash^* (p_{4i}, yz) \vdash (p_{4i+1}, z) \vdash^* (p_1, \epsilon)
                            \end{equation}
                            Da \eqref{equ:13aa} eine Erfolgsrechnung ist, gilt $w \in L(A_n)$.

                        \item[Fall $B$] 
                            \begin{equation}
                                w \in \bigcup_{i=0}^{\frac{n}{2}-1} \left\{ a^{2i+1} \cdot c \cdot b^{2n+1} \right\} \Leftrightarrow w = xyz \text{ mit } x = a^{2i+1} \text{, } y = c \text{, } z = b^{2i+1}
                            \end{equation}
                            \begin{equation}
                                \label{equ:13ab}
                                (p_0, xyz) \vdash^* (p_{4i+2}, yz) \vdash (p_{4i+3}, z) \vdash^* (p_1, \epsilon)
                            \end{equation}
                            Da \eqref{equ:13ab} eine Erfolgsrechnung ist, gilt $w \in L(A_n)$.

                        \item[Fall $C$] 
                            \begin{equation}
                                w \in \left\{ a^n \cdot d \right\} \Leftrightarrow w = xy \text{ mit } x = a^n \text{, } y = d
                            \end{equation}
                            \begin{equation}
                                \label{equ:13ac}
                                (p_0, xy) \vdash^* (p_{2n}, y) \vdash (p_{2n+1}, \epsilon)
                            \end{equation}
                            Da \eqref{equ:13ac} eine Erfolgsrechnung ist, gilt $w \in L(A_n)$.
                            
                    \end{enumerate}
                \item $L(A_n) \subseteq M_n$ \\
                    Sei $w \in L(A_n)$.
                    Es gibt drei grundsätzliche Möglichkeiten in einen
                    Endzustand von $A_n$ zu gelangen.
                    \begin{enumerate}
                        \item
                            Es wird eine gerade Anzahl $k = 2i \leq n$ an $a$s
                            eingelesen, dann ein $d$ und schließlich $k$ $b$s.
                            $A_n$ befindet sich im Endzustand $p_1$.
                            $w = a^{2i}db^{2i} \in M_n$ (Teilmenge $A$)
                        \item
                            Es wird eine ungerade Anzahl $k = 2i+1 \leq n$ an
                            $a$s eingelesen, dann ein $c$ und schließlich $k$
                            $b$s.
                            $A_n$ befindet sich im Endzustand $p_1$.
                            $w = a^{2i+1}db^{2i+1} \in M_n$ (Teilmenge $B$)
                        \item
                            Es werden $n$ $a$s und dann ein $d$ eingelesen.
                            $A_n$ befindet sich im Endzustand $p_{2n+1}$.
                            $w = a^nd \in M_n$ (Teilmenge $C$)
                    \end{enumerate}
            \end{enumerate}
            Aus $M_n \subseteq L(A_n)$ und $L(A_n) \subseteq M_n$ folgt
            $M_n = L(A_n)$.
        \end{proof}

    \item
        Da $n$ fest gewählt ist, können die Auslassungszeichen in $R_n$
        aufgelöst werden.
        Jede Sprache, die durch einen regulären Ausdruck dargestellt werden
        kann, wird auch von einem endlichen Automaten akzeptiert und ist daher
        regulär.

    \item
        \begin{equation}
            A = \bigcup_{n \geq 0} L(A_n)
        \end{equation}
        Es wird davon ausgegangen, dass der Automat $A_n$ mit $n \mod 2 = 1$ das
        Wort $w = a^nc$ akzeptiert; also dass der Zustand $p_{2n+1}$ auch bei
        ungeradem $n$ ein Endzustand ist.

        Die Sprache wäre wie folgt aufgebaut:
        \begin{equation}
            A =  \underbrace{\left\{ a^{2n}d \right\}}_A
            \cup \underbrace{\left\{ a^{2n}db^{2n} \right\}}_B
            \cup \underbrace{\left\{ a^{2n+1}c \right\}}_C
            \cup \underbrace{\left\{ a^{2n+1}cb^{2n+1} \right\}}_D
            \text{ für } n \geq 0
        \end{equation}
        Während es trivial ist endliche Automaten zu konstruieren, welche genau
        die Teilmengen $A$ und $C$ als Sprache akzeptieren, ist dies für $B$
        und $D$ nicht möglich.
        Mit dem Pumping Lemma für reguläre Sprachen kann gezeigt werden, dass
        Mengen der Form $\left\{ a^kcb^k \  | \  k \in \mathbb{N} \right\}$ nicht
        zu der Familie der regulären Sprachen gehören.

        Es ist möglich eine kontextfreie Grammatik anzugeben, die $A$
        generiert.
        \begin{equation*}
            \begin{split}
                S  &\rightarrow A \ |\  B \ |\  C \ |\  D \\
                A  &\rightarrow A'd \\
                A' &\rightarrow aaA' \ |\  \epsilon \\
                B  &\rightarrow aaBbb \ |\ d \\
                C  &\rightarrow aC'c \\
                C' &\rightarrow aaC' \ |\  \epsilon \\
                D  &\rightarrow aD'b \\
                D' &\rightarrow aaD'bb \ |\ c
            \end{split}
        \end{equation*}
        
        

\end{enumerate}

\section*{Übungsaufgabe 1.4}
\begin{enumerate}
    \item
        Das im Folgenden beschriebene Verfahren konstruiert aus einem DFA
        $A = \left( Q, \Sigma, \delta, \left\{ q^0 \right\}, F \right)$
        in drei Schritten einen DFA
        $A' = \left( Q', \Sigma, \delta', \left\{ q'^0 \right\}, F' \right)$
        mit $L(A') = \left\{ w^{rev} \ |\ w \in L(A) \right\}$.

        \begin{enumerate}[1.]
            \item
                Start- und Endzustände werden vertauscht:
                \begin{align*}
                    F'_N  &= \left\{ q^0 \right\} \\
                    Q^0_N &= \left\{ q \ |\ q \in F \right\}
                \end{align*}
                Das Zwischenergebnis ist (insbesondere bei $|F| > 1$) ein NFA 
                $N_1 = \left( Q, \Sigma, \delta, Q^0_N, F'_N \right)$.

            \item
                Die Richtungen der Kanten werden umgekehrt.
                \begin{equation*}
                    \delta' = \left\{ (q_k, \sigma, q_l)
                    \ |\ (q_l, \sigma, q_k) \in \delta \right\}
                \end{equation*}
                Das Zwischenergebnis ist ein NFA
                $N_2 = \left( Q, \Sigma, \delta', Q^0_N, F'_N \right)$.

            \item
                Aus dem NFA $N_2$ wird mittels Potenzautomatenkonstruktion ein
                äquivalenter vollständiger deterministischer Automat
                konstruiert.
                Das genaue Verfahren ist bereits aus den Präsenzaufgaben bekannt.
                Das Ergebnis ist ein DFA
                $A' = \left( Q', \Sigma, \delta', {q'^0}, F' \right)$,
                der die oben definierte Sprache akzeptiert.
        \end{enumerate}

    \item
        Sei $M$ die Menge aller Wörter über $\Sigma = \{ e, f \}$, die das
        Teilwort $eef$ enthalten.
        Sei $A$ der folgende DFA.
        \begin{center}
            \begin{tikzpicture}[->,>=stealth',shorten >=1pt,auto,node distance=1.8cm,
                                semithick]
              \tikzstyle{every state}=[fill=white,draw=black,text=black]
            
              \node[initial,state] 	(A)              {$p_0$};
              \node[state]         	(B) [right of=A] {$p_1$};
              \node[state]         	(C) [right of=B] {$p_2$};
              \node[accepting,state](D) [right of=C] {$p_3$};
            
              \path (A) edge [loop above] 	node {f} (A)
                        edge [bend left]  	node {e} (B)
                    (B) edge [bend left]	node {f} (A)
                        edge 				node {e} (C)
                    (C)	edge [loop above]	node {e} (C)
                        edge				node {f} (D)
                    (D)	edge [loop above]	node {e,f} (D);
            \end{tikzpicture}
        \end{center}
        \begin{behaupt}
            L(A) = M
        \end{behaupt}
        \begin{proof}
            \begin{enumerate}
                \item $L(A) \subseteq M$ \hfill \\
                    Ein Wort $w \in L(A)$ wird von $L(A)$ akzeptiert.
                    Damit muss diese Wort auch die Zeichenkette $eef$ enthalten,
                    um aus dem Startzustand $p_0$ den Endzustand $p_3$ erreichen
                    zu können.
                    Der Endzustand $p_3$ kann aus $p_2$ nur über eine $f$-Kante
                    erreicht werden.
                    Mit einer $e$-Kante ist der Übergang aus $p_1$ auf $p_2$
                    bzw. die Schleife in $p_2$ möglich.
                    Um den Zustand $p_1$ aus dem Startzustand erreichen zu
                    können, wird ebenfalls eine $e$-Kante benötigt.
                    Damit ist in $w$ die Zeichenkette $eef$ enthalten und es
                    gilt $w \in M$.

                \item $M \subseteq L(A)$ \hfill \\
                    Sei nun $w\in M$.
                    Es gibt somit eine Stelle, an der das Wort $eef$ das erste
                    Mal auftritt.
                    Das Wort $w$ lässt sich somit zerlegen in
                    $x\cdot eef \cdot x'$.
                    Da das Teilwort $eef$ zum ersten Mal auftreten muss, ist
                    dieses in $x$ nicht enthalten.
                    $A$ kann $x$ und $x'$ vollständig lesen, darf sich nach dem
                    Lesen von $x$ jedoch nicht in $p_3$ (Endzustand) befinden,
                    da $A$ sonst akzeptiert.
                    Dies würde einen Widerspruch zu der Aussage erzeugen, dass
                    $A$ nur mit dem Teilwort $eef$ akzeptieren kann.
                    $A$ kann sich nach dem Lesen von $x$ nur in den Zuständen
                    $p_0$, $p_1$ oder $p_2$ befinden.
                    Aus all diesen Zuständen ist es möglich $A$ durch das
                    Teilwort $eef$ in den Endzustand $p_3$ zu überführen.
                    Im Endzustand kann $x'$ vollständig gelesen werden.
                    Das gesamte Wort $w$ wird akzeptiert und es gilt
                    $w\in L(A)$.
            \end{enumerate}
            Aus a) und b) folgt $L(A) = M$.
        \end{proof}

    \item
        Der reguläre Ausdruck $R$ beschreibt diese Wortmenge.
        \begin{equation}
            R = (e + f)^* \cdot eef \cdot (e + f)^*
        \end{equation}
        
    \item Anwendung des eben beschriebenen Verfahrens auf $A$.
        \begin{enumerate}[1.]
            \item \hfill \\
                \begin{center}
                    \begin{tikzpicture}[->,>=stealth',shorten >=1pt,auto,node distance=1.8cm,
                                        semithick]
                      \tikzstyle{every state}=[fill=white,draw=black,text=black]
                    
                      \node[accepting,state] 	(A)              {$p_0$};
                      \node[state]         	(B) [right of=A] {$p_1$};
                      \node[state]         	(C) [right of=B] {$p_2$};
                      \node[initial right,state](D) [right of=C] {$p_3$};
                    
                      \path (A) edge [loop above] 	node {f} (A)
                                edge [bend left]  	node {e} (B)
                            (B) edge [bend left]	node {f} (A)
                                edge 				node {e} (C)
                            (C)	edge [loop above]	node {e} (C)
                                edge				node {f} (D)
                            (D)	edge [loop above]	node {e,f} (D);
                    \end{tikzpicture}
                \end{center}

            \item \hfill \\
                \begin{center}
                    \begin{tikzpicture}[<-,>=stealth',shorten >=1pt,auto,node distance=1.8cm,
                                                semithick, every loop/.style={<-}]
                        \tikzstyle{every state}=[fill=white,draw=black,text=black]
                            
                        \node[accepting,state] 	(A)              {$p_0$};
                        \node[state]         	(B) [right of=A] {$p_1$};
                        \node[state]         	(C) [right of=B] {$p_2$};
                        \node[initial right,state](D) [right of=C] {$p_3$};
                            
                        \path 	(A) edge [loop above] 	node {f} (A)
                                    edge [bend left]  	node {e} (B)
                                (B) edge [bend left]	node {f} (A)
                                    edge 				node {e} (C)
                                (C)	edge [loop above]	node {e} (C)
                                    edge				node {f} (D)
                                (D)	edge [loop above]	node {e,f} (D);
                    \end{tikzpicture}
                \end{center}

            \item \hfill \\
                \begin{center}
                    \begin{tikzpicture}[<-,>=stealth',shorten >=1pt,auto,node distance=3.4cm,
                                                        semithick, every loop/.style={<-}]
                        \tikzstyle{every state}=[fill=white,draw=black,text=black]
                                    
                        \node[accepting,state] 	(A)              {$\{p_3,p_2,p_1,p_0\}$};
                        \node[state]         	(B) [right of=A] {$\{p_3,p_2,p_1\}$};
                        \node[state]         	(C) [right of=B] {$\{p_3,p_2\}$};
                        \node[initial right,state](D) [right of=C] {$\{p_3\}$};
                                    
                        \path 	(D)	edge	[loop above]	node	{e}		(D)
                                (C)	edge					node	{f}		(D)
                                    edge	[loop above]	node	{f}		(C)
                                    edge	[bend left]		node	{f}		(B)
                                (B)	edge	[bend left]		node	{e}		(C)
                                (A)	edge	[loop above]	node	{e,f}	(A)
                                    edge					node	{e}		(B);
                    \end{tikzpicture}
                \end{center}

        \end{enumerate}

    \item Begründen Sie die Korrektheit der Lösung. \\
        \begin{behaupt}
            \begin{equation*}
                L(A') = M
            \end{equation*}
        \end{behaupt}
        \begin{proof} \hfill \\
            \begin{enumerate}
                \item $M \subseteq L(A')$ \\
                    Sei $w \in M$ ein Wort, dann hat $w$ die Form 
                    $w = \{e,f\}^* \cdot fee \cdot \{e,f\}^*$.
                    Wir müssen nun zeigen, dass $A'$ dieses Wort akzeptiert, 
                    dass also $w \in L(A')$ gilt. Beginnt $w$ mit $e$ ändert
                    sich der Zustand des Automaten nicht. Sobald ein $f$ 
                    gelesen wird, springt der Automat in den Zustand 
                    $\{p_3,p_2\}$. Falls weitere $f$ gelesen springt der
                    Automat in keinen anderen Zustand. Wenn ein $e$ gelesen wird,
                    springt der Automat in den Zustand $\{p_3,p_2,p_1\}$.
                    Werden noch ein $e$ gelesen, wechselt der Automat in den Endzustand
                    und es können nun Kombinationen aus $\{e,f\}^*$ gelesen werden,
                    ändert sich der Zustand auch nicht mehr und $w$ wird akzeptiert,
                    da es auch $fee$ enthält. Befindet sich der Automat noch im Zustand
                    $\{p_3,p_2,p_1\}$ und es wird ein $f$ gelesen, so wird in den vorherigen 
                    Zustand gesprungen. Dieser Zustandswechsel kann beliebig oft stattfinden.
                    Wenn aber $fee$ gelesen wird, gelangt der Automat in den Endzustand.
                    Daraus folgt $w \in L(A')$.
                    
                \item $L(A') \subseteq M$ \\
                    Sei $w \in L(A')$, d.\,h. ein Wort, das vom Automaten
                    akzeptiert wird.
                    Wir müssen nun zeigen, dass daraus folgt, dass $w \in M$
                    gilt.
                    Zunächst bemerken wir, dass im Startzustand beliebig viele
                    $e$s gelesen werden können und dass der Automat nur in den
                    Folgezustand überführt werden kann, wenn ein $f$ gelesen
                    wird.
                    Um den Automaten in den Endzustand überführt werden kann,
                    müssen hintereinander ein $f$ und zwei $e$s gelesen werden.
                    Ist die Kombination nicht Teil des Wortes, so kann der
                    Automat nicht in den Endzustand gelangen.
                    Da $w$ diese Kombination beinhaltet, ist gezeigt dass $w$
                    gespiegelt ein Eingabewort des Automaten $A$ ist.
            \end{enumerate}
            Aus a) und b) folgt $L(A') = M$.
            $A'$ akzeptiert die gespiegelten Wörter aus $L(A)$.
        \end{proof}

    	
\end{enumerate}


\end{document}
