\documentclass[a4paper]{scrartcl}

% font/encoding packages
\usepackage[utf8]{inputenc}
\usepackage[T1]{fontenc}
\usepackage{lmodern}
\usepackage[ngerman]{babel}

% math
\usepackage{amsmath, amssymb, amsfonts, amsthm, mathtools, MnSymbol}
\allowdisplaybreaks
\newtheorem*{behaupt}{Behauptung}

% tikz
\usepackage{tikz}
\usetikzlibrary{arrows,automata}

% misc
\usepackage{enumerate}

\title{Formale Grundlagen der Informatik II}
\subtitle{Aufgabenblatt 4}
\author{
    Jan-Hendrik Briese (6523408) \\
    Lennart Braun (6523742) \\
    Marc Strothmann (6537646) \\
    Maximilian Knapperzbusch (6535090)
}
\date{zum 10. November 2014}

\begin{document}
\maketitle

\section*{Übungsaufgabe 4.3} 
\begin{enumerate}
    \item
        \begin{align*}
            L(TS_\text{kuchen\_teil}) &= \left( r + v \left( htwb^*k \right)^* o \right)^* \\
            L^\omega(TS_\text{kuchen\_teil}) &= \left( r + v \left( htwb^*k \right)^* o \right)^\omega
        \end{align*}

    \item
        \begin{equation*}
            SS(M) = 1 \left( 1 + 3 \left( 576^+3 \right)^* 1 \right)^\omega
        \end{equation*}

    \item
        \begin{align*}
            \alpha_1 &\quad \text{Hitze} \\
            \alpha_2 &\quad \text{Teig} \\
            \alpha_3 &\quad \text{Zeit} \\
            \alpha_4 &\quad \text{Backen}
        \end{align*}
        \begin{equation*}
            \begin{split}
                E_S(SS(M)) &= E_S(1) \left( E_S(1) + E_S(3) \left( E_S(5) E_S(7) E_S(6)^* E_S(3) \right)^* E_S(1) \right)^\omega \\
                &= \emptyset \left( \emptyset + \left\{ \alpha_1 \right\} \left( \left\{ \alpha_1, \alpha_2 \right\} \left\{ \alpha_1, \alpha_4 \right\} \left\{ \alpha_1, \alpha_3, \alpha_4 \right\}^+ \left\{ \alpha_1 \right\} \right)^* \emptyset \right)^\omega
            \end{split}
        \end{equation*}

    \item
        \begin{align}
            Sat(Teig \lor \lnot Hitze) &= \left\{ 1, 4, 5 \right\} \\
            Sat(\lnot Teig) &= \left\{ 1, 2, 3, 6, 7 \right\}
        \end{align}
        

    \item
        \begin{equation}
            f = \textbf{FG}(Backen \land \textbf{X}Zeit)
        \end{equation}

        \begin{behaupt}
            \begin{equation}
                M_{kuchen} \not\models f
            \end{equation}
        \end{behaupt}

        \begin{proof}
            TODO
        \end{proof}

\end{enumerate}

\section*{Übungsaufgabe 4.4}
\begin{equation*}
    \pi = 114576(32)^\omega
\end{equation*}

\begin{table}[h]
    \centering
    \begin{tabular}{r|l|c|c}
           & $f$ & $M_{kuchen} \models  f$ & $M_{kuchen}, \pi \models  f$ \\ \hline
        1. & $\diamond\square(\lnot Teig \lor Hitze)$                     & $\bot$ & $\top$ \\
        2. & $\square\diamond(\lnot Teig \lor Hitze)$                     & $\top$ & $\top$ \\
        3. & $\square(Hitze ~\textbf{U}~ Backen)$                         & $\bot$ & $\bot$ \\
        4. & $\square\diamond(Backen \Rightarrow \circ\circ\lnot Backen)$ & $\bot$ & $\top$ \\
        5. & $\square((Hitze \land Teig) \Rightarrow \diamond\lnot Teig)$ & $\top$ & $\top$ \\
        6. & $\circ\circ\diamond(\lnot Hitze \land \lnot Teig)$           & $\bot$ & $\bot$ \\
    \end{tabular}
\end{table}

Begründungen
\begin{enumerate}
    \item $f = \diamond\square(\lnot Teig \lor Hitze)$
        \begin{description}
            \item[$M_{kuchen}, \pi \models f$] \hfill \\
                Sobald $M_{kuchen}$ in $\pi$ den Zustand $4$ verlassen hat,
                gilt immer $\lnot Teig \lor Hitze$.

            \item[$M_{kuchen} \not\models f$] \hfill \\
                Sei $\pi' = (1453)^\omega$ ein unendlicher Zustandspfad.
                Da Zustand $4$ unendlich oft durchlaufen wird, gibt es keinen
                Zeitpunkt, ab dem $\lnot Teig \lor Hitze$ immer gilt.

        \end{description}

    \item $f = \square\diamond(\lnot Teig \lor Hitze)$
        \begin{description}
            \item[$M_{kuchen}, \pi \models f$] \hfill \\
                Sobald $M_{kuchen}$ in $\pi$ den Zustand $4$ verlassen hat,
                gilt immer $\lnot Teig \lor Hitze$.

            \item[$M_{kuchen} \models f$] \hfill \\
                Da Zustand $4$ der einzige Zustand ist, indem
                $\lnot Teig \lor Hitze$ nicht gilt, muss gezeigt werden, dass
                jeder Zustandspfad $4$ wieder verlässt.
                Da $4$ keine Schleife, sondern nur eine Kante nach $5$ besitzt,
                gilt $f$.

        \end{description}

    \item $f = \square(Hitze ~\textbf{U}~ Backen)$
        \begin{description}
            \item[$M_{kuchen}, \pi \not\models f$] \hfill \\
                Da schon in $1$ $\lnot Hitze$ gilt, müsste nach $f$ $Backen$
                gelten.
                Dies ist in $1$ nicht der Fall.

            \item[$M_{kuchen} \not\models f$] \hfill \\
                Da schon $M_{kuchen}, \pi \not\models f$, muss auch
                $M_{kuchen} \not\models f$.

        \end{description}

    \item $f = \square\diamond(Backen \Rightarrow \circ\circ\lnot Backen)$
        \begin{description}
            \item[$M_{kuchen}, \pi \models f$] \hfill \\
                Mit $\pi$ befindet sich $M_{kuchen}$ unendlich oft in den
                Zuständen $2$ und $3$.
                Da in diesen immer $\lnot Backen$ gilt, ist $f$ erfüllt.

            \item[$M_{kuchen} \not\models f$] \hfill \\
                Sei $\pi' = 1457(6)^\omega$ eine unendlicher Zustandspfad.
                Da in $6$ $Backen$ gilt und $M_{kuchen}$ nie wieder aus $6$
                heraus kommt, kann bei $\pi'$ nie zwei Schritte nach $Backen$
                $\lnot Backen$ gelten.

        \end{description}

    \item $f = \square((Hitze \land Teig) \Rightarrow \diamond\lnot Teig)$
        \begin{description}
            \item[$M_{kuchen}, \pi \models f$] \hfill \\
                Siehe unten.

            \item[$M_{kuchen} \models f$] \hfill \\
                Es muss gezeigt werden, dass alle Pfade gilt: Nach einem
                Zustand, in dem $Hitze \land Teig$ gilt, muss irgendwann ein
                Zustand kommen, in dem $\lnot Teig$ gilt.
                $Hitze \land Teig$ gilt ausschließlich in Zustand $5$.
                Aus $5$ führt die einzige Kante nach $7$.
                Da in $7$ $\lnot Teig$ gilt, ist $f$ erfüllt.

        \end{description}

    \item $f = \circ\circ\diamond(\lnot Hitze \land \lnot Teig)$
        \begin{description}
            \item[$M_{kuchen}, \pi \not\models f$] \hfill \\
                Ab dem dritten Schritt, müsste irgendwann einmal
                $\lnot Hitze \land \lnot Teig$ gelten.
                Dies ist nur in Zustand $1$ der Fall.
                Da dieser in $\pi$ nur an den ersten beiden Stellen vorkommt,
                ist $f$ nicht erfüllt.

            \item[$M_{kuchen} \not\models f$] \hfill \\
                Da schon $M_{kuchen}, \pi \not\models f$, muss auch
                $M_{kuchen} \not\models f$.

        \end{description}

\end{enumerate}

\end{document}
