\documentclass[a4paper]{scrartcl}

% font/encoding packages
\usepackage[utf8]{inputenc}
\usepackage[T1]{fontenc}
\usepackage{lmodern}
\usepackage[ngerman]{babel}

% math
\usepackage{amsmath, amssymb, amsfonts, amsthm, mathtools, MnSymbol}
\allowdisplaybreaks
\newtheorem*{behaupt}{Behauptung}

% tikz
\usepackage{tikz}
\usetikzlibrary{arrows,automata}

% misc
\usepackage{enumerate}

\title{Formale Grundlagen der Informatik II}
\subtitle{Aufgabenblatt 4}
\author{
    Jan-Hendrik Briese (6523408) \\
    Lennart Braun (6523742) \\
    Marc Strothmann (6537646) \\
    Maximilian Knapperzbusch (6535090)
}
\date{zum 10. November 2014}

\begin{document}
\maketitle

\section*{Übungsaufgabe 4.3} 
\begin{enumerate}
    \item
        \begin{align*}
            L(TS_\text{kuchen\_teil}) &= \left( r + v \left( htwb^*k \right)^* o \right)^* \\
            L^\omega(TS_\text{kuchen\_teil}) &= \left( r + v \left( htwb^*k \right)^* o \right)^\omega
        \end{align*}

    \item
        \begin{equation*}
            SS(M) = 1 \left( 1 + 3 \left( 576^+3 \right)^* 1 \right)^\omega
        \end{equation*}

    \item
        \begin{align*}
            \alpha_1 &\quad \text{Hitze} \\
            \alpha_2 &\quad \text{Teig} \\
            \alpha_3 &\quad \text{Zeit} \\
            \alpha_4 &\quad \text{Backen}
        \end{align*}
        \begin{equation*}
            \begin{split}
                E_S(SS(M)) &= E_S(1) \left( E_S(1) + E_S(3) \left( E_S(5) E_S(7) E_S(6)^* E_S(3) \right)^* E_S(1) \right)^\omega \\
                &= \emptyset \left( \emptyset + \left\{ \alpha_1 \right\} \left( \left\{ \alpha_1, \alpha_2 \right\} \left\{ \alpha_1, \alpha_4 \right\} \left\{ \alpha_1, \alpha_3, \alpha_4 \right\}^+ \left\{ \alpha_1 \right\} \right)^* \emptyset \right)^\omega
            \end{split}
        \end{equation*}

    \item
        \begin{align}
            Sat(Teig \lor \lnot Hitze) &= \left\{ 1, 4, 5 \right\} \\
            Sat(\lnot Teig) &= \left\{ 1, 2, 3, 6, 7 \right\}
        \end{align}
        

    \item

\end{enumerate}

\section*{Übungsaufgabe 4.4}

\end{document}
