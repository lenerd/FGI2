\documentclass[a4paper]{scrartcl}

% font/encoding packages
\usepackage[utf8]{inputenc}
\usepackage[T1]{fontenc}
\usepackage{lmodern}
\usepackage[ngerman]{babel}

% math
\usepackage{amsmath, amssymb, amsfonts, amsthm, mathtools}
\allowdisplaybreaks
\newtheorem*{behaupt}{Behauptung}

% tikz
\usepackage{tikz}
\usetikzlibrary{arrows}
\usetikzlibrary{automata}
\usetikzlibrary{shapes}
\usetikzlibrary{petri}
\usetikzlibrary{positioning}

% misc
\usepackage{enumerate}
\usepackage[section]{placeins}
\usepackage{tabu}
\usepackage{subcaption}

% macros
\newcommand{\gdw}{\Leftrightarrow}

\title{Formale Grundlagen der Informatik II}
\subtitle{Aufgabenblatt 12}
\author{
    Jan-Hendrik Briese (6523408) \\
    Lennart Braun (6523742) \\
    Marc Strothmann (6537646) \\
    Maximilian Knapperzbusch (6535090)
}
\date{zum 19. Januar 2015}

\begin{document}
\maketitle

\section*{Übungsaufgabe 12.3} 
\begin{enumerate}
    \item

    \item

    \item

    \item

    \item

\end{enumerate}

\section*{Übungsaufgabe 12.4} 
\begin{enumerate}
    \item

    \item
        \begin{equation}
            \begin{split}
                t_7 &= (d(c + d))(ab + (a + a)(b + b)) \\
                &\stackrel{R3}{\longrightarrow} (d(c + d))(ab + \underline{a}(b + b)) \\
                &\stackrel{R3}{\longrightarrow} (d(c + d))(ab + a\underline{b}) \\
                &\stackrel{R3}{\longrightarrow} (d(c + d))\underline{ab} \\
                &\stackrel{R5}{\longrightarrow} \underline{d((c + d)ab)} \\
                &\stackrel{R4}{\longrightarrow} d(\underline{cab + dab})
            \end{split}
        \end{equation}

    \item
        \begin{equation}
            \begin{split}
                t_8 &= ((d + d)(a + a + c))(a(b + b)) \\
                &\stackrel{R3}{\longrightarrow} (\underline{d}(a + a + c))(a(b + b)) \\
                &\stackrel{R3}{\longrightarrow} (d(\underline{a} + c))(a(b + b)) \\
                &\stackrel{R3}{\longrightarrow} (d(a + c))a\underline{b} \\
                &\stackrel{R5}{\longrightarrow} \underline{d((a + c)ab)} \\
                &\stackrel{R4}{\longrightarrow} d(\underline{aab + cab})
            \end{split}
        \end{equation}

        $t_7$ und $t_8$ sind nicht äquivalent, da die Normalformen nicht gleich
        (modulo $=_{AC}$) sind.
        Es ist auch zu sehen, dass $t_7$ im Gegensatz zu $t_8$ die Aktionsfolge
        $ddab$ akzeptiert.

\end{enumerate}


\end{document}
