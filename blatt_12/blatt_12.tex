\documentclass[a4paper]{scrartcl}

% font/encoding packages
\usepackage[utf8]{inputenc}
\usepackage[T1]{fontenc}
\usepackage{lmodern}
\usepackage[ngerman]{babel}

% math
\usepackage{amsmath, amssymb, amsfonts, amsthm, mathtools}
\allowdisplaybreaks
\newtheorem*{behaupt}{Behauptung}

% tikz
\usepackage{tikz}
\usetikzlibrary{arrows}
\usetikzlibrary{automata}
\usetikzlibrary{shapes}
\usetikzlibrary{petri}
\usetikzlibrary{positioning}

% misc
\usepackage{enumerate}
\usepackage[section]{placeins}
\usepackage{tabu}
\usepackage{subcaption}

% macros
\newcommand{\gdw}{\Leftrightarrow}
\newcommand{\too}{\longrightarrow}

\title{Formale Grundlagen der Informatik II}
\subtitle{Aufgabenblatt 12}
\author{
    Jan-Hendrik Briese (6523408) \\
    Lennart Braun (6523742) \\
    Marc Strothmann (6537646) \\
    Maximilian Knapperzbusch (6535090)
}
\date{zum 19. Januar 2015}

\begin{document}
\maketitle

\section*{Übungsaufgabe 12.3} 
\begin{align}
    t_3 &= a \cdot (bb + a) \cdot b + a \cdot (b + (b + bb)) \\
    t_4 &= a \cdot (bbb + (ab + b) \cdot (b + b))
\end{align}

\begin{enumerate}
    \item
        \begin{figure}[h]
            \centering
            \begin{subfigure}{0.45\textwidth}
                \begin{tikzpicture}[
                        auto,
                        scale=2,
                    ]
                    \tikzstyle{edge}=[
                        ->,
                        >=stealth',
                    ]
                    \node (a) at (1,4) {$a \cdot (bb + a) \cdot b + a \cdot (b + (b + bb))$};
                    \node (b) at (0,3) {$(bb + a) \cdot b$};
                    \node (c) at (2,3) {$b + (b + bb)$};
                    \node (d) at (0,2) {$bb$};
                    \node (e) at (1,1) {$b$};
                    \node (f) at (1,0) {$\checkmark$};

                    \draw [edge] (a) to node {$a$} (b);
                    \draw [edge] (a) to node {$a$} (c);
                    \draw [edge, bend left] (b) to node {$a$} (e);
                    \draw [edge] (b) to node {$b$} (d);
                    \draw [edge] (d) to node {$b$} (e);
                    \draw [edge] (c) to node {$b$} (e);
                    \draw [edge, bend left] (c) to node {$b$} (f);
                    \draw [edge] (e) to node {$b$} (f);
                \end{tikzpicture}
                \caption{Prozessgraph von $t_3$}
                \label{fig:t3}
            \end{subfigure}
            \begin{subfigure}{0.45\textwidth}
                \centering
                \begin{tikzpicture}[
                        auto,
                        scale=2,
                    ]
                    \tikzstyle{edge}=[
                        ->,
                        >=stealth',
                    ]
                    \node (a) at (1,4) {$a \cdot (bbb + (ab + b) \cdot (b + b))$};
                    \node (b) at (1,3) {$bbb + (ab + b) \cdot (b + b)$};
                    \node (c) at (0,2) {$bb$};
                    \node (d) at (1,2) {$b(b + b)$};
                    \node (e) at (2,1) {$b + b$};
                    \node (f) at (0,1) {$b$};
                    \node (g) at (1,0) {$\checkmark$};

                    \draw [edge] (a) to node {$a$} (b);
                    \draw [edge] (b) to node {$b$} (c);
                    \draw [edge] (b) to node {$a$} (d);
                    \draw [edge, bend left] (b) to node {$b$} (e);
                    \draw [edge] (d) to node {$b$} (e);
                    \draw [edge] (c) to node {$b$} (f);
                    \draw [edge] (e) to node {$b$} (g);
                    \draw [edge] (f) to node {$b$} (g);
                \end{tikzpicture}
                \caption{Prozessgraph von $t_4$}
                \label{fig:t4}
            \end{subfigure}
            \caption{}
        \end{figure}

    \item
        \begin{equation}
            \begin{split}
                \checkmark \ &\mathcal{B}\  \checkmark \\
                b \ &\mathcal{B}\  b \\
                b \ &\mathcal{B}\  (b + b) \\
                bb \ &\mathcal{B}\  bb \\
                bb \ &\mathcal{B}\  b(b + b)
            \end{split}
        \end{equation}

    \item
        $t_3$ und $t_4$ sind nicht bisimilar.

    \item
        \begin{enumerate}
            \item
                \begin{align}
                        a &\stackrel{a}{\too} \checkmark & &\frac{}{v \stackrel{v}{\too} \checkmark} \\
                        bb + a &\stackrel{a}{\too} \checkmark & &\frac{y \stackrel{v}{\too} \checkmark}{x + y \stackrel{v}{\too} \checkmark} \\
                        (bb + a) \cdot b &\stackrel{a}{\too} b & &\frac{x \stackrel{v}{\too} \checkmark}{x \cdot y \stackrel{v}{\too} y}
                \end{align}

            \item
                \begin{align}
                        a &\stackrel{a}{\too} \checkmark & &\frac{}{v \stackrel{v}{\too} \checkmark} \\
                        a \cdot (b + (b + bb) &\stackrel{a}{\too} b + (b + bb) & &\frac{x \stackrel{v}{\too} \checkmark}{x \cdot y \stackrel{v}{\too} y} \\
                        a(bb + a)b + a \cdot (b + (b + bb) &\stackrel{a}{\too} b + (b + bb) & &\frac{y \stackrel{v}{\too} y'}{x + y \stackrel{v}{\too} y'}
                \end{align}

            \item
                \begin{align}
                        a &\stackrel{a}{\too} \checkmark & &\frac{}{v \stackrel{v}{\too} \checkmark} \\
                        a \cdot (bb + a) \cdot b &\stackrel{a}{\too} (bb + a) \cdot b & &\frac{x \stackrel{v}{\too} \checkmark}{x \cdot y \stackrel{v}{\too} y} \\
                        a \cdot (bb + a) \cdot b + a(b + (b + bb)) &\stackrel{a}{\too} (bb + a) \cdot b & &\frac{x \stackrel{v}{\too} x'}{x + y \stackrel{v}{\too} x'}
                \end{align}

        \end{enumerate}
        
    \item
        \begin{align}
            t_5 &= b(cf + df) + a(c + d)f \\
            t_6 &= (a + b)(c + d)f
        \end{align}

        \begin{figure}[h]
            \centering
            \begin{subfigure}{0.45\textwidth}
                \begin{tikzpicture}[
                        auto,
                        scale=2,
                    ]
                    \tikzstyle{edge}=[
                        ->,
                        >=stealth',
                    ]
                    \node (a) at (1,3) {$b(cf + df) + a(c + d)f$};
                    \node (b) at (0,2) {$cf + df$};
                    \node (c) at (2,2) {$(c + d)f$};
                    \node (d) at (1,1) {$f$};
                    \node (e) at (1,0) {$\checkmark$};

                    \draw [edge] (a) to node {$b$} (b);
                    \draw [edge] (a) to node {$a$} (c);
                    \draw [edge, bend right] (b) to node {$c$} (d);
                    \draw [edge, bend left] (b) to node {$d$} (d);
                    \draw [edge, bend right] (c) to node {$c$} (d);
                    \draw [edge, bend left] (c) to node {$d$} (d);
                    \draw [edge] (d) to node {$f$} (e);
                \end{tikzpicture}
                \caption{Prozessgraph von $t_5$}
                \label{fig:t5}
            \end{subfigure}
            \begin{subfigure}{0.45\textwidth}
                \centering
                \begin{tikzpicture}[
                        auto,
                        scale=2,
                    ]
                    \tikzstyle{edge}=[
                        ->,
                        >=stealth',
                    ]
                    \node (a) at (0,3) {$(a + b)(c + d)f$};
                    \node (b) at (0,2) {$(c + d)f$};
                    \node (c) at (0,1) {$f$};
                    \node (d) at (0,0) {$\checkmark$};

                    \draw [edge, bend right] (a) to node {$a$} (b);
                    \draw [edge, bend left] (a) to node {$b$} (b);
                    \draw [edge, bend right] (b) to node {$c$} (c);
                    \draw [edge, bend left] (b) to node {$d$} (c);
                    \draw [edge] (c) to node {$f$} (d);
                \end{tikzpicture}
                \caption{Prozessgraph von $t_6$}
                \label{fig:t6}
            \end{subfigure}
            \caption{}
        \end{figure}
        \begin{equation}
            \begin{split}
                \checkmark \ &\mathcal{B}\  \checkmark \\
                f \ &\mathcal{B}\  f \\
                cf + df \ &\mathcal{B}\  (c + d)f \\
                (c + d)f \ &\mathcal{B}\  (c + d)f \\
                b(cf + df) + a(c + d)f \ &\mathcal{B}\  (a + b)(c + d)f
            \end{split}
        \end{equation}

\end{enumerate}

\section*{Übungsaufgabe 12.4} 
\begin{enumerate}
    \item
        \begin{align}
                a &\stackrel{a}{\too} \checkmark & &\frac{}{v \stackrel{v}{\too} \checkmark} \\
                a \cdot (c + f) \cdot f &\stackrel{a}{\too} (c + f) \cdot f & &\frac{x \stackrel{v}{\too} \checkmark}{x \cdot y \stackrel{v}{\too} y} \\
                b \cdot (cf + df) + a \cdot (c + f) \cdot f &\stackrel{a}{\too} (c + f) \cdot f & &\frac{y \stackrel{v}{\too} y'}{x + y \stackrel{v}{\too} y'}
        \end{align}

    \item
        \begin{equation}
            \begin{split}
                t_7 &= (d(c + d))(ab + (a + a)(b + b)) \\
                &\stackrel{R3}{\longrightarrow} (d(c + d))(ab + \underline{a}(b + b)) \\
                &\stackrel{R3}{\longrightarrow} (d(c + d))(ab + a\underline{b}) \\
                &\stackrel{R3}{\longrightarrow} (d(c + d))\underline{ab} \\
                &\stackrel{R5}{\longrightarrow} \underline{d((c + d)ab)} \\
                &\stackrel{R4}{\longrightarrow} d(\underline{cab + dab})
            \end{split}
        \end{equation}

    \item
        \begin{equation}
            \begin{split}
                t_8 &= ((d + d)(a + a + c))(a(b + b)) \\
                &\stackrel{R3}{\longrightarrow} (\underline{d}(a + a + c))(a(b + b)) \\
                &\stackrel{R3}{\longrightarrow} (d(\underline{a} + c))(a(b + b)) \\
                &\stackrel{R3}{\longrightarrow} (d(a + c))a\underline{b} \\
                &\stackrel{R5}{\longrightarrow} \underline{d((a + c)ab)} \\
                &\stackrel{R4}{\longrightarrow} d(\underline{aab + cab})
            \end{split}
        \end{equation}

        $t_7$ und $t_8$ sind nicht äquivalent, da die Normalformen nicht gleich
        (modulo $=_{AC}$) sind.
        Es ist auch zu sehen, dass $t_7$ im Gegensatz zu $t_8$ die Aktionsfolge
        $ddab$ akzeptiert.

\end{enumerate}


\end{document}
