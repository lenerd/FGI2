\documentclass[a4paper]{scrartcl}

% font/encoding packages
\usepackage[utf8]{inputenc}
\usepackage[T1]{fontenc}
\usepackage{lmodern}
\usepackage[ngerman]{babel}

% math
\usepackage{amsmath, amssymb, amsfonts, amsthm, mathtools}
\allowdisplaybreaks
\newtheorem*{behaupt}{Behauptung}

% tikz
\usepackage{tikz}
\usetikzlibrary{arrows,automata}

% misc
\usepackage{enumerate}

\title{Formale Grundlagen der Informatik II}
\subtitle{Aufgabenblatt 2}
\author{
    Jan-Hendrik Briese (6523408) \\
    Lennart Braun (6523742) \\
    Marc Strothmann (6537646) \\
    Maximilian Knapperzbusch (6535090)
}
\date{zum 27. Oktober 2014}

\begin{document}
\maketitle

\section*{Übungsaufgabe 2.3}
\begin{enumerate}
		\item[zu 1.:] \textbf{Lösung:}\\
		\begin{align*}
            L(A_{2.3}) &= a(ba^*c)^*+bc(abc)^*(e+a) \\
            L(A_{2.3})^{\omega} &= a(ba^*c)^{\omega}+b(cab)^{\omega} \\
            (L(A_{2.3}))^{\omega} &= (a(ba^*c)^*+bc(abc)^*(e+a))^{\omega} \\
		\end{align*}

		\item[zu 2.:] \textbf{Lösung:}\\
		$L^{\omega}(A_{2.3})$ ist die Sprache, die von einem Büchi-Automaten mit
        gleicher Konstruktion wie der vorliegende NFA akzeptiert wird.
        Zwei Wörter sind $w_1=a(bc)^{\omega}$ und $w_2=b(cab)^{\omega}$.

		$(L(A_{2.3}))^{\omega}$ ist eine Sprache, dessen Teilwörter akzeptierte
        Wörter des NFA sind. Diese Teilwörter bilden (konkateniert) wiederum
        $\omega$-Wörter der genannten Sprache.
        Zwei dieser Wörter sind u.A. $w_3=(bca)^{\omega}$ oder $w_4=(bce)^{\omega}$.

		\item[zu 3.:] \textbf{Lösung:}\\
			\begin{equation*}
				\begin{tikzpicture}[->,>=stealth',shorten >=1pt,auto,node distance=3.5cm,
				                    semithick]
				  \tikzstyle{every state}=[fill=white,draw=black,text=black]
				
				  \node[initial,accepting,state] 	(A)              {$q_0$};
				  \node[state]         	(B) [right of=A] {$q_3$};
				  \node[state]         	(C) [right of=B] {$q_4$};
				  \node[state](D) [below of=C] {$q_5$};
				  \node[state]			(E) [below of=A] {$q_1$};
				  \node[state]			(F)	[below of=B] {$q_2$};
				
				  \path (A) edge [loop above] 	node {a} (A)
				            edge 				node {b} (B)
				        (B) edge				node {c} (C)
				        (C) edge 				node {a} (D)
				        	edge [bend right]	node {a,e} (A)
				        (D) edge				node {b} (B)
				        (A) edge				node {a} (E)
				        (E) edge [bend left]	node {b} (F)
				        (F) edge [bend left]	node {c} (E)
				        	edge				node {c} (A)
				        	edge [loop below]	node {a} (F);
				\end{tikzpicture}
			\end{equation*}
			\textbf{Konstruktionsverfahren:}\\
			Mit dem folgenden Verfahren wird aus einem NFA $A$ ein nicht-deterministischer Büchi-Automat $A'$ konstruiert, der die Sprache $(L(A))^{\omega}$ akzeptiert.\\
			Sei $A=(Q,\Sigma,\delta,Q_0,F)$ ein gegebener NFA.\\
			Jeder ursprüngliche Startzustand in $A$ wird nun zu einem Start- und Endzustand in $A'$ ($Q_0=F'$). Alle Kanten, die $A$ aus $q_k\in Q$in einen Endzustand überführt haben, werden in $A'$ kopiert und bilden eine neue Kante von $q'_k$ in die konstruierten Endzustände in $A'$.\\
			$\delta'=\delta \cup \{(q_k, a, q_l)|(q_k, a, q_f)\in\delta, q_f\in F, q_l \in Q_0, q_k \in Q \}$\\
			Endzustände in $A$, die keinen Folgezustand besitzen, können in $A'$ weggelassen werden. $Q'=Q\backslash\{q_l|\nexists(q_l,a,q_k)\}$ mit $q_k\in Q$ und $q_l\in F$. Der konstruierte Automat $A'=(Q',\Sigma',\delta',Q_0',F')$ ist ein Büchi-Automat mit Omega-Abschluss.\\
\end{enumerate}

\section*{Übungsaufgabe 2.4}
\begin{enumerate}
    \item

    \item
        Sei $R_W$ ein regulärer Ausdruck, der die Menge $W$ beschreibt.
        Sei $R_U$ ein $\omega$-regulärer Ausdruck, der die Menge $U$ beschreibt.
        Es gelten also $M_{R_W} = W$ und $M_{R_U} = U$.
        Nach den Definitionen 1.6 und 1.17 ist $R_W \cdot R_U$ ein
        $\omega$-regulärer Ausdruck, der die Menge $M_{R_W} \cdot M_{R_U} = W \cdot U$
        beschreibt.

    \item
        $R_U$ hat die Form $R_U = A_1 B_1^\omega + \cdots + A_n B_n^\omega$.
        \begin{equation}
            \begin{split}
                R_W \cdot R_U
                &= R_W \cdot \left( A_1 B_1^\omega + \cdots + A_n B_n^\omega \right) \\
                &= A_1' B_1^\omega + \cdots + A_n' B_n^\omega \text{ mit } A_i' = R_W \cdot A_i
            \end{split}
        \end{equation}
        $R_W \cdot R_U$ ist also ebenfalls ein wohlgeformter $\omega$-regulärer Ausdruck.
        Nach 1.6 beschreibt $R_w \cdot R_U$ die gesuchte Menge
        $M_{R_W \cdot R_U} = M_{R_W} \cdot R_{R_U} = W \cdot U$.

        Das Verfahren besteht aus endlich vielen Schritten und terminiert daher immer.

    \item
        \begin{align}
            R_{A_W} &= \left( e + f \right)^* \cdot eef \cdot \left( e + f \right)^* \\
            R_{A_U} &= a \cdot \left( ba^*c \right)^\omega + b \cdot \left( cab \right)^\omega \\
            \begin{split}
                R_{A_w \cdot A_U}
                &= \left( \left( e + f \right)^* \cdot eef \cdot \left( e + f \right)^* \right)
                \cdot \left( a \cdot \left( ba^*c \right)^\omega + b \cdot \left( cab \right)^\omega \right) \\
                &= \underbrace{\left( e + f \right)^* \cdot eef \cdot \left( e + f \right)^* \cdot a}_{A_1}
                \cdot \underbrace{\left( ba^*c \right)^\omega}_{B_1^\omega}
                + \underbrace{\left( e + f \right)^* \cdot eef \cdot \left( e + f \right)^* \cdot b}_{A_2}
                \cdot \underbrace{\left( cab \right)^\omega}_{B_2^\omega} \\
                &= A_1B_1^\omega + A_2B_2^\omega
            \end{split}
        \end{align}
        
\end{enumerate}


\end{document}
