\documentclass[a4paper]{scrartcl}

% font/encoding packages
\usepackage[utf8]{inputenc}
\usepackage[T1]{fontenc}
\usepackage{lmodern}
\usepackage[ngerman]{babel}

% math
\usepackage{amsmath, amssymb, amsfonts, amsthm, mathtools, MnSymbol}
\allowdisplaybreaks
\newtheorem*{behaupt}{Behauptung}

% tikz
\usepackage{tikz}
\usetikzlibrary{arrows,automata}

% misc
\usepackage{enumerate}
\usepackage[section]{placeins}
\usepackage{subcaption}

\title{Formale Grundlagen der Informatik II}
\subtitle{Aufgabenblatt 6}
\author{
    Jan-Hendrik Briese (6523408) \\
    Lennart Braun (6523742) \\
    Marc Strothmann (6537646) \\
    Maximilian Knapperzbusch (6535090)
}
\date{zum 24. November 2014}

\begin{document}
\maketitle

\section*{Übungsaufgabe 6.3} 
\begin{enumerate}
    \item
        \begin{align*}
            VC(\phi_{01}) &= (1, 0, 0, 0) \\
            VC(\phi_{02}) &= (2, 0, 0, 0) \\
            VC(\phi_{03}) &= (3, 0, 1, 3) \\
            VC(\phi_{04}) &= (4, 0, 1, 3) \\
            VC(\phi_{05}) &= (5, 3, 5, 4) \\
            VC(\phi_{06}) &= (6, 3, 5, 4)
        \end{align*}
        \begin{align*}
            VC(\phi_{11}) &= (1, 1, 0, 0) \\
            VC(\phi_{12}) &= (1, 2, 0, 1) \\
            VC(\phi_{13}) &= (1, 3, 0, 1) \\
            VC(\phi_{14}) &= (6, 4, 5, 4)
        \end{align*}
        \begin{align*}
            VC(\phi_{21}) &= (0, 0, 1, 0) \\
            VC(\phi_{22}) &= (2, 0, 2, 0) \\
            VC(\phi_{23}) &= (2, 0, 3, 4) \\
            VC(\phi_{24}) &= (2, 3, 4, 4) \\
            VC(\phi_{25}) &= (2, 3, 5, 4) \\
            VC(\phi_{26}) &= (4, 3, 6, 6)
        \end{align*}
        \begin{align*}
            VC(\phi_{31}) &= (0, 0, 0, 1) \\
            VC(\phi_{32}) &= (0, 0, 1, 2) \\
            VC(\phi_{33}) &= (0, 0, 1, 3) \\
            VC(\phi_{34}) &= (0, 0, 1, 4) \\
            VC(\phi_{35}) &= (4, 0, 1, 5) \\
            VC(\phi_{36}) &= (4, 0, 1, 6)
        \end{align*}

    \item
        \begin{equation*}
            \phi_{21} \textbf{ vor } \phi_{32} \textbf{ vor } \phi_{03} \textbf{ vor } \phi_{14}
        \end{equation*}

    \item $\phi_{02}$, $\phi_{11}$, $\phi_{21}$, $\phi_{31}$
        \begin{gather*}
            \phi_{02} \ ||\  \phi_{11} \\
            \phi_{02} \ ||\  \phi_{21} \\
            \phi_{02} \ ||\  \phi_{31} \\
            \phi_{11} \ ||\  \phi_{21} \\
            \phi_{11} \ ||\  \phi_{31} \\
            \phi_{21} \ ||\  \phi_{31}
        \end{gather*}

\end{enumerate}

\end{document}
