\documentclass[a4paper]{scrartcl}

% font/encoding packages
\usepackage[utf8]{inputenc}
\usepackage[T1]{fontenc}
\usepackage{lmodern}
\usepackage[ngerman]{babel}

% math
\usepackage{amsmath, amssymb, amsfonts, amsthm, mathtools}
\allowdisplaybreaks
\newtheorem*{behaupt}{Behauptung}

% tikz
\usepackage{tikz}
\usetikzlibrary{arrows}
\usetikzlibrary{automata}
\usetikzlibrary{shapes}
\usetikzlibrary{petri}
\usetikzlibrary{positioning}

% misc
\usepackage{enumerate}
\usepackage[section]{placeins}
\usepackage{tabu}
\usepackage{subcaption}

% macros
\newcommand{\gdw}{\Leftrightarrow}

\title{Formale Grundlagen der Informatik II}
\subtitle{Aufgabenblatt 11}
\author{
    Jan-Hendrik Briese (6523408) \\
    Lennart Braun (6523742) \\
    Marc Strothmann (6537646) \\
    Maximilian Knapperzbusch (6535090)
}
\date{zum 12. Januar 2015}

\begin{document}
\maketitle

\section*{Übungsaufgabe 11.3} 
\begin{enumerate}
    \item

    \item
        \begin{behaupt}
            Das P/T-Netz $N = ( \{ p \}, \emptyset, \emptyset, p)$ ist das
            kleinste korrekte Workflownetz.
        \end{behaupt}
        \begin{proof} \hfill \\
            \begin{enumerate}[I.]
                \item $N$ ist ein Workflownetz
                    \begin{enumerate}[a)]
                        \item
                            Für die besonderen Plätze $a, e$ mit
                            $^\bullet a = e^\bullet = \emptyset$
                            gilt $a = e = p$.

                        \item
                            Es gibt keine Transitionen und außer $p$ keinen
                            weiteren Platz.
                            Daher liegen alle Transitionen und Plätze auf
                            Pfaden zwischen $a$ und $e$.

                        \item
                            In der Anfangsmarkierung $\textbf{m}_a = p$ ist
                            ausschließlich $a$ mit einer Marke belegt.
                            Es gilt $\textbf{m}_a = \textbf{m}_e$.

                    \end{enumerate}

                \item $N$ ist ein korrektes Workflownetz
                    \begin{enumerate}[a)]
                        \item
                            Da es keine Transitionen gibt, gilt
                            $\textbf{R}(N) = \{ \textbf{m}_a = p \}$.
                            Für $p$ gibt es die Transitionsfolge
                            $\varepsilon \in T^*$ mit
                            $p \stackrel{\varepsilon}{\to} p = \textbf{m}_e$
                            (Termination möglich)

                        \item
                            In der einzigen erreichbaren Markierung $p$ (s.\,o.)
                            ist $e = p$ markiert und es gilt $p = \textbf{m}_e$.
                            (korrekte Termination)

                        \item
                            Da es keine Transitionen gibt, sind alle
                            aktivierbar. (Nützlichkeit)

                    \end{enumerate}

            \end{enumerate}
        \end{proof}

    \item
        siehe \emph{11-3-3.rnw}

    \item
        siehe Tabelle \ref{tab:11-3-3} auf Seite \pageref{tab:11-3-3}

        \begin{table}
            \centering
            \begin{tabu}{|r|l|l|}
                \hline
                Nr. & Transformationsregel & Hinzugekommene Elemente \\
                \hline
                 0 & Start mit trivialem Netz & $a$ \\
                \tabucline [on 2pt]{-}
                 1 & P-Seq & $p_2, t_1$ \\
                 2 & P-Seq & $p_{17}, t_7$ \\
                 3 & Par& $p_6$ \\
                 4 & P-Seq & $p_3, t_2$ \\
                \tabucline [on 2pt]{-}
                 5 & P-Seq & $p_8, t_4$ \\
                 6 & P-Seq & $p_{12}, t_6$ \\
                 7 & P-Seq & $p_{14}, t_8$ \\
                 8 & P-Seq & $p_{16}, t_9$ \\
                \tabucline [on 2pt]{-}
                 9 & Par& $p_5$ \\
                10 & P-Seq & $p_4, t_3$ \\
                11 & Par& $p_7$ \\
                12 & P-Seq & $p_{10}, t_5$ \\
                \tabucline [on 2pt]{-}
                13 & Par& $p_9$ \\
                14 & Par& $p_{11}$ \\
                15 & Par& $p_{13}$ \\
                16 & Par& $p_{15}$ \\
                \hline
            \end{tabu}
            \caption{Konstruktion von $N_{11.3.3}$}
            \label{tab:11-3-3}
        \end{table}

    \item
        siehe \emph{11-3-5.rnw}

    \item
        siehe Tabelle \ref{tab:11-3-5} auf Seite \pageref{tab:11-3-5}

        \begin{table}
            \centering
            \begin{tabu}{|r|X|X|X|}
                \hline
                Nr. & Transformationsregel & Verbliebene Knoten & Entfernte Knoten \\
                \hline
                0 & Start mit $N_{11.3.3}$ &
                $p_{17}$, $p_{18}$, $p_{19}$, $p_{20}$, $p_{21}$, $p_{22}$, $p_{23}$, $p_{24}$, $p_{25}$, $p_{26}$, $p_{27}$, $p_{28}$, $p_{29}$, $p_{30}$, $p_{31}$, $p_{32}$, $p_{33}$, $p_{34}$, $p_{35}$, $p_{36}$, $e$,
                $t_{10}$, $t_{11}$, $t_{12}$, $t_{13}$, $t_{14}$, $t_{15}$, $t_{16}$, $t_{17}$, $t_{18}$, $t_{19}$, $t_{20}$, $t_{21}$, $t_{22}$, $t_{23}$, $t_{24}$, $t_{25}$, $t_{26}$, $t_{27}$,  $t_{28}$
                & \\ \hline
                1 & Zusammenfassung gleicher Prozessabschnitte (Alt) &
                $p_{17}$, $p_{18}$, $p_{19}$, $p_{20}$, $p_{22}$, $p_{23}$, $p_{24}$, $p_{25}$, $p_{26}$, $p_{31}$, $p_{32}$, $p_{33}$, $p_{34}$, $p_{35}$, $p_{36}$, $e$,
                $t_{10}$, $t_{11}$, $t_{12}$, $t_{14}$, $t_{15}$, $t_{16}$, $t_{17}$, $t_{21}$, $t_{22}$, $t_{23}$, $t_{25}$, $t_{26}$, $t_{27}$,  $t_{28}$
                & 
                $p_{21}$, $p_{27}$, $p_{28}$, $p_{29}$, $p_{30}$,
                $t_{13}$, $t_{18}$, $t_{19}$, $t_{20}$, $t_{24}$
                \\ \hline
                2 & P-Seq/T-Seq &
                $p_{17}$, $p_{18}$, $p_{19}$, $p_{20}$, $p_{22}$, $p_{23}$, $p_{24}$, $p_{25}$, $p_{26}$, $p_{31}$, $p_{32}$, $p_{33}$, $p_{34}$, $p_{36}$, $e$,
                $t_{10}$, $t_{11}$, $t_{12}$, $t_{14}$, $t_{15}$, $t_{16}$, $t_{17}$, $t_{21}$, $t_{22}$, $t_{23}$, $t_{26}$, $t_{27}$,  $t_{28}$
                & 
                $p_{35}$,
                $t_{25}$
                \\ \hline
                3 & P-Seq/T-Seq &
                $p_{17}$, $p_{18}$, $p_{19}$, $p_{22}$, $p_{23}$, $p_{24}$, $p_{25}$, $p_{26}$, $p_{31}$, $p_{32}$, $p_{33}$, $p_{34}$, $p_{36}$, $e$,
                $t_{10}$, $t_{11}$, $t_{14}$, $t_{15}$, $t_{16}$, $t_{17}$, $t_{21}$, $t_{22}$, $t_{23}$, $t_{26}$, $t_{27}$,  $t_{28}$
                & 
                $p_{20}$,
                $t_{12}$
                \\ \hline
                4 & P-Seq/T-Seq &
                $p_{17}$, $p_{18}$, $p_{19}$, $p_{23}$, $p_{24}$, $p_{25}$, $p_{26}$, $p_{31}$, $p_{32}$, $p_{33}$, $p_{34}$, $p_{36}$, $e$,
                $t_{10}$, $t_{11}$, $t_{15}$, $t_{16}$, $t_{17}$, $t_{21}$, $t_{22}$, $t_{23}$, $t_{26}$, $t_{27}$,  $t_{28}$
                & 
                $p_{22}$,
                $t_{14}$
                \\ \hline
                5 & Zusammenfassung gleicher Prozessabschnitte (Alt) &
                $p_{17}$, $p_{19}$, $ p_{23}$, $p_{24}$, $p_{25}$, $p_{26}$, $p_{36}$, $e$,
                $t_{11}$, $t_{15}$, $t_{16}$, $t_{17}$, $t_{27}$,  $t_{28}$
                & 
                $p_{18}$, $p_{31}$, $p_{32}$, $p_{33}$, $p_{34}$, 
                $t_{10}$, $t_{21}$, $t_{22}$, $t_{23}$, $t_{26}$
                \\ \hline
                6 & P-Seq &
                $p_{17}$, $p_{19}$, $p_{24}$, $p_{25}$, $p_{26}$, $p_{36}$, $e$,
                $t_{11}$, $t_{15}$, $t_{17}$, $t_{27}$, $t_{28}$
                & 
                $p_{23}$,
                $t_{16}$
                \\ \hline
                7 & Par &
                $p_{17}$, $p_{19}$, $p_{24}$, $p_{26}$, $p_{36}$, $e$,
                $t_{11}$, $t_{15}$, $t_{17}$, $t_{27}$, $t_{28}$
                & 
                $p_{25}$
                \\ \hline
                8 & P-Seq/T-Seq &
                $p_{17}$, $p_{19}$, $p_{24}$, $p_{36}$, $e$,
                $t_{11}$, $t_{15}$, $t_{27}$, $t_{28}$
                & 
                $p_{26}$,
                $t_{17}$
                \\ \hline
                9 & P-Seq/T-Seq &
                $p_{17}$, $p_{19}$, $p_{36}$, $e$,
                $t_{11}$, $t_{27}$, $t_{28}$
                & 
                $p_{24}$,
                $t_{15}$
                \\ \hline
                10 & P-Seq/T-Seq &
                $p_{17}$, $p_{36}$, $e$,
                $t_{11}$, $t_{28}$
                & 
                $p_{19}$,
                $t_{27}$
                \\ \hline
                11 & P-Seq/T-Seq &
                $p_{17}$, $e$,
                $t_{28}$
                & 
                $p_{36}$,
                $t_{11}$
                \\ \hline
                11 & P-Seq &
                $p_{17}$
                & 
                $e$,
                $t_{28}$
                \\ \hline
            \end{tabu}
            \caption{Analyse von $N_{11.3.5}$}
            \label{tab:11-3-5}
        \end{table}

    \item

    \item

    \item

\end{enumerate}


\end{document}
