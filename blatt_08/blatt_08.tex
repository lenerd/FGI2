\documentclass[a4paper]{scrartcl}

% font/encoding packages
\usepackage[utf8]{inputenc}
\usepackage[T1]{fontenc}
\usepackage{lmodern}
\usepackage[ngerman]{babel}

% math
\usepackage{amsmath, amssymb, amsfonts, amsthm, mathtools}
\allowdisplaybreaks
\newtheorem*{behaupt}{Behauptung}

% tikz
\usepackage{tikz}
\usetikzlibrary{arrows}
\usetikzlibrary{automata}
\usetikzlibrary{petri}
\usetikzlibrary{positioning}

% misc
\usepackage{enumerate}
\usepackage[section]{placeins}
\usepackage{subcaption}

% macros
\newcommand{\gdw}{\Leftrightarrow}

\title{Formale Grundlagen der Informatik II}
\subtitle{Aufgabenblatt 7}
\author{
    Jan-Hendrik Briese (6523408) \\
    Lennart Braun (6523742) \\
    Marc Strothmann (6537646) \\
    Maximilian Knapperzbusch (6535090)
}
\date{zum 8. Dezember 2014}

\begin{document}
\maketitle

\section*{Übungsaufgabe 8.3} 
\begin{enumerate}
    \item foo
        \begin{figure}[h]
            \centering
            \begin{tikzpicture}
                \tikzstyle{place}=[circle, thick, draw, minimum size=6mm]
                \tikzstyle{transition}=[rectangle, thick, draw, minimum size=4mm]

                \node [place] (b1) [label=below:$p_1|b_1$,] {};
                \node [transition] (ta1) [right= of b1] {$a$};
                \node [place] (b3) [label=below:$p_3|b_3$, above right= of ta1] {};
                \node [place] (b2) [label=above:$p_2|b_2$, above= of b3] {};
                \node [place] (b4) [label=above:$p_3|b_4$, below right= of ta1] {};
                \node [place] (b5) [label=below:$p_4|b_5$, below= of b4] {};

                \node [transition] (tb1) [right= of b3] {$b$};
                \node [transition] (tc1) [right= of b4] {$c$};

                \node [place] (b6) [label=below:$p_5|b_6$, right= of tb1] {};
                \node [place] (b7) [label=below:$p_5|b_7$, right= of tc1] {};

                \node [transition] (td1) [right= of b6] {$d$};
                \node [transition] (td2) [right= of b7] {$d$};

                \node [place] (b8) [label=below:$p_6|b_8$, right= of td1] {};
                \node [place] (b9) [label=below:$p_6|b_9$, right= of td2] {};

                \draw [pre]  (ta1) to (b1);
                \draw [post] (ta1) to (b2);
                \draw [post] (ta1) to (b3);
                \draw [post] (ta1) to (b4);
                \draw [post] (ta1) to (b5);

                \draw [pre]  (tb1) to (b2);
                \draw [pre]  (tb1) to (b3);
                \draw [post] (tb1) to (b6);

                \draw [pre]  (tc1) to (b4);
                \draw [pre]  (tc1) to (b5);
                \draw [post] (tc1) to (b7);

                \draw [pre]  (td1) to (b6);
                \draw [post] (td1) to (b8);

                \draw [pre]  (td2) to (b7);
                \draw [post] (td2) to (b9);
            \end{tikzpicture}
            \caption{Prozesse von $N_{8.3}$}
            \label{fig:prozesse}
        \end{figure}

    \item

    \item

\end{enumerate}

\section*{Übungsaufgabe 8.4} 

\section*{Übungsaufgabe 8.5} 
\begin{enumerate}
    \item

    \item

    \item

    \item

\end{enumerate}

\end{document}
