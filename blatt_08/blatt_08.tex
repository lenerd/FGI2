\documentclass[a4paper]{scrartcl}

% font/encoding packages
\usepackage[utf8]{inputenc}
\usepackage[T1]{fontenc}
\usepackage{lmodern}
\usepackage[ngerman]{babel}

% math
\usepackage{amsmath, amssymb, amsfonts, amsthm, mathtools}
\allowdisplaybreaks
\newtheorem*{behaupt}{Behauptung}

% tikz
\usepackage{tikz}
\usetikzlibrary{arrows}
\usetikzlibrary{automata}
\usetikzlibrary{petri}
\usetikzlibrary{positioning}

% misc
\usepackage{enumerate}
\usepackage[section]{placeins}
\usepackage{subcaption}

% macros
\newcommand{\gdw}{\Leftrightarrow}

\title{Formale Grundlagen der Informatik II}
\subtitle{Aufgabenblatt 7}
\author{
    Jan-Hendrik Briese (6523408) \\
    Lennart Braun (6523742) \\
    Marc Strothmann (6537646) \\
    Maximilian Knapperzbusch (6535090)
}
\date{zum 8. Dezember 2014}

\begin{document}
\maketitle

\section*{Übungsaufgabe 8.3} 
\begin{enumerate}
    \item foo
        \begin{figure}[h]
            \centering
            \begin{tikzpicture}
                \tikzstyle{place}=[circle, thick, draw, minimum size=6mm]
                \tikzstyle{transition}=[rectangle, thick, draw, minimum size=4mm]

                \node [place] (b1) [label=below:$p_1|b_1$,] {};

                \node [red, transition] (ta1) [right= of b1] {$a$};
                \node [red, place] (b3) [label=below:$p_3|b_3$, above right= of ta1] {};
                \node [red, place] (b2) [label=above:$p_2|b_2$, above= of b3] {};
                \node [red, place] (b4) [label=above:$p_3|b_4$, below right= of ta1] {};
                \node [red, place] (b5) [label=below:$p_4|b_5$, below= of b4] {};

                \node [green, transition] (tb1) [right= of b3] {$b$};
                \node [blue, transition] (tc1) [right= of b4] {$c$};

                \node [green, place] (b6) [label=below:$p_5|b_6$, right= of tb1] {};
                \node [blue, place] (b7) [label=below:$p_5|b_7$, right= of tc1] {};

                \node [orange, transition] (td1) [right= of b6] {$d_1$};
                \node [purple, transition] (td2) [right= of b7] {$d_2$};

                \node [orange, place] (b8) [label=below:$p_6|b_8$, right= of td1] {};
                \node [purple, place] (b9) [label=below:$p_6|b_9$, right= of td2] {};

                \draw [red, pre]  (ta1) to (b1);
                \draw [red, post] (ta1) to (b2);
                \draw [red, post] (ta1) to (b3);
                \draw [red, post] (ta1) to (b4);
                \draw [red, post] (ta1) to (b5);

                \draw [green, pre]  (tb1) to (b2);
                \draw [green, pre]  (tb1) to (b3);
                \draw [green, post] (tb1) to (b6);

                \draw [blue, pre]  (tc1) to (b4);
                \draw [blue, pre]  (tc1) to (b5);
                \draw [blue, post] (tc1) to (b7);

                \draw [orange, pre]  (td1) to (b6);
                \draw [orange, post] (td1) to (b8);

                \draw [purple, pre]  (td2) to (b7);
                \draw [purple, post] (td2) to (b9);
            \end{tikzpicture}
            \caption{Prozesse von $N_{8.3}$}
            \label{fig:prozesse}
        \end{figure}

    \item

    \item

\end{enumerate}

\section*{Übungsaufgabe 8.4} 
\begin{figure}[h]
    \centering
    \begin{tikzpicture}[
        auto,
        scale=2,
        ->,
        >=stealth',
    ]

        % line a
        \node (fa) at (5,0) {$(2,0,0)$};
        \node (ga) at (6,0) {$(1,0,3)$};
        \node (ha) at (7,0) {$(0,0,6)$};

        % line b
        \node (gb) at (6,-1) {$(1,1,2)$};
        \node (hb) at (7,-1) {$(0,1,5)$};

        % line c
        \node (ec) at (4,-2) {$(1,0,2)$};
        \node (fc) at (5,-2) {$(0,0,5)$};
        \node (gc) at (6,-2) {$(1,2,1)$};
        \node (hc) at (7,-2) {$(0,2,4)$};

        % line d
        \node (ed) at (4,-3) {$(1,1,1)$};
        \node (fd) at (5,-3) {$(0,1,4)$};
        \node (gd) at (6,-3) {$(1,3,0)$};
        \node (hd) at (7,-3) {$(0,3,3)$};

        % line e
        \node (ce) at (2,-4) {$(1,0,1)$};
        \node (de) at (3,-4) {$(0,0,4)$};
        \node (ee) at (4,-4) {$(1,2,0)$};
        \node (fe) at (5,-4) {$(0,2,3)$};
        \node (he) at (7,-4) {$(0,4,2)$};

        % line f
        \node (cf) at (2,-5) {$(1,1,0)$};
        \node (df) at (3,-5) {$(0,1,3)$};
        \node (ff) at (5,-5) {$(0,3,2)$};

        % line g
        \node (ag) at (0,-6) {$(1,0,0)$};
        \node (bg) at (1,-6) {$(0,0,3)$};
        \node (dg) at (3,-6) {$(0,2,2)$};

        % line h
        \node (bh) at (1,-7) {$(0,1,2)$};

        % start
        \node (start) at (4,0) {};
        \draw (start) to (fa);

        % a edges
        \draw (fa) to node {$a$} (ga);
        \draw (ga) to node {$a$} (ha);
        \draw (gb) to node {$a$} (hb);
        \draw (ec) to node {$a$} (fc);
        \draw (gc) to node {$a$} (hc);
        \draw (ed) to node {$a$} (fd);
        \draw (gd) to node {$a$} (hd);
        \draw (ee) to node {$a$} (fe);
        \draw (ce) to node {$a$} (de);
        \draw (cf) to node {$a$} (df);
        \draw (ag) to node {$a$} (bg);

        % b edges
        \draw (gb) to node {$b$} (ec);
        \draw (gc) to node {$b$} (ed);
        \draw (ed) to node {$b$} (ce);
        \draw (gd) to node {$b$} (ee);
        \draw (ee) to node {$b$} (cf);
        \draw (cf) to node {$b$} (ag);

        % c edges
        \draw (gb) to node {$c$} (fa);
        \draw (hb) to node {$c$} (ga);
        \draw (hc) to node {$c$} (gb);
        \draw (fd) to node {$c$} (ec);
        \draw (hd) to node {$c$} (gc);
        \draw (fe) to node {$c$} (ed);
        \draw (he) to node {$c$} (gd);
        \draw (df) to node {$c$} (ce);
        \draw (ff) to node {$c$} (ee);
        \draw (dg) to node {$c$} (cf);
        \draw (bh) to node {$c$} (ag);

        % d edges
        \draw (ga) to node {$d$} (gb);
        \draw (ha) to node {$d$} (hb);
        \draw (hb) to node {$d$} (hc);
        \draw (fc) to node {$d$} (fd);
        \draw (hc) to node {$d$} (hd);
        \draw (fd) to node {$d$} (fe);
        \draw (hd) to node {$d$} (he);
        \draw (de) to node {$d$} (df);
        \draw (fe) to node {$d$} (ff);
        \draw (df) to node {$d$} (dg);
        \draw (bg) to node {$d$} (bh);

    \end{tikzpicture}
    \caption{Erreichbarkeitsgraph von $N_{8.4}$}
    \label{fig:prozesse}
\end{figure}

\section*{Übungsaufgabe 8.5} 
\begin{enumerate}
    \item

    \item

    \item

    \item

\end{enumerate}

\end{document}
