\documentclass[a4paper]{scrartcl}

% font/encoding packages
\usepackage[utf8]{inputenc}
\usepackage[T1]{fontenc}
\usepackage{lmodern}
\usepackage[ngerman]{babel}

% math
\usepackage{amsmath, amssymb, amsfonts, amsthm, mathtools}
\allowdisplaybreaks
\newtheorem*{behaupt}{Behauptung}

% tikz
\usepackage{tikz}
\usetikzlibrary{arrows}
\usetikzlibrary{automata}
\usetikzlibrary{petri}
\usetikzlibrary{positioning}

% misc
\usepackage{enumerate}
\usepackage[section]{placeins}
\usepackage{subcaption}

% macros
\newcommand{\gdw}{\Leftrightarrow}

\title{Formale Grundlagen der Informatik II}
\subtitle{Aufgabenblatt 7}
\author{
    Jan-Hendrik Briese (6523408) \\
    Lennart Braun (6523742) \\
    Marc Strothmann (6537646) \\
    Maximilian Knapperzbusch (6535090)
}
\date{zum 8. Dezember 2014}

\begin{document}
\maketitle

\section*{Übungsaufgabe 8.3} 
\begin{enumerate}
    \item Abbildung \ref{fig:prozesse} zeigt den größtmöglichen Prozess.
        Farbige Abschnitte können inklusive aller rechts an ihnen
        anschließenden Abschnitte (transitiv) weggelassen werden.
        Der Rest ist wiederum ein möglicher Prozess.
        \begin{figure}[h]
            \centering
            \begin{tikzpicture}
                \tikzstyle{place}=[circle, thick, draw, minimum size=6mm]
                \tikzstyle{transition}=[rectangle, thick, draw, minimum size=4mm]

                \node [place] (b1) [label=below:$p_1|b_1$,] {};

                \node [red, transition] (ta1) [right= of b1] {$a$};
                \node [red, place] (b3) [label=below:$p_3|b_3$, above right= of ta1] {};
                \node [red, place] (b2) [label=above:$p_2|b_2$, above= of b3] {};
                \node [red, place] (b4) [label=above:$p_3|b_4$, below right= of ta1] {};
                \node [red, place] (b5) [label=below:$p_4|b_5$, below= of b4] {};

                \node [green, transition] (tb1) [right= of b3] {$b$};
                \node [blue, transition] (tc1) [right= of b4] {$c$};

                \node [green, place] (b6) [label=below:$p_5|b_6$, right= of tb1] {};
                \node [blue, place] (b7) [label=below:$p_5|b_7$, right= of tc1] {};

                \node [orange, transition] (td1) [right= of b6] {$d_1$};
                \node [purple, transition] (td2) [right= of b7] {$d_2$};

                \node [orange, place] (b8) [label=below:$p_6|b_8$, right= of td1] {};
                \node [purple, place] (b9) [label=below:$p_6|b_9$, right= of td2] {};

                \draw [red, pre]  (ta1) to (b1);
                \draw [red, post] (ta1) to (b2);
                \draw [red, post] (ta1) to (b3);
                \draw [red, post] (ta1) to (b4);
                \draw [red, post] (ta1) to (b5);

                \draw [green, pre]  (tb1) to (b2);
                \draw [green, pre]  (tb1) to (b3);
                \draw [green, post] (tb1) to (b6);

                \draw [blue, pre]  (tc1) to (b4);
                \draw [blue, pre]  (tc1) to (b5);
                \draw [blue, post] (tc1) to (b7);

                \draw [orange, pre]  (td1) to (b6);
                \draw [orange, post] (td1) to (b8);

                \draw [purple, pre]  (td2) to (b7);
                \draw [purple, post] (td2) to (b9);
            \end{tikzpicture}
            \caption{Prozesse von $N_{8.3}$}
            \label{fig:prozesse}
        \end{figure}

    \item $R_{8.3} = (B, E, F_R)$
        \begin{figure}[h]
            \centering
            \begin{tikzpicture}[
                    auto,
                    scale=1.5,
                ]
                \tikzstyle{edge}=[->,>=stealth']
                \tikzstyle{place}=[circle, thick, draw, minimum size=6mm]
                \tikzstyle{transition}=[rectangle, thick, draw, minimum size=4mm]

                \node [place] (b1) at (0,2) [label=below:$b_1$] {};
                \node [place] (b2) at (0,0) [label=below:$b_2$] {};
                \node [transition] (a) at (1,2) {$a$};
                \node [place] (b3) at (2,2) [label=below:$b_3$] {};
                \node [place] (b4) at (2,1) [label=below:$b_4$] {};
                \node [transition] (c) at (3,0) {$c$};
                \node [place] (b5) at (4,0) [label=below:$b_5$] {};

                \draw [edge] (b1) to (a);
                \draw [edge] (a)  to (b3);
                \draw [edge] (a)  to (b4);

                \draw [edge] (b2) to (c);
                \draw [edge] (b4) to (c);
                \draw [edge] (c)  to (b5);
            \end{tikzpicture}
            \caption{$\lessdot \subseteq (B \cup E)^2$}
            \label{fig:rel1}
        \end{figure}

        \begin{figure}[h]
            \centering
            \begin{tikzpicture}[
                    auto,
                    scale=1.5,
                ]
                \tikzstyle{edge}=[->,>=stealth']
                \tikzstyle{place}=[circle, thick, draw, minimum size=6mm]
                \tikzstyle{transition}=[rectangle, thick, draw, minimum size=4mm]

                \node [place] (b1) at (0,2) [label=below:$b_1$] {};
                \node [place] (b2) at (0,0) [label=below:$b_2$] {};
                \node [transition] (a) at (1,2) {$a$};
                \node [place] (b3) at (2,2) [label=below:$b_3$] {};
                \node [place] (b4) at (2,1) [label=below:$b_4$] {};
                \node [transition] (c) at (3,0) {$c$};
                \node [place] (b5) at (4,0) [label=below:$b_5$] {};

                \draw [edge] (b1) to (a);
                \draw [edge] (b1) to [bend left] (b3);
                \draw [edge] (b1) to (b4);
                \draw [edge] (b1) to [bend right] (c);
                \draw [edge] (b1) to [bend left=60] (b5);

                \draw [edge] (a)  to (b3);
                \draw [edge] (a)  to (b4);
                \draw [edge] (a)  to [bend right] (c);
                \draw [edge] (a)  to (b5);

                \draw [edge] (b2) to (c);
                \draw [edge] (b2) to [bend right] (b5);

                \draw [edge] (b4) to (c);
                \draw [edge] (b4) to (b5);

                \draw [edge] (c)  to (b5);
            \end{tikzpicture}
            \caption{$< \subseteq (B \cup E)^2$}
            \label{fig:rel2}
        \end{figure}

        \begin{figure}[h]
            \centering
            \begin{tikzpicture}[
                    auto,
                    scale=1.5,
                ]
                \tikzstyle{edge}=[]
                \tikzstyle{place}=[circle, thick, draw, minimum size=6mm]
                \tikzstyle{transition}=[rectangle, thick, draw, minimum size=4mm]

                \node [place] (b1) at (0,2) [label=below:$b_1$] {};
                \node [place] (b2) at (0,0) [label=below:$b_2$] {};
                \node [transition] (a) at (1,2) {$a$};
                \node [place] (b3) at (2,2) [label=below:$b_3$] {};
                \node [place] (b4) at (2,1) [label=below:$b_4$] {};
                \node [transition] (c) at (3,0) {$c$};
                \node [place] (b5) at (4,0) [label=below:$b_5$] {};

                \draw [edge] (b1) to [loop left] (b1);
                \draw [edge] (b2) to [loop left] (b2);
                \draw [edge] (b3) to [loop right] (b3);
                \draw [edge] (b4) to [loop left] (b4);
                \draw [edge] (b5) to [loop right] (b5);
                \draw [edge] (a) to [loop above] (a);
                \draw [edge] (c) to [loop below] (a);

                \draw [edge] (b1) to (a);
                \draw [edge] (b1) to [bend left=60] (b3);
                \draw [edge] (b1) to (b4);
                \draw [edge] (b1) to [bend right] (c);
                \draw [edge] (b1) to [bend left=90] (b5);

                \draw [edge] (a)  to (b3);
                \draw [edge] (a)  to (b4);
                \draw [edge] (a)  to [bend right=40] (c);
                \draw [edge] (a)  to (b5);

                \draw [edge] (b2) to (c);
                \draw [edge] (b2) to [bend right=40] (b5);

                \draw [edge] (b4) to (c);
                \draw [edge] (b4) to (b5);

                \draw [edge] (c)  to (b5);
            \end{tikzpicture}
            \caption{$\textbf{li} \subseteq (B \cup E)^2$}
            \label{fig:rel3}
        \end{figure}

        \begin{figure}[h]
            \centering
            \begin{tikzpicture}[
                    auto,
                    scale=1.5,
                ]
                \tikzstyle{edge}=[]
                \tikzstyle{place}=[circle, thick, draw, minimum size=6mm]
                \tikzstyle{transition}=[rectangle, thick, draw, minimum size=4mm]

                \node [place] (b1) at (0,2) [label=above:$b_1$] {};
                \node [place] (b2) at (0,0) [label=below:$b_2$] {};
                \node [transition] (a) at (1,2) {$a$};
                \node [place] (b3) at (2,2) [label=above:$b_3$] {};
                \node [place] (b4) at (2,1) [label=below:$b_4$] {};
                \node [transition] (c) at (3,0) {$c$};
                \node [place] (b5) at (4,0) [label=below:$b_5$] {};

                \draw [edge] (b1) to [loop left] (b1);
                \draw [edge] (b2) to [loop left] (b2);
                \draw [edge] (b3) to [loop left] (b3);
                \draw [edge] (b4) to [loop left] (b4);
                \draw [edge] (b5) to [loop left] (b5);
                \draw [edge] (a) to [loop left] (a);
                \draw [edge] (c) to [loop left] (a);

                \draw [edge] (b1) to (b2);

                \draw [edge] (b2) to (a);
                \draw [edge] (b2) to (b3);
                \draw [edge] (b2) to (b4);

                \draw [edge] (b3) to (b4);
                \draw [edge] (b3) to (b5);
                \draw [edge] (b3) to (c);
            \end{tikzpicture}
            \caption{$\textbf{co} \subseteq (B \cup E)^2$}
            \label{fig:rel4}
        \end{figure}
        

    \item
        Der größtmögliche $P$-Schnitt ist $\{ b_2, b_3, b_4 \}$. \\
        Die größtmöglichen $T$-Schnitte sind $\{ a \}$ und $\{ b \}$.

\end{enumerate}

\section*{Übungsaufgabe 8.4} 
\begin{figure}[h]
    \centering
    \begin{tikzpicture}[
        auto,
        scale=2,
    ]
        \tikzstyle{edge}=[->,>=stealth']
        \tikzstyle{szk}=[red, rounded corners=8pt]

        % row a
        \node (fa) at (5,0) {$(2,0,0)$};
        \node (ga) at (6,0) {$(1,0,3)$};
        \node (ha) at (7,0) {$(0,0,6)$};

        % row b
        \node (gb) at (6,-1) {$(1,1,2)$};
        \node (hb) at (7,-1) {$(0,1,5)$};

        % row c
        \node (ec) at (4,-2) {$(1,0,2)$};
        \node (fc) at (5,-2) {$(0,0,5)$};
        \node (gc) at (6,-2) {$(1,2,1)$};
        \node (hc) at (7,-2) {$(0,2,4)$};

        % row d
        \node (ed) at (4,-3) {$(1,1,1)$};
        \node (fd) at (5,-3) {$(0,1,4)$};
        \node (gd) at (6,-3) {$(1,3,0)$};
        \node (hd) at (7,-3) {$(0,3,3)$};

        % row e
        \node (ce) at (2,-4) {$(1,0,1)$};
        \node (de) at (3,-4) {$(0,0,4)$};
        \node (ee) at (4,-4) {$(1,2,0)$};
        \node (fe) at (5,-4) {$(0,2,3)$};
        \node (he) at (7,-4) {$(0,4,2)$};

        % row f
        \node (cf) at (2,-5) {$(1,1,0)$};
        \node (df) at (3,-5) {$(0,1,3)$};
        \node (ff) at (5,-5) {$(0,3,2)$};

        % row g
        \node (ag) at (0,-6) {$(1,0,0)$};
        \node (bg) at (1,-6) {$(0,0,3)$};
        \node (dg) at (3,-6) {$(0,2,2)$};

        % row h
        \node (bh) at (1,-7) {$(0,1,2)$};

        % start
        \node (start) at (4,0) {};
        \draw[edge] (start) to (fa);

        % a edges
        \draw[edge] (fa) to node {$a$} (ga);
        \draw[edge] (ga) to node {$a$} (ha);
        \draw[edge] (gb) to node {$a$} (hb);
        \draw[edge] (ec) to node {$a$} (fc);
        \draw[edge] (gc) to node {$a$} (hc);
        \draw[edge] (ed) to node {$a$} (fd);
        \draw[edge] (gd) to node {$a$} (hd);
        \draw[edge] (ee) to node {$a$} (fe);
        \draw[edge] (ce) to node {$a$} (de);
        \draw[edge] (cf) to node {$a$} (df);
        \draw[edge] (ag) to node {$a$} (bg);

        % b edges
        \draw[edge] (gb) to node {$b$} (ec);
        \draw[edge] (gc) to [near start] node {$b$} (ed);
        \draw[edge] (ed) to node {$b$} (ce);
        \draw[edge] (gd) to [near start] node {$b$} (ee);
        \draw[edge] (ee) to [near start] node {$b$} (cf);
        \draw[edge] (cf) to node {$b$} (ag);

        % c edges
        \draw[edge] (gb) to node {$c$} (fa);
        \draw[edge] (hb) to node {$c$} (ga);
        \draw[edge] (hc) to node {$c$} (gb);
        \draw[edge] (fd) to [near end] node {$c$} (ec);
        \draw[edge] (hd) to node {$c$} (gc);
        \draw[edge] (fe) to [near end] node {$c$} (ed);
        \draw[edge] (he) to node {$c$} (gd);
        \draw[edge] (df) to [near end] node {$c$} (ce);
        \draw[edge] (ff) to node {$c$} (ee);
        \draw[edge] (dg) to node {$c$} (cf);
        \draw[edge] (bh) to node {$c$} (ag);

        % d edges
        \draw[edge] (ga) to node {$d$} (gb);
        \draw[edge] (ha) to node {$d$} (hb);
        \draw[edge] (hb) to node {$d$} (hc);
        \draw[edge] (fc) to [at start] node {$d$} (fd);
        \draw[edge] (hc) to node {$d$} (hd);
        \draw[edge] (fd) to [at start] node {$d$} (fe);
        \draw[edge] (hd) to node {$d$} (he);
        \draw[edge] (de) to [at start] node {$d$} (df);
        \draw[edge] (fe) to node {$d$} (ff);
        \draw[edge] (df) to node {$d$} (dg);
        \draw[edge] (bg) to node {$d$} (bh);

        % SZKs
        \draw[szk] (6.5,0.25) to (7.4,0.25)
                              to (7.4,-4.25)
                              to (6.6,-4.25)
                              to (5.6,-3.25)
                              to (5.6,-1.25)
                              to (4.5,-0.25)
                              to (4.5,0.25)
                              to (6.5,0.25)
                              ;
        \draw[szk] (4.5,-1.75) to (5.35,-1.75)
                               to (5.35,-5.25)
                               to (4.6,-5.25)
                               to (3.6,-4.25)
                               to (3.6,-1.75)
                               to (4.5,-1.75)
                               ;
        \draw[szk] (2.5,-3.75) to (3.35,-3.75)
                               to (3.35,-6.25)
                               to (2.6,-6.25)
                               to (1.6,-5.25)
                               to (1.6,-3.75)
                               to (2.5,-3.75)
                               ;
        \draw[szk] (0.5,-5.75) to (1.4,-5.75)
                               to (1.4,-7.2)
                               to (0.6,-7.2)
                               to (-0.4,-6.2)
                               to (-0.4,-5.75)
                               to (0.5,-5.75)
                               ;
       \node[red] (c1) at (7,-4.5) {$C_1$};
       \node[red] (c2) at (5,-5.5) {$C_2$};
       \node[red] (c3) at (3,-6.5) {$C_3$};
       \node[red] (c4) at (1.75,-6.75) {$C_4$};

    \end{tikzpicture}
    \caption{Erreichbarkeitsgraph $RG$ von $N_{8.4}$ mit den strengen
        Zusammenhangskomponenten $C_i$}
    \label{fig:reachability}
\end{figure}

\begin{figure}[h]
    \centering
    \begin{tikzpicture}[
        auto,
        scale=2,
        ->,
        >=stealth',
    ]

    \tikzstyle{szk}=[draw, circle]

    \node [szk] (c1) at (3,3) {$C_1$};
    \node [szk] (c2) at (2,2) {$C_2$};
    \node [szk] (c3) at (1,1) {$C_3$};
    \node [szk] (c4) at (0,0) {$C_4$};

    \draw (c1) to node {$b$} (c2);
    \draw (c2) to node {$b$} (c3);
    \draw (c3) to node {$b$} (c4);

    \end{tikzpicture}
    \caption{Reduzierter Erreichbarkeitsgraph $RG^C$ von $N_{8.4}$}
    \label{fig:rgc}
\end{figure}

\paragraph{Vorgehen}
Der Erreichbarkeitsgraph $RG$ (Abb. \ref{fig:reachability}) wird als Eingabe
des Algorithmus' 7.3 verwendet.
Im 1. Schritt werden die strengen Zusammenhangskomponenten $C_i$ von $RG$
berechnet (Rot in \ref{fig:reachability}).
Im 2. Schritt wird der reduzierte Graph $RG^C$ (Abb. \ref{fig:rgc}) berechnet.
Die einzige terminale SZK von $RG^C$ ist $C_4$: $F = \{ C_4 \}$.
Die Lebendigkeitsinvarianz von $t \in \{ a,b,c,d \}$ für eine Markierung $m$
lautet
\begin{equation}
    \prod\nolimits_t (m) = (m \geq W(\bullet,t)) \text{ .}
\end{equation}
Für jede Transition $t$ wird geprüft, ob eine Markierung $m \in C_4$ existiert,
welche $\prod\nolimits_t(m)$ erfüllt.
Für $a$, $c$ und $d$ ist jeweils eine vorhanden.
Es existiert jedoch keine Markierung, in der $b$ aktiviert ist.
Daher kann gesagt werden, dass alle Transitionen bis auf $b$ lebendig sind.


\section*{Übungsaufgabe 8.5} 
\begin{enumerate}
    \item
        Zwei oder mehr Transitionen, die zusammen in einem $T$-Schnitt
        auftauchen, können nebenläufig ausgeführt werden.

    \item
        Jeder $P$-Schnitt entspricht einer möglichen Markierung in einem
        Prozess.
        Alle Stellen, die zusammen in einem $P$-Schnitt enthalten sind,
        sind gleichzeitig in dieser Markierung belegt.

    \item
        Ein Prozess eines Petrinetzes ist ein Kausalnetz.
        Diese muss vorgänger-endlich sein.
        Es ist jedoch möglich, Kausalnetze zu konstruieren, welche nicht
        vorgänger-endlich sind und damit auch keine Prozesse darstellen können.
        % Prozess -> endlich verzweigtes Kausalnetz ? (Skript S.113)

    \item
        Sei $R = (B,E,F_R)$, $N_{8.4i} = (P_i,T_i,F_i,W_i,m_{0i})$ für
        $i \in \{ a,b,c \}$.

        Das Kausalnetz $R$ passt zu den Petrinetzen $N_{8.4a}$ und $N_{8.4b}$.
        Die Paarungen werden durch die Abbildungspaare $\phi_a$ und $\phi_b$
        spezifiziert.

        $\phi_{Pa} \colon B \to P_a$
        \begin{equation}
            \begin{split}
                b_1 &\mapsto p_1 \\
                b_2 &\mapsto p_1 \\
                b_3 &\mapsto p_1 \\
                b_4 &\mapsto p_1 \\
                b_5 &\mapsto p_1
            \end{split}
        \end{equation}
        $\phi_{Ta} \colon B \to T_a$
        \begin{equation}
            \begin{split}
                x &\mapsto a \\
                y &\mapsto b
            \end{split}
        \end{equation}

        $\phi_{Pb} \colon B \to P_b$
        \begin{equation}
            \begin{split}
                b_1 &\mapsto q_1 \\
                b_2 &\mapsto q_2 \\
                b_3 &\mapsto q_3 \\
                b_4 &\mapsto q_3 \\
                b_5 &\mapsto q_4
            \end{split}
        \end{equation}
        $\phi_{Tb} \colon B \to T_b$
        \begin{equation}
            \begin{split}
                x &\mapsto s \\
                y &\mapsto t
            \end{split}
        \end{equation}
        

\end{enumerate}

\end{document}
