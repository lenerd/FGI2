\documentclass[a4paper]{scrartcl}

% font/encoding packages
\usepackage[utf8]{inputenc}
\usepackage[T1]{fontenc}
\usepackage{lmodern}
\usepackage[ngerman]{babel}

% math
\usepackage{amsmath, amssymb, amsfonts, amsthm, mathtools}
\allowdisplaybreaks
\newtheorem*{behaupt}{Behauptung}
\usepackage{stmaryrd}

% tikz
\usepackage{tikz}
\usetikzlibrary{arrows}
\usetikzlibrary{automata}
\usetikzlibrary{shapes}
\usetikzlibrary{petri}
\usetikzlibrary{positioning}

% misc
\usepackage{enumerate}
\usepackage[section]{placeins}
\usepackage{tabu}
\usepackage{rotating}
\usepackage{subcaption}

% macros
\newcommand{\gdw}{\Leftrightarrow}
\newcommand{\too}{\longrightarrow}

\title{Formale Grundlagen der Informatik II}
\subtitle{Aufgabenblatt 13}
\author{
    Jan-Hendrik Briese (6523408) \\
    Lennart Braun (6523742) \\
    Marc Strothmann (6537646) \\
    Maximilian Knapperzbusch (6535090)
}
\date{zum 26. Januar 2015}

\begin{document}
\maketitle

\section*{Bonusaufgabe 13.3} 

\begin{enumerate}
    \item Abbildung \ref{fig:reach} auf Seite \pageref{fig:reach}
        \begin{sidewaysfigure}[h]
            \centering
            \begin{tikzpicture}[
                    auto,
                ]
                \tikzset{
                    edge/.style={
                        ->,
                        >=stealth',
                    }
                }
                \node (03)               {$p_1 + p_2 + p_3 + p_4 + p_5$};
                \node (02) [below=of 03] {$p_1 + p_2 + p_5 + p_7$};
                \node (01) [below=of 02] {$p_1 + p_2 + p_3 + p_4 + p_8$};
                \node (00) [below=of 01] {$p_1 + p_2 + p_7 + p_8$};

                \node (13) [right=of 03] {$p_2 + p_3 + p_4 + p_5 + p_6$};
                \node (12) [below=of 13] {$p_2 + p_5 + p_6 + p_7$};
                \node (11) [below=of 12] {$p_2 + p_3 + p_4 + p_6 + p_8$};
                \node (10) [below=of 11] {$p_2 + p_6 + p_7 + p_8$};

                \node (23) [right=of 13] {$p_3 + p_4 + p_5 + p_9$};
                \node (22) [below=of 23] {$p_5 + p_7 + p_9$};
                \node (21) [below=of 22] {$p_3 + p_4 + p_8 + p_9$};
                \node (20) [below=of 21] {$p_7 + p_8 + p_9$};

                \node (32) [right=of 22] {$p_5 + p_{10}$};
                \node (30) [right=of 20] {$p_8 + p_{10}$};

                \node (42) [right=of 32] {$p_5 + p_{11}$};
                \node (40) [right=of 30] {$p_8 + p_{11}$};

                \node (50) [right=of 40] {$\emptyset$};

                \draw [edge] (03) to node {$a$} (02);
                \draw [edge] (01) to node {$a$} (00);
                \draw [edge] (13) to node {$a$} (12);
                \draw [edge] (11) to node {$a$} (10);
                \draw [edge] (23) to node {$a$} (22);
                \draw [edge] (21) to node {$a$} (20);

                \draw [edge, bend right=60] (03.west) to node [near start] {$h$} (01.west);
                \draw [edge, bend right=60] (02.west) to node [near start] {$h$} (00.west);
                \draw [edge, bend right=30] (13.west) to node [near start] {$h$} (11.west);
                \draw [edge, bend right=40] (12.west) to node [near start] {$h$} (10.west);
                \draw [edge, bend right=30] (23.west) to node [near start] {$h$} (21.west);
                \draw [edge, bend right=40] (22.west) to node [near start] {$h$} (20.west);
                \draw [edge, bend right] (32.west) to node [near start] {$h$} (30.west);
                \draw [edge, bend right] (42.west) to node [near start] {$h$} (40.west);

                \draw [edge] (03) to node [near start] {$v, w$} (13);
                \draw [edge] (02) to node [near start] {$v, w$} (12);
                \draw [edge] (01) to node [near start] {$v, w$} (11);
                \draw [edge] (00) to node [near start] {$v, w$} (10);

                \draw [edge] (13) to node [near start] {$m$} (23);
                \draw [edge] (12) to node [near start] {$m$} (22);
                \draw [edge] (11) to node [near start] {$m$} (21);
                \draw [edge] (10) to node [near start] {$m$} (20);

                \draw [edge] (22) to node {$k$} (32);
                \draw [edge] (20) to node {$k$} (30);

                \draw [edge] (32) to node {$f$} (42);
                \draw [edge] (30) to node {$f$} (40);

                \draw [edge] (40) to node {$b$} (50);

            \end{tikzpicture}
            \caption{Erreichbarkeitsgraph}
            \label{fig:reach}
        \end{sidewaysfigure}

    \item
        \begin{align}
            t &= \partial_H \Big( \left( (v + w) \cdot m \cdot t \cdot f \cdot l \right)
            || \left( a \cdot r \right) || \left( h \cdot s \right) \Big) \\
            H &= \left\{ l, r, s, t \right\} \\
            \gamma (x, y) &= \begin{cases}
                b & \text{if } (x, y) = (l, s) \lor (x, y) = (s, l) \\
                k & \text{if } (x, y) = (r, t) \lor (x, y) = (t, r) \\
                \delta & \text{else}
            \end{cases}
        \end{align}
        

    \item
        \begin{sidewaysfigure}[h]
            \centering
            \scalebox{0.8}{
                \begin{tikzpicture}[
                        auto,
                    ]
                    \tikzset{
                        edge/.style={
                            ->,
                            >=stealth',
                        }
                    }
                    \node (03)               {
                        $\partial_H \Big( ((v + w) m t f l)
                        || (a r) || (h s) \Big)$
                    };
                    \node (02) [below=of 03] {
                        $\partial_H \Big( ((v + w) m t f l)
                        || r ||  (h s)  \Big)$
                    };
                    \node (01) [below=of 02] {
                        $\partial_H \Big( ((v + w) m t f l)
                        || (a r) || s \Big)$
                    };
                    \node (00) [below=of 01] {
                        $\partial_H \Big( ((v + w) m t f l)
                        || r || s \Big)$
                    };

                    \node (13) [right=of 03] {
                        $\partial_H \Big( (m t f l)
                        || (a r) || (h s) \Big)$
                    };
                    \node (12) [below=of 13] {
                        $\partial_H \Big( (m t f l)
                        || r || (h s)  \Big)$
                    };
                    \node (11) [below=of 12] {
                        $\partial_H \Big( (m t f l)
                        || (a r) || s \Big)$
                    };
                    \node (10) [below=of 11] {
                        $\partial_H \Big( (m t f l)
                        || r || s \Big)$
                    };

                    \node (23) [right=of 13] {
                        $\partial_H \Big( (t f l)
                        || (a r) || (h s) \Big)$
                    };
                    \node (22) [below=of 23] {
                        $\partial_H \Big( (t f l)
                        || r || (h s) \Big)$
                    };
                    \node (21) [below=of 22] {
                        $\partial_H \Big( (t f l)
                        || (a r) || s \Big)$
                    };
                    \node (20) [below=of 21] {
                        $\partial_H \Big( (t f l)
                        || r || s \Big)$
                    };

                    \node (32) [right=of 22] {
                        $\partial_H \Big( (f l)
                        || (h s) \Big)$
                    };
                    \node (30) [right=of 20] {
                        $\partial_H \Big( ( f l )
                        || s \Big)$
                    };

                    \node (42) [right=of 32] {
                        $\partial_H \Big( l
                        || (h s) \Big)$
                    };
                    \node (40) [right=of 30] {
                        $\partial_H \Big( l
                        || s \Big)$
                    };

                    \node (50) [right=of 40] {$\checkmark$};

                    \draw [edge] (03) to node {$a$} (02);
                    \draw [edge] (01) to node {$a$} (00);
                    \draw [edge] (13) to node {$a$} (12);
                    \draw [edge] (11) to node {$a$} (10);
                    \draw [edge] (23) to node {$a$} (22);
                    \draw [edge] (21) to node {$a$} (20);

                    \draw [edge, bend right=60] (03.west) to node [near start] {$h$} (01.west);
                    \draw [edge, bend right=60] (02.west) to node [near start] {$h$} (00.west);
                    \draw [edge, bend right=30] (13.west) to node [near start] {$h$} (11.west);
                    \draw [edge, bend right=40] (12.west) to node [near start] {$h$} (10.west);
                    \draw [edge, bend right=30] (23.west) to node [near start] {$h$} (21.west);
                    \draw [edge, bend right=40] (22.west) to node [near start] {$h$} (20.west);
                    \draw [edge, bend right] (32.west) to node [near start] {$h$} (30.west);
                    \draw [edge, bend right] (42.west) to node [near start] {$h$} (40.west);

                    \draw [edge] (03) to node [near start] {$v, w$} (13);
                    \draw [edge] (02) to node [near start] {$v, w$} (12);
                    \draw [edge] (01) to node [near start] {$v, w$} (11);
                    \draw [edge] (00) to node [near start] {$v, w$} (10);

                    \draw [edge] (13) to node [near start] {$m$} (23);
                    \draw [edge] (12) to node [near start] {$m$} (22);
                    \draw [edge] (11) to node [near start] {$m$} (21);
                    \draw [edge] (10) to node [near start] {$m$} (20);

                    \draw [edge] (22) to node {$k$} (32);
                    \draw [edge] (20) to node {$k$} (30);

                    \draw [edge] (32) to node {$f$} (42);
                    \draw [edge] (30) to node {$f$} (40);

                    \draw [edge] (40) to node {$b$} (50);

                \end{tikzpicture}
            }
            \caption{Prozessgraph}
            \label{fig:13-3-process}
        \end{sidewaysfigure}

         \begin{equation}
            \begin{split}
                \checkmark \ &\mathcal{B}\  \emptyset \\
                %
                \partial_H \Big( l || s \Big) \ &\mathcal{B}\  p_8 + p_{11} \\
                \partial_H \Big( l || (hs) \Big) \ &\mathcal{B}\  p_5 + p_{11} \\
                %
                \partial_H \Big( (fl) || s \Big) \ &\mathcal{B}\  p_8 + p_{10} \\
                \partial_H \Big( (fl) || (hs) \Big) \ &\mathcal{B}\  p_5 + p_{10} \\
                %
                \partial_H \Big( (tfl) || r || s \Big) \ &\mathcal{B}\  p_7 + p_8 + p_9 \\
                \partial_H \Big( (tfl) || (ar) || s \Big) \ &\mathcal{B}\  p_3 + p_4 + p_8 + p_9 \\
                \partial_H \Big( (tfl) || r || (hs) \Big) \ &\mathcal{B}\  p_5 + p_7 + p_9 \\
                \partial_H \Big( (tfl) || (ar) || (hs) \Big) \ &\mathcal{B}\  p_3 + p_4 + p_5 + p_9 \\
                %
                \partial_H \Big( (mtfl) || r || s \Big) \ &\mathcal{B}\  p_2 + p_6 + p_7 + p_8 \\
                \partial_H \Big( (mtfl) || (ar) || s \Big) \ &\mathcal{B}\  p_2 + p_3 + p_4 + p_6 + p_8 \\
                \partial_H \Big( (mtfl) || r || (hs) \Big) \ &\mathcal{B}\  p_2 + p_5 + p_6 + p_7 \\
                \partial_H \Big( (mtfl) || (ar) || (hs) \Big) \ &\mathcal{B}\  p_2 + p_3 + p_4 + p_5 + p_6 \\
                %
                \partial_H \Big( ((v + w)mtfl) || r || s \Big) \ &\mathcal{B}\  p_1 + p_2 + p_7 + p_8 \\
                \partial_H \Big( ((v + w)mtfl) || (ar) || s \Big) \ &\mathcal{B}\  p_1 + p_2 + p_3 + p_4 + p_8 \\
                \partial_H \Big( ((v + w)mtfl) || r || (hs) \Big) \ &\mathcal{B}\  p_1 + p_2 + p_5 + p_7 \\
                \partial_H \Big( ((v + w)mtfl) || (ar) || (hs) \Big) \ &\mathcal{B}\  p_1 + p_2 + p_3 + p_4 + p_5
            \end{split}
        \end{equation}
        Der Prozessgraph \ref{fig:13-3-process} ist bisimilar zu dem
        Erreichbarkeitsgraphen \ref{fig:reach}.

\end{enumerate}

\section*{Übungsaufgabe 13.4} 

\begin{enumerate}
    \item \hfill \\
        \begin{figure}[h]
            \centering
            \begin{subfigure}{0.45\textwidth}
                \centering
                \begin{tikzpicture}[
                        auto,
                        scale=2,
                    ]
                    \tikzset{
                        edge/.style={
                            ->,
                            >=stealth',
                        }
                    }
                    \node (a) at (0,3) {$(a + b) \llfloor (b \cdot c)$};
                    \node (b) at (0,2) {$b \cdot c$};
                    \node (c) at (0,1) {$c$};
                    \node (d) at (0,0) {$\checkmark$};

                    \draw[edge, bend right] (a) to node {$a$} (b);
                    \draw[edge, bend left]  (a) to node {$b$} (b);
                    \draw[edge] (b) to node {$b$} (c);
                    \draw[edge] (c) to node {$c$} (d);
                \end{tikzpicture}
                \caption{Prozessgraph von $t_6$}
            \end{subfigure}
            \begin{subfigure}{0.45\textwidth}
                \centering
                \begin{tikzpicture}[
                        auto,
                        scale=2,
                    ]
                    \tikzset{
                        edge/.style={
                            ->,
                            >=stealth',
                        }
                    }
                    \node (a) at (1,3) {
                        $\partial_H \left( \left( a \llfloor (b + b) + (d || e)
                        \cdot b \right) \cdot c \right)$
                    };
                    \node (b) at (0,2) {$\partial_H \left( (b + b) \cdot c \right)$};
                    \node (c) at (2,2) {$\partial_H \left( b \cdot c \right)$};
                    \node (d) at (1,1) {$\partial_H \left( c \right)$};
                    \node (e) at (1,0) {$\checkmark$};

                    \draw[edge] (a) to node {$a$} (b);
                    \draw[edge] (a) to node {$f = \gamma(d,e)$} (c);
                    \draw[edge] (b) to node {$b$} (d);
                    \draw[edge] (c) to node {$b$} (d);
                    \draw[edge] (d) to node {$c$} (e);
                \end{tikzpicture}
                \caption{Prozessgraph von $t_7$}
            \end{subfigure}
            \label{fig:13-4-process}
        \end{figure}
        \begin{equation}
            \begin{split}
                \checkmark \ &\mathcal{B}\  \checkmark \\
                c \ &\mathcal{B}\  \partial_H \left( c \right) \\
                b \cdot c \ &\mathcal{B}\  \partial_H \left( b \cdot c \right) \\
                b \cdot c \ &\mathcal{B}\  \partial_H \left( (b + b) \cdot c \right) \\
            \end{split}
        \end{equation}
        Die Prozesse $t_6$ und $t_7$ sind nicht bisimilar.
        $t_6$ akzeptiert die Aktionsfolgen $abc$ und $bbc$ während $t_7$ $abc$
        und $fbc$ akzeptiert.

    \item
        \begin{align}
                a &\stackrel{a}{\too} \checkmark & &\frac{}{v \stackrel{v}{\too} \checkmark} \\
                (a + b) &\stackrel{a}{\too} \checkmark & &\frac{x \stackrel{v}{\too} \checkmark}{x + y \stackrel{v}{\too} \checkmark} \\
                (a + b) \llfloor (b \cdot c) &\stackrel{a}{\too} b \cdot c & &T_\llfloor^\checkmark \\
        \end{align}
        \begin{align}
                d &\stackrel{d}{\too} \checkmark ,\ e \stackrel{e}{\too} \checkmark & &\frac{}{v \stackrel{v}{\too} \checkmark} \\
                d||e &\stackrel{f}{\too} \checkmark & &T_{||\gamma}^{\checkmark\checkmark} \quad \gamma(d,e) = f \\
                (d||e) \cdot b &\stackrel{f}{\too} b & &\frac{x \stackrel{v}{\too} \checkmark}{x \cdot y \stackrel{v}{\too} y} \\
                a \llfloor (b + b) + (d||e) \cdot b &\stackrel{f}{\too} b & &\frac{y \stackrel{v}{\too} y'}{x + y \stackrel{v}{\too} y'} \\
                \left( a \llfloor (b + b) + (d||e) \cdot b \right) \cdot c &\stackrel{f}{\too} b \cdot c & &\frac{x \stackrel{v}{\too} x'}{x \cdot y \stackrel{v}{\too} x' \cdot y} \\
                \partial_H \Big( \left( a \llfloor (b + b) + (d||e) \cdot b \right) \cdot c \Big) &\stackrel{f}{\too} \partial_H \Big( b \cdot c \Big) & &T_\partial
        \end{align}

    \item
        \begin{equation}
            \begin{split}
                t_6 &= (a + b) \llfloor (b \cdot c) \\
                &\stackrel{LM4}{=} a \llfloor (b \cdot c) + b \llfloor (b \cdot c) \\
                &\stackrel{LM2}{=} a \cdot (b \cdot c) + b \cdot (b \cdot c)
            \end{split}
        \end{equation}
        \begin{equation}
            \begin{split}
                t_7 &= \partial_H \Big( (a \llfloor (b + b) + (d || e)
                    \cdot b) \cdot c \Big) \\
                &\stackrel{A3}{=} \partial_H \Big( (a \llfloor b + (d || e)
                    \cdot b) \cdot c \Big) \\
                &\stackrel{A4}{=} \partial_H \Big( (a \llfloor b) \cdot c + ((d || e)
                    \cdot b) \cdot c \Big) \\
                &\stackrel{D4}{=} \partial_H \Big( (a \llfloor b) \cdot c \Big)
                    + \partial_H \Big( ((d || e) \cdot b) \cdot c \Big) \\
                &\stackrel{LM2}{=} \partial_H \Big( (a \cdot b) \cdot c \Big)
                    + \partial_H \Big( ((d || e) \cdot b) \cdot c \Big) \\
                &\stackrel{A5}{=} \partial_H \Big( a \cdot (b \cdot c) \Big)
                    + \partial_H \Big( (d || e) \cdot (b \cdot c) \Big) \\
                &\stackrel{M1}{=} \partial_H \Big( a \cdot (b \cdot c) \Big)
                    + \partial_H \Big( ((d \llfloor e + e \llfloor d) + d | e) \cdot (b \cdot c) \Big) \\
                &\stackrel{LM2}{=} \partial_H \Big( a \cdot (b \cdot c) \Big)
                    + \partial_H \Big( ((d \cdot e + e \cdot d) + d | e) \cdot (b \cdot c) \Big) \\
                &\stackrel{CM5}{=} \partial_H \Big( a \cdot (b \cdot c) \Big)
                    + \partial_H \Big( ((d \cdot e + e \cdot d) + \gamma(d,e)) \cdot (b \cdot c) \Big) \\
                &\stackrel{D5}{=} \partial_H \Big( a \Big) \cdot \partial_H \Big( b \cdot c \Big)
                    + \partial_H \Big( ((d \cdot e + e \cdot d) + \gamma(d,e) \Big) \cdot \partial_H \Big( b \cdot c \Big) \\
                &\stackrel{D5}{=} \partial_H \Big( a \Big) \cdot \left( \partial_H \Big( b \Big) \cdot \partial_H \Big( c \Big) \right)
                    + \partial_H \Big( ((d \cdot e + e \cdot d) + \gamma(d,e) \Big) \cdot \left( \partial_H \Big( b \Big) \cdot \partial_H \Big( c \Big) \right) \\
                &\stackrel{D1}{=} a \cdot (b \cdot c)
                    + \partial_H \Big( ((d \cdot e + e \cdot d) + \gamma(d,e) \Big) \cdot (b \cdot c) \\
                &\stackrel{D4}{=} a \cdot (b \cdot c)
                    + \left( \partial_H \Big( d \cdot e + e \cdot d \Big) + \partial_H \Big( \gamma(d,e) \Big) \right) \cdot (b \cdot c) \\
                &\stackrel{D4}{=} a \cdot (b \cdot c)
                    + \left( \left( \partial_H \Big( d \cdot e \Big) + \partial_H \Big( e \cdot d \Big) \right) + \partial_H \Big( \gamma(d,e) \Big) \right) \cdot (b \cdot c) \\
                &\stackrel{D5}{=} a \cdot (b \cdot c)
                    + \left( \left( \left( \partial_H \Big( d \Big) \cdot \partial_H \Big( e \Big) \right) + \left( \partial_H \Big( e  \Big) \cdot \partial_H \Big( d \Big) \right) \right) + \partial_H \Big( \gamma(d,e) \Big) \right) \cdot (b \cdot c) \\
                &\stackrel{D2}{=} a \cdot (b \cdot c)
                    + \left( \left( \left( \delta \cdot \delta \right) + \left( \delta \cdot \delta \right) \right) + \partial_H \Big( \gamma(d,e) \Big) \right) \cdot (b \cdot c) \\
                &\stackrel{\gamma}{=} a \cdot (b \cdot c)
                    + \left( \left( \left( \delta \cdot \delta \right) + \left( \delta \cdot \delta \right) \right) + \partial_H \Big( f \Big) \right) \cdot (b \cdot c) \\
                &\stackrel{D1}{=} a \cdot (b \cdot c)
                    + \left( \left( \left( \delta \cdot \delta \right) + \left( \delta \cdot \delta \right) \right) + f \right) \cdot (b \cdot c) \\
                &\stackrel{A7}{=} a \cdot (b \cdot c)
                    + \left( \left( \delta + \delta \right) + f \right) \cdot (b \cdot c) \\
                &\stackrel{A6}{=} a \cdot (b \cdot c) + f \cdot (b \cdot c)
            \end{split}
        \end{equation}
        Anscheinend sind $t_6$ und $t_7$ nicht äquivalent
        (Es sei denn, $b$ und $f$ wären äquivalent.).

\end{enumerate}

\section*{Bonusaufgabe 13.5} 
Angenommen die Distributivaxiome würden gelten.
Dann müsste bei Verwendung dieser die gleiche Normalform entstehen, wie ohne
deren Verwendung.
Die Verwendung der Distributivaxiome ist mit $(\star)$ gekennzeichnet.
Seien im Folgenden $a, b, d$ atomare Aktionen, $H = \{ a, b \}$ und
\begin{equation}
    \gamma(x,y) =
    \begin{cases}
        c & \text{if } (x,y) = (a,b) \lor (x,y) = (b,a) \\
        \delta & \text{else}
    \end{cases}
    \text{ .}
\end{equation}
\begin{enumerate}
    \item Parallelitätsoperator
        \begin{align}
            \begin{split}
                \partial_H (a||b) &\stackrel{M1}{=} \partial_H ((a \llfloor b + b \llfloor a) + a|b) \\
                &\stackrel{LM2,CM5}{=} \partial_H ((a \cdot b + b \cdot a) + c) \\
                &\stackrel{D\{4,5,1,2\}}{=} ((\delta \cdot \delta + \delta \cdot \delta) + c) \\
                &\stackrel{A6,A7}{=} c
                \label{eq:||}
            \end{split} \\
            \begin{split}
                \partial_H (a||b) &\stackrel{(\star)}{=} \partial_H (a) || \partial_H (b) \\
                &\stackrel{D2}{=} \delta || \delta \\
                &\stackrel{CM12}{=} \delta \neq c \quad \lightning
            \end{split}
        \end{align}

    \item Leftmerge-Operator
        \begin{align}
            \begin{split}
                \partial_H ((da) \llfloor b ) &\stackrel{LM3}{=} \partial_H (d \cdot (a||b)) \\
                &\stackrel{D5}{=} \partial_H (d) \cdot \partial_H (a||b) \\
                &\stackrel{D1}{=} d \cdot \partial_H (a||b) \\
                &\stackrel{\eqref{eq:||}}{=} d \cdot c
            \end{split} \\
            \begin{split}
                \partial_H ((da) \llfloor b ) &\stackrel{(\star)}{=} \partial_H (da) \llfloor \partial_H (b) \\
                &\stackrel{D\{5,1,2\}}{=} (d\delta) \llfloor \delta \\
                &\stackrel{LM3}{=} d \cdot (\delta || \delta) \\
                &\stackrel{CM12}{=} d \cdot \delta \neq d \cdot c \quad \lightning
            \end{split}
        \end{align}

\end{enumerate}

\section*{Übungsaufgabe 13.6} 

\begin{enumerate}
    \item
        $E$ ist eine geschütze rekursize Spezifikation, da sich die rechten
        Seiten der Gleichungen in die entsprechende Form überführen lassen.
        \begin{align}
            \begin{split}
                X &= (a + b) \cdot Y + c \cdot X \\
                &= a \cdot Y + b \cdot Y + c \cdot X
            \end{split} \\
            \begin{split}
                Y &= X \cdot d + e \\
                &\stackrel{!}{=} (a \cdot Y + b \cdot Y + c \cdot X) \cdot d + e \\
                &= a \cdot (Y \cdot d) + b \cdot (Y \cdot d) + c \cdot (X \cdot d) + e \\
            \end{split}
        \end{align}

    \item \hfill \\
        \begin{figure}[h]
            \centering
            \begin{tikzpicture}[
                    auto,
                    scale=3,
                ]
                \tikzset{
                    edge/.style={
                        ->,
                        >=stealth',
                    }
                }
                \node (x)    at (0,3) {$\langle X|E \rangle$};
                \node (xd)   at (0,2) {$\langle X|E \rangle \cdot d$};
                \node (xdd)  at (0,1) {$\langle X|E \rangle \cdot d \cdot d$};
                \node (xddd) at (0,0) {};

                \node (y)    at (1,3) {$\langle Y|E \rangle$};
                \node (yd)   at (1,2) {$\langle Y|E \rangle \cdot d$};
                \node (ydd)  at (1,1) {$\langle Y|E \rangle \cdot d \cdot d$};
                \node (yddd) at (1,0) {};

                \node (chk)  at (2,3) {$\checkmark$};
                \node (d)    at (2,2) {$d$};
                \node (dd)   at (2,1) {$d \cdot d$};
                \node (ddd)  at (2,0) {};

                \draw [edge, loop left]  (x) to node {$c$} (x);
                \draw [edge, bend left]  (x) to node {$a$} (y);
                \draw [edge, bend right] (x) to node {$b$} (y);
                \draw [edge]             (y) to node {$e$} (chk);
                \draw [edge]             (y) to node {$c$} (xd);
                \draw [edge]             (d) to node {$d$} (chk);

                \draw [edge, loop left]  (xd) to node {$c$} (xd);
                \draw [edge, bend left]  (xd) to node {$a$} (yd);
                \draw [edge, bend right] (xd) to node {$b$} (yd);
                \draw [edge]             (yd) to node {$e$} (d);
                \draw [edge]             (yd) to node {$c$} (xdd);
                \draw [edge]             (dd) to node {$d$} (d);

                \draw [edge, loop left]  (xdd) to node {$c$} (xdd);
                \draw [edge, bend left]  (xdd) to node {$a$} (ydd);
                \draw [edge, bend right] (xdd) to node {$b$} (ydd);
                \draw [edge]             (ydd) to node {$e$} (dd);
                \draw [edge, dashed]     (ydd) to node {$c$} (xddd);
                \draw [edge, dashed]     (ddd) to node {$d$} (dd);
            \end{tikzpicture}
            \caption{Prozessgraph zu $\langle X|E \rangle$}
            \label{fig:recursion}
        \end{figure}

    \item
        \begin{align}
                c &\stackrel{c}{\too} \checkmark & &A_0 \\
                c \langle X|E \rangle &\stackrel{c}{\too} \langle X|E \rangle & &T_\cdot \\
                (a + b) \langle Y|E \rangle + c \langle X|E \rangle &\stackrel{c}{\too} \langle X|E \rangle & &T_{+L} \\
                \langle X|E \rangle &\stackrel{c}{\too} \langle X|E \rangle & &T_{X_i} \\
                \langle X|E \rangle \cdot d &\stackrel{c}{\too} \langle X|E \rangle \cdot d & &T_\cdot \\
                (\langle X|E \rangle \cdot d) + e &\stackrel{c}{\too} \langle X|E \rangle \cdot d & &T_{+R} \\
                \langle Y|E \rangle &\stackrel{c}{\too} \langle X|E \rangle \cdot d & &T_{X_i}
        \end{align}

\end{enumerate}

\end{document}
