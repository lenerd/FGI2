\documentclass[a4paper]{scrartcl}

% font/encoding packages
\usepackage[utf8]{inputenc}
\usepackage[T1]{fontenc}
\usepackage{lmodern}
\usepackage[ngerman]{babel}

% math
\usepackage{amsmath, amssymb, amsfonts, amsthm, mathtools, MnSymbol}
\allowdisplaybreaks
\newtheorem*{behaupt}{Behauptung}

% tikz
\usepackage{tikz}
\usetikzlibrary{arrows,automata}

% misc
\usepackage{enumerate}
\usepackage[section]{placeins}
\usepackage{subcaption}

\title{Formale Grundlagen der Informatik II}
\subtitle{Aufgabenblatt 7}
\author{
    Jan-Hendrik Briese (6523408) \\
    Lennart Braun (6523742) \\
    Marc Strothmann (6537646) \\
    Maximilian Knapperzbusch (6535090)
}
\date{zum 1. Dezember 2014}

\begin{document}
\maketitle

\section*{Übungsaufgabe 7.3} 
\begin{enumerate}
    \item
        Beschreibung der Transitionen
        \begin{description}
            \item[sg] Generiert Service-Module
            \item[wg] Generiert Weltherrschafts-Module
            \item[og] Generiert optische Sensoren
            \item[eg] Generiert Empfänger
            \item[sma] Wählt ein Service-Modul als Programmiermodul
            \item[wma] Wählt ein Weltherrschafts-Modul als Programmiermodul
            \item[kf] Baut einen Kopf zusammen
            \item[kta] Wählt einen Kopf und eine Torsoschnittstelle aus;
                inkrementiert den Counter für ausgewählte und noch nicht
                verbaute Kopf-Schnittstellen Paare
            \item[gta] Wählt eine Gliedmaße und eine Torsoschnittstelle aus
            \item[ks] Verschweißt einen Kopf mit einer Torsoschnittstelle
            \item[s] Verschweißt eine Gliedmaße mit einer Torsoschnittstelle
            \item[ks] Lagert ein verschweißtes Kopf-Schnittstellen Paar
            \item[s] Lagert ein verschweißtes Gliedmaßen-Schnittstellen Paar
            \item[gc] Wählt Gliedmaßen für den Roboter aus
            \item[gce] Beendet das Auswählen
            \item[rz] Setzt einen Roboter aus einem Kopf und den ausgewählten
                Gliedmaßen zusammen;
                decrementiert den Counter für ausgewählte und noch nicht
                verbaute Kopf-Schnittstellen Paare
        \end{description}

    \item

    \item

    \item

    \item

\end{enumerate}

\section*{Übungsaufgabe 7.4} 
\begin{enumerate}
    \item

    \item

    \item
        \begin{enumerate}[(a)]
            \item

            \item

        \end{enumerate}

\end{enumerate}

\end{document}
