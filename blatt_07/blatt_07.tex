\documentclass[a4paper]{scrartcl}

% font/encoding packages
\usepackage[utf8]{inputenc}
\usepackage[T1]{fontenc}
\usepackage{lmodern}
\usepackage[ngerman]{babel}

% math
\usepackage{amsmath, amssymb, amsfonts, amsthm, mathtools, MnSymbol}
\allowdisplaybreaks
\newtheorem*{behaupt}{Behauptung}

% tikz
\usepackage{tikz}
\usetikzlibrary{arrows,automata}
\usetikzlibrary{positioning}

% misc
\usepackage{enumerate}
\usepackage[section]{placeins}
\usepackage{subcaption}

\title{Formale Grundlagen der Informatik II}
\subtitle{Aufgabenblatt 7}
\author{
    Jan-Hendrik Briese (6523408) \\
    Lennart Braun (6523742) \\
    Marc Strothmann (6537646) \\
    Maximilian Knapperzbusch (6535090)
}
\date{zum 1. Dezember 2014}

\begin{document}
\maketitle

\section*{Übungsaufgabe 7.3} 
\begin{enumerate}
    \item
        Beschreibung der Transitionen
        \begin{description}
            \item[sg] Generiert Service-Module
            \item[wg] Generiert Weltherrschafts-Module
            \item[og] Generiert optische Sensoren
            \item[eg] Generiert Empfänger
            \item[sma] Wählt ein Service-Modul als Programmiermodul
            \item[wma] Wählt ein Weltherrschafts-Modul als Programmiermodul
            \item[kf] Baut einen Kopf zusammen
            \item[kta] Wählt einen Kopf und eine Torsoschnittstelle aus;
                inkrementiert den Counter für ausgewählte und noch nicht
                verbaute Kopf-Schnittstellen Paare
            \item[gta] Wählt eine Gliedmaße und eine Torsoschnittstelle aus
            \item[ks] Verschweißt einen Kopf mit einer Torsoschnittstelle
            \item[s] Verschweißt eine Gliedmaße mit einer Torsoschnittstelle
            \item[ks] Lagert ein verschweißtes Kopf-Schnittstellen Paar
            \item[s] Lagert ein verschweißtes Gliedmaßen-Schnittstellen Paar
            \item[gc] Wählt Gliedmaßen für den Roboter aus
            \item[gce] Beendet das Auswählen
            \item[rz] Setzt einen Roboter aus einem Kopf und den ausgewählten
                Gliedmaßen zusammen;
                decrementiert den Counter für ausgewählte und noch nicht
                verbaute Kopf-Schnittstellen Paare
        \end{description}

    \item
        Das P/T-Netz $N_1$ ist
        \begin{itemize}
            \item lebendig (Es können beliebig viele Roboter gebaut werden.)
            \item nicht beschränkt (Es können z.\,B. belibig viele Köpfe gebaut
                werden, bevor weiter gemacht wird.)
            \item reversibel (Jedes Bauteil, welches erzeugt wird, kann auch
                verbaut werden; fertige Roboter verschwinden.)
        \end{itemize}

    \item
        $N_1 = N_2$

    \item

    \item

\end{enumerate}

\section*{Übungsaufgabe 7.4} 
\begin{enumerate}
    \item \hfill \\
        \begin{figure}[h]
            \centering
            \begin{tikzpicture}[
                ->,
                >=stealth',
                semithick,
                auto,
                node distance=1.5cm,
            ]
                \node (0) {};
                \node (a) [below=0.5cm of 0] {$(0,0,2,4)^T$};
                \node (b) [below= of a]      {$(1,0,1,3)^T$};
                \node (c) [below= of b]      {$(2,0,0,2)^T$};
                \node (d) [left= of b]       {$(0,1,1,2)^T$};
                \node (e) [below= of d]      {$(1,1,0,1)^T$};

                \draw (0) to node {   } (a);
                \draw (a) to node {$v$} (b);
                \draw (b) to node {$v$} (c);
                \draw (d) to node {$v$} (e);
                \draw (d) to node {$t$} (b);
                \draw (e) to node {$t$} (c);
                \draw (c) to node {$u$} (d);
            \end{tikzpicture}
            \caption{Erreichbarkeitsgraph für $N_{7.4a}$}
        \end{figure}
        \begin{figure}[h]
            \centering
            \begin{tikzpicture}[
                ->,
                >=stealth',
                semithick,
                auto,
                node distance=1.5cm,
            ]
                \node (0) {};
                \node (aa) [below=0.5cm of 0]  {$(0,0,1,4)^T$};
                \node (ab) [below= of aa]      {$(3,0,0,3)^T$};
                \node (ac) [below= of ab]      {$(1,2,1,3)^T$};
                \node (ad) [below= of ac]      {\underline{$(3,0,1,4)^T$}};
                \node (ae) [below= of ad]      {$(1,2,2,4)^T$};
                \node (af) [below= of ae]      {$(3,0,2,5)^T$};
                \node (ag) [below= of af]      {$(1,2,3,5)^T$};
                \node (a)  [below=0.5cm of ag] {};

                \node (bc) [right= of ac]      {$(4,2,0,2)^T$};
                \node (bd) [below= of bc]      {$(6,0,0,3)^T$};
                \node (be) [below= of bd]      {$(4,2,1,3)^T$};
                \node (bf) [below= of be]      {$(6,0,1,4)^T$};
                \node (bg) [below= of bf]      {$(4,2,2,4)^T$};
                \node (b)  [below=0.5cm of bg] {};

                \node (ce) [right= of be]      {$(7,2,0,2)^T$};
                \node (cf) [below= of ce]      {$(9,0,0,3)^T$};
                \node (cg) [below= of cf]      {$(7,2,1,3)^T$};
                \node (c)  [below=0.5cm of cg] {};

                \node (dg) [right= of cg]      {$(10,2,0,2)^T$};
                \node (d)  [below=0.5cm of dg] {};

                % vertical edges
                \draw (0)  to node {   } (aa);
                \draw (aa) to node {$v$} (ab);
                \draw (ab) to node {$u$} (ac);
                \draw (ac) to node {$t$} (ad);
                \draw (ad) to node {$u$} (ae);
                \draw (ae) to node {$t$} (af);
                \draw (af) to node {$u$} (ag);
                \draw (ag) to node {$t$} (a);

                \draw (bc) to node {$t$} (bd);
                \draw (bd) to node {$u$} (be);
                \draw (be) to node {$t$} (bf);
                \draw (bf) to node {$u$} (bg);
                \draw (bg) to node {$t$} (b);

                \draw (ce) to node {$t$} (cf);
                \draw (cf) to node {$u$} (cg);
                \draw (cg) to node {$t$} (c);

                \draw (dg) to node {$t$} (d);

                % horizontal edges
                \draw (ac) to node {$v$} (bc);
                \draw (ad) to node {$v$} (bd);
                \draw (ae) to node {$v$} (be);
                \draw (af) to node {$v$} (bf);
                \draw (ag) to node {$v$} (bg);

                \draw (be) to node {$v$} (ce);
                \draw (bf) to node {$v$} (cf);
                \draw (bg) to node {$v$} (cg);

                \draw (cg) to node {$v$} (dg);

            \end{tikzpicture}
            \caption{Erreichbarkeitsgraph für $N_{7.4b}$}
        \end{figure}


    \item
        $N_{7.4a}$ ist
        \begin{itemize}
            \item beschränkt
            \item $k$-beschränkt für $k \in \{4,5\}$
            \item verklemmungsfrei
            \item lebendig
            \item $\lnot$ reversibel
            \item strukturell lebendig
        \end{itemize}

        $N_{7.4b}$ ist $\lnot$ strukturell beschränkt.

        TODO: Begründungen

    \item TODO
        \begin{enumerate}[(a)]
            \item

            \item

        \end{enumerate}

\end{enumerate}

\end{document}
