\documentclass[a4paper]{scrartcl}

% font/encoding packages
\usepackage[utf8]{inputenc}
\usepackage[T1]{fontenc}
\usepackage{lmodern}
\usepackage[ngerman]{babel}

% math
\usepackage{amsmath, amssymb, amsfonts, amsthm, mathtools}
\allowdisplaybreaks
\newtheorem*{behaupt}{Behauptung}

% tikz
\usepackage{tikz}
\usetikzlibrary{arrows}
\usetikzlibrary{automata}
\usetikzlibrary{petri}
\usetikzlibrary{positioning}

% misc
\usepackage{enumerate}
\usepackage[section]{placeins}
\usepackage{subcaption}

% macros
\newcommand{\gdw}{\Leftrightarrow}

\title{Formale Grundlagen der Informatik II}
\subtitle{Aufgabenblatt 7}
\author{
    Jan-Hendrik Briese (6523408) \\
    Lennart Braun (6523742) \\
    Marc Strothmann (6537646) \\
    Maximilian Knapperzbusch (6535090)
}
\date{zum 1. Dezember 2014}

\begin{document}
\maketitle

\section*{Übungsaufgabe 7.3} 
\begin{enumerate}
    \item
        Beschreibung der Transitionen
        \begin{description}
            \item[sg] Generiert Service-Module
            \item[wg] Generiert Weltherrschafts-Module
            \item[og] Generiert optische Sensoren
            \item[eg] Generiert Empfänger
            \item[sma] Wählt ein Service-Modul als Programmiermodul
            \item[wma] Wählt ein Weltherrschafts-Modul als Programmiermodul
            \item[kf] Baut einen Kopf zusammen
            \item[kta] Wählt einen Kopf und eine Torsoschnittstelle aus;
                inkrementiert den Counter für ausgewählte und noch nicht
                verbaute Kopf-Schnittstellen Paare
            \item[gta] Wählt eine Gliedmaße und eine Torsoschnittstelle aus
            \item[ks] Verschweißt einen Kopf mit einer Torsoschnittstelle
            \item[s] Verschweißt eine Gliedmaße mit einer Torsoschnittstelle
            \item[ks] Lagert ein verschweißtes Kopf-Schnittstellen Paar
            \item[s] Lagert ein verschweißtes Gliedmaßen-Schnittstellen Paar
            \item[gc] Wählt Gliedmaßen für den Roboter aus
            \item[gce] Beendet das Auswählen
            \item[rz] Setzt einen Roboter aus einem Kopf und den ausgewählten
                Gliedmaßen zusammen;
                decrementiert den Counter für ausgewählte und noch nicht
                verbaute Kopf-Schnittstellen Paare
        \end{description}

    \item
        Das P/T-Netz $N_1$ ist
        \begin{itemize}
            \item lebendig (Es können beliebig viele Roboter gebaut werden.)
            \item nicht beschränkt (Es können z.\,B. belibig viele Köpfe gebaut
                werden, bevor weiter gemacht wird.)
            \item reversibel (Jedes Bauteil, welches erzeugt wird, kann auch
                verbaut werden; fertige Roboter verschwinden.)
        \end{itemize}

    \item
        $N_1 = N_2$

    \item In $N_3$ können in jedem Zwischenschritt nur noch maximal 42
        Einheiten von jeder Art von Dingen gespeichert werden.

    \item \ldots

\end{enumerate}

\section*{Übungsaufgabe 7.4} 
\begin{enumerate}
    \item \hfill \\
        \begin{figure}[h]
            \centering
            \begin{tikzpicture}[
                ->,
                >=stealth',
                semithick,
                auto,
                node distance=1.5cm,
            ]
                \node (0) {};
                \node (a) [below=0.5cm of 0] {$(0,0,2,4)^T$};
                \node (b) [below= of a]      {$(1,0,1,3)^T$};
                \node (c) [below= of b]      {$(2,0,0,2)^T$};
                \node (d) [left= of b]       {$(0,1,1,2)^T$};
                \node (e) [below= of d]      {$(1,1,0,1)^T$};

                \draw (0) to node {   } (a);
                \draw (a) to node {$v$} (b);
                \draw (b) to node {$v$} (c);
                \draw (d) to node {$v$} (e);
                \draw (d) to node {$t$} (b);
                \draw (e) to node {$t$} (c);
                \draw (c) to node {$u$} (d);
            \end{tikzpicture}
            \caption{Erreichbarkeitsgraph für $N_{7.4a}$}
        \end{figure}
        \begin{figure}[h]
            \centering
            \begin{tikzpicture}[
                ->,
                >=stealth',
                semithick,
                auto,
                node distance=1.5cm,
            ]
                \node (0) {};
                \node (aa) [below=0.5cm of 0]  {$(0,0,1,4)^T$};
                \node (ab) [below= of aa]      {$(3,0,0,3)^T$};
                \node (ac) [below= of ab]      {$(1,2,1,3)^T$};
                \node (ad) [below= of ac]      {\underline{$(3,0,1,4)^T$}};
                \node (ae) [below= of ad]      {$(1,2,2,4)^T$};
                \node (af) [below= of ae]      {$(3,0,2,5)^T$};
                \node (ag) [below= of af]      {$(1,2,3,5)^T$};
                \node (a)  [below=0.5cm of ag] {};

                \node (bc) [right= of ac]      {$(4,2,0,2)^T$};
                \node (bd) [below= of bc]      {$(6,0,0,3)^T$};
                \node (be) [below= of bd]      {$(4,2,1,3)^T$};
                \node (bf) [below= of be]      {$(6,0,1,4)^T$};
                \node (bg) [below= of bf]      {$(4,2,2,4)^T$};
                \node (b)  [below=0.5cm of bg] {};

                \node (ce) [right= of be]      {$(7,2,0,2)^T$};
                \node (cf) [below= of ce]      {$(9,0,0,3)^T$};
                \node (cg) [below= of cf]      {$(7,2,1,3)^T$};
                \node (c)  [below=0.5cm of cg] {};

                \node (dg) [right= of cg]      {$(10,2,0,2)^T$};
                \node (d)  [below=0.5cm of dg] {};

                % vertical edges
                \draw (0)  to node {   } (aa);
                \draw (aa) to node {$v$} (ab);
                \draw (ab) to node {$u$} (ac);
                \draw (ac) to node {$t$} (ad);
                \draw (ad) to node {$u$} (ae);
                \draw (ae) to node {$t$} (af);
                \draw (af) to node {$u$} (ag);
                \draw (ag) to node {$t$} (a);

                \draw (bc) to node {$t$} (bd);
                \draw (bd) to node {$u$} (be);
                \draw (be) to node {$t$} (bf);
                \draw (bf) to node {$u$} (bg);
                \draw (bg) to node {$t$} (b);

                \draw (ce) to node {$t$} (cf);
                \draw (cf) to node {$u$} (cg);
                \draw (cg) to node {$t$} (c);

                \draw (dg) to node {$t$} (d);

                % horizontal edges
                \draw (ac) to node {$v$} (bc);
                \draw (ad) to node {$v$} (bd);
                \draw (ae) to node {$v$} (be);
                \draw (af) to node {$v$} (bf);
                \draw (ag) to node {$v$} (bg);

                \draw (be) to node {$v$} (ce);
                \draw (bf) to node {$v$} (cf);
                \draw (bg) to node {$v$} (cg);

                \draw (cg) to node {$v$} (dg);

            \end{tikzpicture}
            \caption{Erreichbarkeitsgraph für $N_{7.4b}$}
        \end{figure}


    \item
        $N_{7.4a}$ ist
        \begin{itemize}
            \item beschränkt (Es ist möglich, den Erreichbarkeitsgraph zu
                zeichnen.)
            \item $k$-beschränkt für $k \in \{4,5\}$ (In keiner Markierung
                liegen auf einem Platz mehr als 4 Marken.)
            \item verklemmungsfrei (In jeder Markierung ist mindestens eine
                Transition aktiviert.)
            \item lebendig (Alle Transitionen sind lebendig.)
            \item $\lnot$ reversibel (Die Markierung $M_{0a}$ kann nicht wieder
                erreicht werden.)
            \item strukturell lebendig ($(N_{7.4a}, m_{0a})$ ist lebendig.)
        \end{itemize}

        $N_{7.4b}$ ist $\lnot$ strukturell beschränkt, da $(N_{7.4b}, m_{0b})$
        nicht beschränkt ist.

    \item
        Gegeben sei ein P/T-Netz ohne Senken
        ($\forall p \in P : p^\bullet \cap T \neq \emptyset$), für das gilt:
        \begin{equation}
            \forall t \in T : \sum_{p \in P} \widetilde{W}(t, p)
            = 2 \cdot \sum_{p \in P} \widetilde{W}(p, t) - 1
            \label{eq:def}
        \end{equation}
        
        \begin{enumerate}[(a)]
            \item
                Abbildung \ref{fig:beispiel} zeigt ein Netz, welches die
                Definition erfüllt und sowohl lebendig als auch beschränkt ist.
                \begin{figure}[h]
                    \centering
                    \begin{tikzpicture}
                        \tikzstyle{place}=[circle, thick, draw, minimum size=6mm]
                        \tikzstyle{transition}=[rectangle, thick, draw, minimum size=4mm]

                        \node [place, tokens=1] (p1) [label=above:$p_1$] {};
                        \node [transition] (t1) [left= of p1, label=above:$t_1$] {} 
                            edge [pre, bend right] (p1)
                            edge [post, bend left] (p1);
                    \end{tikzpicture}
                    \caption{Beispielnetz}
                    \label{fig:beispiel}
                \end{figure}

            \item
                \begin{behaupt}
                    In einem P/T-Netz, welches die Definition \eqref{eq:def}
                    erfüllt, wird die Summe nachfolgender Markierungen nie
                    kleiner.
                    \begin{equation}
                        \forall m, m' \in R(N, m_0) \emph{ mit } m \stackrel{w}{\to} m'
                        : |m'| \geq |m|
                    \end{equation}
                    \begin{proof}[Beweis durch vollständige Induktion] \hfill \\
                        \textbf{Induktionsanfang} \\
                        Sei $|w| = 0 \gdw w = \varepsilon$ und $m, m' \in R(N, m_0)$.
                        \begin{equation}
                            m \stackrel{w}{\to} m' \gdw m = m_0 \Rightarrow |m| = |m'|
                        \end{equation}

                        Gelte die Behauptung für eine Transitionsfolge der
                        festen Länge $n \in \mathbb{N}$.

                        \textbf{Induktionsschritt} \\
                        Sei $|w| = n + 1$ mit $w = \sigma t$ und $|t| = 1$.
                        Für $m, m', m'' \in R(N, m_0)$ gelte
                        $m \stackrel{\sigma}{\to} m'' \stackrel{t}{\to} m'$

                        Nach Induktionsannahme gilt $|m''| \geq |m|$.

                        Damit $|m'| \geq |m''|$ gilt, müssen nach $t$ auf jedem
                        Platz mindestens genauso viele Marken liegen, wie
                        zuvor.

                        Seien $a, e$ wie folgt definiert.
                        \begin{align}
                            a &= \sum_{p \in P} \widetilde{W}(t, p)
                            \qquad \text{ (von $t$ ausgehende Kantengewichte)} \\
                            e &= \sum_{p \in P} \widetilde{W}(p, t)
                            \qquad \text{ (eingehende Kantengewichte)}
                        \end{align}
                        
                        Wenn $a \geq e$ gilt, dann sind nach der Ausführung
                        der Transition $a - e$ mehr Marken auf den
                        Plätzen vorhanden.

                        Aus der Definition \eqref{eq:def} kennen wir den
                        Zusammenhang von $a$ und $e$.
                        \begin{equation}
                            a = 2e-1
                        \end{equation}

                        Da die Kantengewichte aus natürlichen Zahlen bestehen,
                        muss also gelten
                        \begin{equation}
                            \forall e \in \mathbb{N} \backslash \{0\} : 2e-1 \geq e
                        \end{equation}

                        Offensichtlich ist dies der Fall, auf einen Beweis
                        verzichten wir an dieser Stelle.
                        
                        Es ist gezeigt, dass $|m'| \geq |m''|$ und damit die
                        Behauptung gelten.
                    \end{proof}
                \end{behaupt}

        \end{enumerate}

\end{enumerate}

\end{document}
