\documentclass[a4paper]{scrartcl}

% font/encoding packages
\usepackage[utf8]{inputenc}
\usepackage[T1]{fontenc}
\usepackage{lmodern}
\usepackage[ngerman]{babel}

% math
\usepackage{amsmath, amssymb, amsfonts, amsthm, mathtools, MnSymbol}
\allowdisplaybreaks
\newtheorem*{behaupt}{Behauptung}

% tikz
\usepackage{tikz}
\usetikzlibrary{arrows,automata}

% misc
\usepackage{enumerate}
\usepackage[section]{placeins}
\usepackage{subcaption}

\title{Formale Grundlagen der Informatik II}
\subtitle{Aufgabenblatt 5}
\author{
    Jan-Hendrik Briese (6523408) \\
    Lennart Braun (6523742) \\
    Marc Strothmann (6537646) \\
    Maximilian Knapperzbusch (6535090)
}
\date{zum 17. November 2014}

\begin{document}
\maketitle

\section*{Übungsaufgabe 5.3} 
\begin{enumerate}
    \item Abwicklungsgraph von $M_{AKW}$ \\
        \begin{center}
            \begin{tikzpicture}[->,>=stealth',shorten >=1pt,auto,node distance=1.8cm,
	   semithick]
                \tikzstyle{every state}=[fill=white,draw=black,text=black]
		
				\node[initial, state] 	(A)						{$\{b\}$};
				\node[state]			(B)	[below right of=A]	{$\{b\}$};
				\node[state]			(C)	[below right of=B]	{$\{b\}$};
				\node[state]			(D)	[below right of=C]	{$\{b\}$};
				\node[state][draw=white,text=white]	(E)	[below right of=D]	{$\{b\}$};
				\node[state]			(F) [below left of=A]		{$\{s\}$};
				\node[state]			(G)	[below left of=B]		{$\{s\}$};
				\node[state]			(H)	[below left of=C]		{$\{s\}$};	
				\node[state][draw=white,text=white]	(J)	[below left of=D]		{$\{s\}$};	
				\node[state]			(K)	[below left of=F]		{$\{g\}$};
				\node[state]			(L)	[below left of=K]		{$\{g\}$};
				\node[state]			(M)	[below left of=G]		{$\{g\}$};
				\node[state][draw=white,text=white]	(N)	[below left of=M]		{$\{g\}$};
				\node[state][draw=white,text=white]	(O)	[below left of=H]		{$\{g\}$};
				\node[state][draw=white,text=white]	(P)	[below left of=L]		{$\{g\}$};

					\path 	(A) 	edge 			 	node { } (B)
									edge				node { } (F)
							(B)		edge				node { } (C)
									edge				node { } (G)
							(C)		edge				node { } (D)
									edge				node { } (H)
							(D)		edge				node { } (J)
									edge				node { } (E)
							(F)		edge				node { } (K)
							(K)		edge				node { } (L)
							(L)		edge				node { } (P)
							(F)		edge				node { } (K)
							(G)		edge				node { } (M)
							(M)		edge				node { } (N)
							(H)		edge				node { } (O);
            \end{tikzpicture}
        \end{center}

    \item Sei $\alpha_1 = \textbf{EX}(s \lor g)$.
        \begin{enumerate}[(a)]
            \item
                \begin{equation}
                    Sat(\alpha_1) = \left\{ s_0, s_1, s_2 \right\}
                \end{equation}

            \item
                \begin{equation}
                    Sat(\textbf{AG}\alpha_1) = \left\{ s_0, s_1, s_2 \right\}
                \end{equation}

            \item
                \begin{equation}
                    Sat(\textbf{EG} \lnot b) = \left\{ s_1, s_2 \right\}
                \end{equation}

            \item
                \begin{equation}
                    Sat(\textbf{AX} \lnot g) = \left\{ s_0 \right\}
                \end{equation}

        \end{enumerate}

    \item Sei $\beta_1=\textbf{AGEX}(s\vee g)$ und $\beta_2=\textbf{EG}\neg b$
        \begin{enumerate}[(a)]
            \item $M_{AKW}\models\textbf{AGEX}(s\vee g)$ \\
                $\beta_1$ sagt aus: Auf allen Pfaden gilt, dass immer ein Pfade
                existiert, mit dem im nächsten Schritt $s$ oder $g$ erreicht
                werden kann.
                $\beta_1$ trifft zu, da $s_0$ der einzige Zustand ist, in dem
                $\neg(s\vee g)$, jedoch $s_0\in Sat(\textbf{AGEX}(s\vee g))$.

            \item $M_{AKW}\models\textbf{EG}\neg b$ \\
                Die Aussage für $\beta_2$ ist, dass es einen Pfad gibt, auf dem
                immer $\neg b$ gilt.
                Dies trifft gem. 5.3.2 nicht zu, da $s_0$ Startzustand ist und
                $s_0\notin Sat(\textbf{EG}\neg b)$ gilt.

        \end{enumerate}

    \item
        \begin{enumerate}[(a)]
            \item TODO
                
            \item
                \begin{itemize}
                    \item $M_{AKW} \models \textbf{EGEX}s$ \\
                        Sei $\pi = s_0^\omega$.
                        Es gäbe in jedem Schritt die Möglichkeit, nach $s_1$
                        abzubiegen.

                    \item $M_{AKW} \not\models \textbf{EXEG}s$ \\
                        Es gibt keinen Pfad, der ab dem zweiten Schritt immer
                        in $s_1$ bleibt.

                \end{itemize}
                
        \end{enumerate}

    \item
        \begin{enumerate}[(a)]
            \item
                
            \item
                \begin{itemize}
                    \item $M_{AKW} \models \textbf{EG}\lnot s$ \\
                        Es gibt einen Pfad $\pi = s_0^\omega$, welcher $s_1$
                        nie betritt.
                        Auf $\pi$ gilt somit nie $s$.

                    \item $M_{AKW} \not\models \textbf{G}\lnot s$ \\
                        Da die Formel z.\,B. für den Pfad
                        $\pi' = s_0s_1s_2^\omega$ nicht gilt, kann sie nicht für
                        alle Pfade, die in Anfangszuständen starten, gelten.

                \end{itemize}
                
        \end{enumerate}

\end{enumerate}

\section*{Übungsaufgabe 5.4}
\begin{enumerate}
    \item
        \begin{align}
            \begin{split}
                \beta_1 &= \textbf{AGEX} \lnot s \\
                &\equiv \lnot \textbf{EF} (\lnot \textbf{EX} \lnot s) \\
                &\equiv \lnot \textbf{E} (true \textbf{U} (\lnot \textbf{EX} \lnot s)) = \beta_1'
            \end{split} \\
            \begin{split}
                \beta_2 &= \textbf{EXAG} \lnot s \\
                &\equiv \textbf{EX} (\lnot \textbf{EF} s) \\
                &\equiv \textbf{EX} (\lnot \textbf{E} (true \textbf{U} s)) = \beta_2'
            \end{split}
        \end{align}
        

    \item \hfill \\
        \begin{table}[h]
            \begin{subtable}{0.5\textwidth}
                \begin{tabular}{r|ccc}
                    Teilformel & $s_0$ & $s_1$ & $s_2$ \\ \hline
                    $s$ & $-$ & $+$ & $-$ \\
                    $\lnot s$ & $+$ & $-$ & $+$ \\
                    $\textbf{EX} \lnot s$ & $+$ & $+$ & $+$ \\
                    $\lnot \textbf{EX} \lnot s$ & $-$ & $-$ & $-$ \\
                    $\textbf{E} (true \textbf{U} (\lnot \textbf{EX} \lnot s))$ & $-$ & $-$ & $-$ \\
                    $\lnot \textbf{E} (true \textbf{U} (\lnot \textbf{EX} \lnot s))$ & $+$ & $+$ & $+$ \\
                \end{tabular}
            \end{subtable}
            \begin{subtable}{0.5\textwidth}
                \begin{tabular}{r|ccc}
                    Teilformel & $s_0$ & $s_1$ & $s_2$ \\ \hline
                    $s$ & $-$ & $+$ & $-$ \\
                    $\textbf{E} (true \textbf{U} s)$ & $+$ & $+$ & $-$ \\
                    $\lnot \textbf{E} (true \textbf{U} s)$ & $-$ & $-$ & $+$ \\
                    $\textbf{EX}(\lnot \textbf{E} (true \textbf{U} s))$ & $-$ & $+$ & $+$ \\
                \end{tabular}
            \end{subtable}
        \end{table}

    \item
        \begin{itemize}
            \item Da $\beta_1$ für den einzigen Startzustand $s_0$ von $M_{AKW}$
                gilt, folgt $M_{AKW} \models \beta_1$.
            \item Da $\beta_2$ nicht für den Startzustand $s_0$ von $M_{AKW}$
                gilt, folgt $M_{AKW} \not\models \beta_2$.
        \end{itemize}

    \item Interessant.

\end{enumerate}

\end{document}
