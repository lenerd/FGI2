\documentclass[a4paper]{scrartcl}

% font/encoding packages
\usepackage[utf8]{inputenc}
\usepackage[T1]{fontenc}
\usepackage{lmodern}
\usepackage[ngerman]{babel}

% math
\usepackage{amsmath, amssymb, amsfonts, amsthm, mathtools, MnSymbol}
\allowdisplaybreaks
\newtheorem*{behaupt}{Behauptung}

% tikz
\usepackage{tikz}
\usetikzlibrary{arrows,automata}

% misc
\usepackage{enumerate}
\usepackage[section]{placeins}
\usepackage{subcaption}

\title{Formale Grundlagen der Informatik II}
\subtitle{Aufgabenblatt 5}
\author{
    Jan-Hendrik Briese (6523408) \\
    Lennart Braun (6523742) \\
    Marc Strothmann (6537646) \\
    Maximilian Knapperzbusch (6535090)
}
\date{zum 17. November 2014}

\begin{document}
\maketitle

\section*{Übungsaufgabe 5.3} 
\begin{enumerate}
    \item TODO: Graph

    \item Sei $\alpha_1 = \textbf{EX}(s \lor g)$.
        \begin{enumerate}[(a)]
            \item
                \begin{equation}
                    Sat(\alpha_1) = \left\{ s_0, s_1, s_2 \right\}
                \end{equation}

            \item
                \begin{equation}
                    Sat(\textbf{AG}\alpha_1) = \left\{ s_0, s_1, s_2 \right\}
                \end{equation}

            \item
                \begin{equation}
                    Sat(\textbf{EG} \lnot b) = \left\{ s_1, s_2 \right\}
                \end{equation}

            \item
                \begin{equation}
                    Sat(\textbf{AX} \lnot g) = \left\{ s_0 \right\}
                \end{equation}

        \end{enumerate}

    \item
        \begin{enumerate}[(a)]
            \item
                
            \item
                
        \end{enumerate}

    \item
        \begin{enumerate}[(a)]
            \item
                
            \item
                
        \end{enumerate}

    \item
        \begin{enumerate}[(a)]
            \item
                
            \item
                
        \end{enumerate}

\end{enumerate}

\section*{Übungsaufgabe 5.4}
\begin{enumerate}
    \item
        \begin{align}
            \begin{split}
                \beta_1 &= \textbf{AGEX} \lnot s \\
                &\equiv \lnot \textbf{EF} (\lnot \textbf{EX} \lnot s) \\
                &\equiv \lnot \textbf{E} (true \textbf{U} (\lnot \textbf{EX} \lnot s)) = \beta_1'
            \end{split} \\
            \begin{split}
                \beta_2 &= \textbf{EXAG} \lnot s \\
                &\equiv \textbf{EX} (\lnot \textbf{EF} s) \\
                &\equiv \textbf{EX} (\lnot \textbf{E} (true \textbf{U} s)) = \beta_2'
            \end{split}
        \end{align}
        

    \item \hfill \\
        \begin{table}[h]
            \begin{subtable}{0.5\textwidth}
                \begin{tabular}{r|ccc}
                    Teilformel & $s_0$ & $s_1$ & $s_2$ \\ \hline
                    $s$ & $-$ & $+$ & $-$ \\
                    $\lnot s$ & $+$ & $-$ & $+$ \\
                    $\textbf{EX} \lnot s$ & $+$ & $+$ & $+$ \\
                    $\lnot \textbf{EX} \lnot s$ & $-$ & $-$ & $-$ \\
                    $\textbf{E} (true \textbf{U} (\lnot \textbf{EX} \lnot s))$ & $-$ & $-$ & $-$ \\
                    $\lnot \textbf{E} (true \textbf{U} (\lnot \textbf{EX} \lnot s))$ & $+$ & $+$ & $+$ \\
                \end{tabular}
            \end{subtable}
            \begin{subtable}{0.5\textwidth}
                \begin{tabular}{r|ccc}
                    Teilformel & $s_0$ & $s_1$ & $s_2$ \\ \hline
                    $s$ & $-$ & $+$ & $-$ \\
                    $\textbf{E} (true \textbf{U} s)$ & $+$ & $+$ & $-$ \\
                    $\lnot \textbf{E} (true \textbf{U} s)$ & $-$ & $-$ & $+$ \\
                    $\textbf{EX}(\lnot \textbf{E} (true \textbf{U} s))$ & $-$ & $+$ & $+$ \\
                \end{tabular}
            \end{subtable}
        \end{table}

    \item
        \begin{itemize}
            \item Da $\beta_1$ für den einzigen Startzustand $s_0$ von $M_{AKW}$
                gilt, folgt $M_{AKW} \models \beta_1$.
            \item Da $\beta_2$ nicht für den Startzustand $s_0$ von $M_{AKW}$
                gilt, folgt $M_{AKW} \not\models \beta_2$.
        \end{itemize}

    \item Interessant.

\end{enumerate}

\end{document}
