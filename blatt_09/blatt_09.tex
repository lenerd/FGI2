\documentclass[a4paper]{scrartcl}

% font/encoding packages
\usepackage[utf8]{inputenc}
\usepackage[T1]{fontenc}
\usepackage{lmodern}
\usepackage[ngerman]{babel}

% math
\usepackage{amsmath, amssymb, amsfonts, amsthm, mathtools}
\allowdisplaybreaks
\newtheorem*{behaupt}{Behauptung}

% tikz
\usepackage{tikz}
\usetikzlibrary{arrows}
\usetikzlibrary{automata}
\usetikzlibrary{petri}
\usetikzlibrary{positioning}

% misc
\usepackage{enumerate}
\usepackage[section]{placeins}
\usepackage{subcaption}

% macros
\newcommand{\gdw}{\Leftrightarrow}

\title{Formale Grundlagen der Informatik II}
\subtitle{Aufgabenblatt 9}
\author{
    Jan-Hendrik Briese (6523408) \\
    Lennart Braun (6523742) \\
    Marc Strothmann (6537646) \\
    Maximilian Knapperzbusch (6535090)
}
\date{zum 15. Dezember 2014}

\begin{document}
\maketitle

\section*{Übungsaufgabe 9.3} 
\begin{enumerate}
    \item
        \begin{behaupt}
            Beschränktheit ist eine monotone Eigenschaft.
            Wenn ein P/T-Netz unter der Anfangsmarkierung $\textbf{m}_0$
            unbeschränkt ist, ist es auch unter jeder größeren Anfangsmarkierung
            $\textbf{m}_0' \geq \textbf{m}_0$ unbeschränkt.
        \end{behaupt}
        \begin{proof} \hfill \\
            \begin{enumerate}
                \item $\textbf{m}_0 = \textbf{m}_0'$ \\
                    In der Behauptung wird vorausgesetzt, dass $N$ unter
                    $\textbf{m}_0$ unbeschränkt ist. 

                \item $\textbf{m}_0 > \textbf{m}_0'$ \\
                    Sei $\Delta' = \textbf{m}_0' - \textbf{m}_0 > 0$.
                    Wenn $N$ unter $\textbf{m}_0$ unbeschränkt ist, so gibt es
                    eine Transitionsfolge $\sigma \in T^*$ mit
                    $\textbf{m}_0 \stackrel{\sigma}{\to} \textbf{m}_1$ und
                    $\textbf{m}_1 > \textbf{m}_0$.
                    Sei $\Delta_\sigma = \textbf{m}_1 - \textbf{m}_0 > 0$.
                    Da $\textbf{m}_0' > \textbf{m}_0$ gilt, kann die
                    Transitionsfolge $\sigma$ auch von $\textbf{m}_0'$
                    ausgeführt werden:
                    $\textbf{m}_0' \stackrel{\sigma}{\to} \textbf{m}_1'$.
                    Damit ist Bedingung a) von Satz 7.1 erfüllt.
                    \begin{equation}
                        \begin{split}
                            \textbf{m}_0 &< \textbf{m}_1 \\
                            \gdw \textbf{m}_0 &< \textbf{m}_0 + \Delta_\sigma \\
                            \gdw \textbf{m}_0 + \Delta' &< \textbf{m}_0 + \Delta' + \Delta_\sigma \\
                            \gdw \textbf{m}_0' &< \textbf{m}_0' + \Delta_\sigma \\
                            \gdw \textbf{m}_0' &< \textbf{m}_1'
                        \end{split}
                        \label{eq:9-3-1}
                    \end{equation}
                    Nach Gleichung \ref{eq:9-3-1} ist auch die Bedingung b) von
                    Satz 7.1 erfüllt.
                    
            \end{enumerate}
        \end{proof}

    \item
        \begin{equation}
            B = \left\{
                \begin{pmatrix}
                    0 \\ 0 \\ 0 \\ 2
                \end{pmatrix}
                ,
                \begin{pmatrix}
                    4 \\ 0 \\ 0 \\ 0
                \end{pmatrix}
                ,
                \begin{pmatrix}
                    0 \\ 6 \\ 0 \\ 0
                \end{pmatrix}
                ,
                \begin{pmatrix}
                    2 \\ 3 \\ 0 \\ 0
                \end{pmatrix}
                ,
                \begin{pmatrix}
                    3 \\ 2 \\ 0 \\ 0
                \end{pmatrix}
                ,
                \begin{pmatrix}
                    1 \\ 5 \\ 0 \\ 0
                \end{pmatrix}
            \right\}
        \end{equation}
        \paragraph{Erläuterung:} \hfill \\
        Für alle $m \in B$ muss $m_3 = 0$ gelten, da keine Transition aus $N$
        Marken von $p_3$ entfernt.
        \begin{itemize}
            \item $m = \begin{pmatrix} 0 & 0 & 0 & 2 \end{pmatrix}^T \in B$ \\
                Damit $c$ aktiviert ist, müssen mindestens 2 Marken auf $p_4$
                liegen.
                Bei jedem Feuern von $c$ werden wieder zwei Marken nach $p_4$
                gelegt, sodass $c$ anschließend wieder aktiviert ist.
                Zusätzlich wird die Anzahl der Marken auf $p_3$ um eine erhöht.
                \begin{equation}
                    \begin{pmatrix}
                        m_1 & m_2 & m_3 & 2
                    \end{pmatrix}
                    \stackrel{c}{\to}
                    \begin{pmatrix}
                        m_1 & m_2 & m_3 + 1 & 2
                    \end{pmatrix}
                \end{equation}
                Es ist zu sehen, dass die Folgemarkierung immer größer als die
                ursprüngliche ist.
                Damit ist $(N, m)$ unbeschränkt.
                Es existiert keine Markierung $m' < m$, da $c$ als einzige
                Transition Marken aus $p_4$ benötigt und dort 2 liegen müssen
                (s.\,o.).

            \item
                \begin{equation}
                    \begin{pmatrix}
                        4 \\ 0 \\ 0 \\ 0
                    \end{pmatrix}
                    \stackrel{a}{\to}
                    \begin{pmatrix}
                        2 \\ 3 \\ 0 \\ 0
                    \end{pmatrix}
                    \stackrel{a}{\to}
                    \begin{pmatrix}
                        0 \\ 6 \\ 0 \\ 0
                    \end{pmatrix}
                    \stackrel{b}{\to}
                    \begin{pmatrix}
                        3 \\ 2 \\ 2 \\ 0
                    \end{pmatrix}
                    \stackrel{a}{\to}
                    \begin{pmatrix}
                        1 \\ 5 \\ 2 \\ 0
                    \end{pmatrix}
                    \stackrel{b}{\to}
                    \begin{pmatrix}
                        4 \\ 1 \\ 4 \\ 0
                    \end{pmatrix}
                \end{equation}
                

        \end{itemize}
        

\end{enumerate}

\section*{Übungsaufgabe 9.4} 
\begin{enumerate}
    \item
        \begin{equation}
            \Delta_{N_{9.4a}} =
            \begin{pmatrix}
                 1 & -1 &  0 &  0 &  0 &  0 \\
                 0 &  0 &  1 & -1 &  0 &  0 \\
                 0 &  0 & -1 &  1 &  0 &  0 \\
                 0 &  0 &  0 &  0 & -1 &  1 \\
                 0 &  1 &  0 &  0 & -1 &  0 \\
            \end{pmatrix}
        \end{equation}
        

    \item
        Die S-Invariantenvektoren $i \in \mathbb{Z}^{|P|}$ von $N_{9.4a}$ sind
        die Lösungen der Gleichung
        \begin{equation}
            \Delta_{N_{9.4a}}^T \cdot i = 0 \text{ .}
        \end{equation}
        Die Lösung des entsprechenden Gleichungssystems lautet
        \begin{equation}
            i =
            \begin{pmatrix}
                0 \\ a \\ a \\ 0 \\ 0
            \end{pmatrix}
            \quad
            a \in \mathbb{Z} \backslash \{ 0 \}
        \end{equation}
        Damit gilt für die Menge der S-Invariantenvektoren $S$
        \begin{equation}
            S = \left\{ 
                \begin{pmatrix}
                    0 & a & a & 0 & 0
                \end{pmatrix}^T
                \ \vert \ 
                a \in \mathbb{Z} \backslash \{ 0 \}
            \right\}
            \text{ .}
        \end{equation}
        
        

    \item

    \item
        \begin{equation}
            \Delta_{N_{9.4b}} =
            \begin{pmatrix}
                 1 & -1 &  0 &  0 &  0 &  0 \\
                 0 &  0 &  1 & -1 &  0 &  0 \\
                 0 &  0 & -1 &  1 &  0 &  0 \\
                 0 &  0 &  0 &  0 & -1 &  1 \\
                 0 &  1 &  0 &  0 & -1 &  0 \\
                -1 &  1 &  0 &  0 &  0 &  0 \\
                 0 & -1 &  0 &  0 &  1 &  0 \\
                 0 &  0 &  0 &  0 &  1 & -1 \\
            \end{pmatrix}
        \end{equation}
        Die S-Invariantenvektoren $i \in \mathbb{Z}^{|P|}$ von $N_{9.4a}$ sind
        die Lösungen der Gleichung
        \begin{equation}
            \Delta_{N_{9.4a}}^T \cdot i = 0 \text{ .}
        \end{equation}
        Die Lösung des entsprechenden Gleichungssystems lautet
        \begin{equation}
            i =
            \begin{pmatrix}
                a \\ b \\ b \\ c \\ d \\ a \\ d \\ c
            \end{pmatrix}
            \quad
            a, b, c, d \in \mathbb{Z} \land i \neq 0
        \end{equation}
        Damit gilt für die Menge der S-Invariantenvektoren $S$
        \begin{equation}
            S = \left\{ 
                \begin{pmatrix}
                    a & b & b & c & d & a & d & c
                \end{pmatrix}^T
                \ \vert \ 
                a, b, c, d \in \mathbb{Z}
                \land
                \lnot \left( a = b = c = d = 0 \right)
            \right\}
            \text{ .}
        \end{equation}

    \item
        Die hinzugefügten Plätze stellen jeweils eine Kapazitätsbeschränkung zu
        einem anderen Platz dar.
        Ausgehend von einer Startmarkierung $\textbf{m}_0$ muss in jeder
        erreichbaren Markierung $m$
        ($\textbf{m}_0 \stackrel{w}{\to} \textbf{m}, w \in T^*$)
        \begin{equation}
            \begin{gathered}
                \textbf{m}(p_1) + \textbf{m}(p_6) = \textbf{m}_0(p_1) + \textbf{m}_0(p_6) \\
                \land \\
                \textbf{m}(p_5) + \textbf{m}(p_7) = \textbf{m}_0(p_5) + \textbf{m}_0(p_7) \\
                \land \\
                \textbf{m}(p_4) + \textbf{m}(p_8) = \textbf{m}_0(p_4) + \textbf{m}_0(p_8)
            \end{gathered}
        \end{equation}
        gelten.
        In dem Kontext der Apotheke, lassen sich die Beschränkungen wie folgt
        übersetzen.
        \begin{itemize}
            \item
                An der Annahmestelle ($p_1$) ist nicht unbegrenzt Platz, daher
                müssen ggf. erst Dinge ins Lager eingeräumt werden, bevor die
                nächste Lieferung angenommen werden kann.

            \item
                Das Lager ($p_5$) hat ebenfalls nur eine begrenzte Kapazität:
                Es können nicht beliebig viele Medikamente vor Ort gelagert
                werden.

            \item
                Es passen nicht beliebig viele Kunden in den Verkaufsraum
                ($p_4$).
                Ist dieser voll, müssen die Kunden draußen warten, bis genug
                Platz ist, damit sie die Apotheke betreten können (damit $f$
                aktiviert ist).

        \end{itemize}

    \item
        \begin{align}
            i_1 &= \left( 2, 1, 1, 3, 3, 2, 3, 3 \right)^T \\
            m_0 &= \left( 0, 1, 2, 0, 1, 5, 10, 8 \right)^T
        \end{align}
        Da $i_1$ ein Invariantenvektor ist, gilt nach Satz 7.32
        \begin{equation}
            \begin{gathered}
                i_1^T \cdot \textbf{m} = i_1^T \cdot \textbf{m}_0 \quad \gdw \\
                2m_1 + m_2 + m_3 + 3m_4 + 3m_5 + 2m_6 + 3m_7 + 3m_8 = 70
                \text{ .}
            \end{gathered}
        \end{equation}
        


\end{enumerate}

\end{document}
