\documentclass[a4paper]{scrartcl}

% font/encoding packages
\usepackage[utf8]{inputenc}
\usepackage[T1]{fontenc}
\usepackage{lmodern}
\usepackage[ngerman]{babel}

% math
\usepackage{amsmath, amssymb, amsfonts, amsthm, mathtools}
\allowdisplaybreaks
\newtheorem*{behaupt}{Behauptung}

% tikz
\usepackage{tikz}
\usetikzlibrary{arrows}
\usetikzlibrary{automata}
\usetikzlibrary{petri}
\usetikzlibrary{positioning}

% misc
\usepackage{enumerate}
\usepackage[section]{placeins}
\usepackage{subcaption}

% macros
\newcommand{\gdw}{\Leftrightarrow}

\title{Formale Grundlagen der Informatik II}
\subtitle{Aufgabenblatt 9}
\author{
    Jan-Hendrik Briese (6523408) \\
    Lennart Braun (6523742) \\
    Marc Strothmann (6537646) \\
    Maximilian Knapperzbusch (6535090)
}
\date{zum 15. Dezember 2014}

\begin{document}
\maketitle

\section*{Übungsaufgabe 9.3} 
\begin{enumerate}
    \item
        \begin{behaupt}
            Beschränktheit ist eine monotone Eigenschaft.
            Wenn ein P/T-Netz unter der Anfangsmarkierung $\textbf{m}_0$
            unbeschränkt ist, ist es auch unter jeder größeren Anfangsmarkierung
            $\textbf{m}_0' \geq \textbf{m}_0$ unbeschränkt.
        \end{behaupt}
        \begin{proof} \hfill \\
            \begin{enumerate}
                \item $\textbf{m}_0 = \textbf{m}_0'$ \\
                    In der Behauptung wird vorausgesetzt, dass $N$ unter
                    $\textbf{m}_0$ unbeschränkt ist. 

                \item $\textbf{m}_0 > \textbf{m}_0'$ \\
                    Sei $\Delta' = \textbf{m}_0' - \textbf{m}_0 > 0$.
                    Wenn $N$ unter $\textbf{m}_0$ unbeschränkt ist, so gibt es
                    eine Transitionsfolge $\sigma \in T^*$ mit
                    $\textbf{m}_0 \stackrel{\sigma}{\to} \textbf{m}_1$ und
                    $\textbf{m}_1 > \textbf{m}_0$.
                    Sei $\Delta_\sigma = \textbf{m}_1 - \textbf{m}_0 > 0$.
                    Da $\textbf{m}_0' > \textbf{m}_0$ gilt, kann die
                    Transitionsfolge $\sigma$ auch von $\textbf{m}_0'$
                    ausgeführt werden:
                    $\textbf{m}_0' \stackrel{\sigma}{\to} \textbf{m}_1'$.
                    Damit ist Bedingung a) von Satz 7.1 erfüllt.
                    \begin{equation}
                        \begin{split}
                            \textbf{m}_0 &< \textbf{m}_1 \\
                            \gdw \textbf{m}_0 &< \textbf{m}_0 + \Delta_\sigma \\
                            \gdw \textbf{m}_0 + \Delta' &< \textbf{m}_0 + \Delta' + \Delta_\sigma \\
                            \gdw \textbf{m}_0' &< \textbf{m}_0' + \Delta_\sigma \\
                            \gdw \textbf{m}_0' &< \textbf{m}_1'
                        \end{split}
                        \label{eq:9-3-1}
                    \end{equation}
                    Nach Gleichung \ref{eq:9-3-1} ist auch die Bedingung b) von
                    Satz 7.1 erfüllt.
                    
            \end{enumerate}
        \end{proof}

    \item

\end{enumerate}

\section*{Übungsaufgabe 9.4} 
\begin{enumerate}
    \item

    \item

    \item

    \item

    \item

    \item

\end{enumerate}

\end{document}
