\documentclass[a4paper]{scrartcl}

% font/encoding packages
\usepackage[utf8]{inputenc}
\usepackage[T1]{fontenc}
\usepackage{lmodern}
\usepackage[ngerman]{babel}

% math
\usepackage{amsmath, amssymb, amsfonts, amsthm, mathtools}
\allowdisplaybreaks
\newtheorem*{behaupt}{Behauptung}

% tikz
\usepackage{tikz}
\usetikzlibrary{arrows}
\usetikzlibrary{automata}
\usetikzlibrary{petri}
\usetikzlibrary{positioning}

% misc
\usepackage{enumerate}
\usepackage[section]{placeins}
\usepackage{subcaption}

% macros
\newcommand{\gdw}{\Leftrightarrow}

\title{Formale Grundlagen der Informatik II}
\subtitle{Aufgabenblatt 10}
\author{
    Jan-Hendrik Briese (6523408) \\
    Lennart Braun (6523742) \\
    Marc Strothmann (6537646) \\
    Maximilian Knapperzbusch (6535090)
}
\date{zum 5. Januar 2015}

\begin{document}
\maketitle

\section*{Übungsaufgabe 10.3} 
\begin{enumerate}
    \item
        Die Menge aller $T$-Invarianten-Vektoren ist die Menge $J$ der Lösungen
        $j$ des folgenden Gleichungssystems.
        \begin{equation}
            \begin{split}
                \Delta_{N_{10.3}} \cdot \textbf{j} &= 0 \\
                \begin{pmatrix}
                    -1 & -1 &  3 &  2 & -1 \\
                     0 & -1 &  3 &  0 &  0 \\
                     0 &  1 & -3 &  0 &  0 \\
                     1 &  2 & -6 & -2 &  1 \\
                \end{pmatrix}
                \cdot
                \begin{pmatrix}
                    j_1 \\ j_2 \\ j_3 \\ j_4 \\ j_5
                \end{pmatrix}
                &=
                \begin{pmatrix}
                    0 \\ 0 \\ 0 \\ 0
                \end{pmatrix}
            \end{split}
        \end{equation}
        \begin{equation}
            J = \left\{
                \textbf{j} = \left( j_1, j_2, j_3, j_4, j_5 \right)^T
                \in \mathbb{N}^5
                \ |\  j_1 - 2j_2 + j_5 = 0 \land j_2 - j_3 = 0
                \right\}
        \end{equation}
        
    \item
        \begin{align}
            \textbf{j} &= \left( 1, 3, 1, 1, 1 \right)^T \\
            w &= aedb^3c
        \end{align}
        \begin{equation}
            \textbf{m}_0 =
            \begin{pmatrix}
                3 \\ 3 \\ 0 \\ 0
            \end{pmatrix}
            \stackrel{a}{\to}
            \begin{pmatrix}
                2 \\ 3 \\ 0 \\ 1
            \end{pmatrix}
            \stackrel{e}{\to}
            \begin{pmatrix}
                1 \\ 3 \\ 0 \\ 2
            \end{pmatrix}
            \stackrel{d}{\to}
            \begin{pmatrix}
                3 \\ 3 \\ 0 \\ 0
            \end{pmatrix}
            \stackrel{b}{\to}
            \begin{pmatrix}
                2 \\ 2 \\ 1 \\ 2
            \end{pmatrix}
            \stackrel{b}{\to}
            \begin{pmatrix}
                1 \\ 1 \\ 2 \\ 4
            \end{pmatrix}
            \stackrel{b}{\to}
            \begin{pmatrix}
                0 \\ 0 \\ 3 \\ 6
            \end{pmatrix}
            \stackrel{c}{\to}
            \begin{pmatrix}
                3 \\ 3 \\ 0 \\ 0
            \end{pmatrix}
            = \textbf{m}_0
        \end{equation}

\end{enumerate}

\section*{Übungsaufgabe 10.4} 
\begin{enumerate}
    \item
        \begin{behaupt}
            Wenn ein Siphon $A$ eines einfachen P/T-Netzes $N$ in $\textbf{m}_0$
            unmarkiert ist, dann bleibt er dies auch in allen von $\textbf{m}_0$
            aus erreichbaren Markierungen.
        \end{behaupt}
        \begin{proof}
            Nach Definition 7.45 benötigt jede Transition $t$, welche Marken in
            $A$ generiert, Marken aus $A$.
            Da $A$ unmarkiert ist, wird kein solche Transition $t$ aktiviert und
            $A$ bleibt unmarkiert.
        \end{proof}

    \item

    \item

\end{enumerate}

\section*{Übungsaufgabe 10.5} 
\begin{enumerate}
    \item

    \item

    \item

\end{enumerate}

\end{document}
