\documentclass[a4paper]{scrartcl}

% font/encoding packages
\usepackage[utf8]{inputenc}
\usepackage[T1]{fontenc}
\usepackage{lmodern}
\usepackage[ngerman]{babel}

% math
\usepackage{amsmath, amssymb, amsfonts, amsthm, mathtools}
\allowdisplaybreaks
\newtheorem*{behaupt}{Behauptung}

% tikz
\usepackage{tikz}
\usetikzlibrary{arrows}
\usetikzlibrary{automata}
\usetikzlibrary{shapes}
\usetikzlibrary{petri}
\usetikzlibrary{positioning}

% misc
\usepackage{enumerate}
\usepackage[section]{placeins}
\usepackage{subcaption}

% macros
\newcommand{\gdw}{\Leftrightarrow}

\title{Formale Grundlagen der Informatik II}
\subtitle{Aufgabenblatt 10}
\author{
    Jan-Hendrik Briese (6523408) \\
    Lennart Braun (6523742) \\
    Marc Strothmann (6537646) \\
    Maximilian Knapperzbusch (6535090)
}
\date{zum 5. Januar 2015}

\begin{document}
\maketitle

\section*{Übungsaufgabe 10.3} 
\begin{enumerate}
    \item
        Die Menge aller $T$-Invarianten-Vektoren ist die Menge $J$ der Lösungen
        $j$ des folgenden Gleichungssystems.
        \begin{equation}
            \begin{split}
                \Delta_{N_{10.3}} \cdot \textbf{j} &= 0 \\
                \begin{pmatrix}
                    -1 & -1 &  3 &  2 & -1 \\
                     0 & -1 &  3 &  0 &  0 \\
                     0 &  1 & -3 &  0 &  0 \\
                     1 &  2 & -6 & -2 &  1 \\
                \end{pmatrix}
                \cdot
                \begin{pmatrix}
                    j_1 \\ j_2 \\ j_3 \\ j_4 \\ j_5
                \end{pmatrix}
                &=
                \begin{pmatrix}
                    0 \\ 0 \\ 0 \\ 0
                \end{pmatrix}
            \end{split}
        \end{equation}
        \begin{equation}
            J = \left\{
                \textbf{j} = \left( j_1, j_2, j_3, j_4, j_5 \right)^T
                \in \mathbb{N}^5
                \ |\  j_1 - 2j_2 + j_5 = 0 \land j_2 - j_3 = 0
                \right\}
        \end{equation}
        
    \item
        \begin{align}
            \textbf{j} &= \left( 1, 3, 1, 1, 1 \right)^T \\
            w &= aedb^3c
        \end{align}
        \begin{equation}
            \textbf{m}_0 =
            \begin{pmatrix}
                3 \\ 3 \\ 0 \\ 0
            \end{pmatrix}
            \stackrel{a}{\to}
            \begin{pmatrix}
                2 \\ 3 \\ 0 \\ 1
            \end{pmatrix}
            \stackrel{e}{\to}
            \begin{pmatrix}
                1 \\ 3 \\ 0 \\ 2
            \end{pmatrix}
            \stackrel{d}{\to}
            \begin{pmatrix}
                3 \\ 3 \\ 0 \\ 0
            \end{pmatrix}
            \stackrel{b}{\to}
            \begin{pmatrix}
                2 \\ 2 \\ 1 \\ 2
            \end{pmatrix}
            \stackrel{b}{\to}
            \begin{pmatrix}
                1 \\ 1 \\ 2 \\ 4
            \end{pmatrix}
            \stackrel{b}{\to}
            \begin{pmatrix}
                0 \\ 0 \\ 3 \\ 6
            \end{pmatrix}
            \stackrel{c}{\to}
            \begin{pmatrix}
                3 \\ 3 \\ 0 \\ 0
            \end{pmatrix}
            = \textbf{m}_0
        \end{equation}

\end{enumerate}

\section*{Übungsaufgabe 10.4} 
\begin{enumerate}
    \item
        \begin{behaupt}
            Wenn ein Siphon $A$ eines einfachen P/T-Netzes $N$ in $\textbf{m}_0$
            unmarkiert ist, dann bleibt er dies auch in allen von $\textbf{m}_0$
            aus erreichbaren Markierungen.
        \end{behaupt}
        \begin{proof}
            Nach Definition 7.45 benötigt jede Transition $t$, welche Marken in
            $A$ generiert, Marken aus $A$.
            Da $A$ unmarkiert ist, wird kein solche Transition $t$ aktiviert und
            $A$ bleibt unmarkiert.
        \end{proof}

    \item
        Seien $F$ die Menge aller Fallen und $S$ die Menge aller Siphone.
        \begin{equation}
            \begin{split}
                F = \Big\{
                    & \emptyset,
                    \left\{ p_2, p_3 \right\},
                    \left\{ p_1, p_2, p_3 \right\},
                    \left\{ p_1, p_3, p_5 \right\},
                    \left\{ p_2, p_3, p_4 \right\},
                    \left\{ p_2, p_3, p_5 \right\}, \\
                    & \left\{ p_1, p_2, p_3, p_4 \right\},
                    \left\{ p_1, p_2, p_3, p_5 \right\},
                    \left\{ p_1, p_3, p_4, p_5 \right\},
                    \left\{ p_2, p_3, p_4, p_5 \right\}, \\
                    & \left\{ p_1, p_2, p_3, p_4, p_5 \right\}
                \Big\}
            \end{split}
        \end{equation}
        \begin{equation}
            \begin{split}
                S = \Big\{
                    & \emptyset,
                    \left\{ p_4 \right\},
                    \left\{ p_4, p_5 \right\},
                    \left\{ p_1, p_4, p_5 \right\},
                    \left\{ p_2, p_4, p_5 \right\},
                    \left\{ p_1, p_2, p_4, p_5 \right\}, \\
                    & \left\{ p_1, p_3, p_4, p_5 \right\},
                    \left\{ p_2, p_3, p_4, p_5 \right\},
                    \left\{ p_1, p_2, p_3, p_4, p_5 \right\},
                \Big\}
            \end{split}
        \end{equation}

    \item
        \begin{equation}
            \textbf{m}_0 = \left( 0, 0, 0, 1, 1 \right)^T
        \end{equation}
        \begin{equation}
            \begin{split}
                \sigma_a = \lambda \quad &\Rightarrow \quad 
                \textbf{m}_0 \stackrel{a}{\longrightarrow} \\
                \sigma_b = a \quad &\Rightarrow \quad 
                \textbf{m}_0 \stackrel{ab}{\longrightarrow} \\
                \sigma_c = a \quad &\Rightarrow \quad 
                \textbf{m}_0 \stackrel{ac}{\longrightarrow} \\
                \sigma_d = ab \quad &\Rightarrow \quad 
                \textbf{m}_0 \stackrel{abd}{\longrightarrow} \\
                \sigma_e = a \quad &\Rightarrow \quad 
                \textbf{m}_0 \stackrel{ae}{\longrightarrow}
            \end{split}
        \end{equation}
        
\end{enumerate}

\section*{Übungsaufgabe 10.5} 
\begin{enumerate}
    \item
        Farbmengen $\in \mathcal{C}$:
        \begin{align}
            beds &= \left\{ b_0, b_1, b_2 \right\} \\
            childen &= \left\{ c_0, c_1, c_2 \right\} \\
            dolls &= \left\{ d_0, d_1, d_2 \right\}
        \end{align}
        
        Funktionen:
        \begin{align}
            \begin{split}
                & bed\colon children \to beds \\
                & c_i \mapsto b_i \text{ für } i \in \mathbb{Z}_3
            \end{split}
            \begin{split}
                & dolls\colon children \to Bag(dolls) \\
                & c_i \mapsto d_i + d_{(i + 1 \mod 3)} \text{ für } i \in \mathbb{Z}_3
            \end{split}
        \end{align}
        
        \begin{figure}[h]
            \centering
            \begin{tikzpicture}[
                    auto,
                    scale=2,
                ]
                \tikzstyle{edge}=[->,>=stealth']
                \tikzstyle{place}=[ellipse, thick, draw, minimum height=15mm, minimum width=35mm]
                \tikzstyle{transition}=[rectangle, thick, draw, minimum size=10mm]

                \node [place] (play)  [label=above:Play]  at (0,0) {};
                \node [place] (dolls) [label=above:Dolls] at (0,2) {'$d_0$', '$d_1$', '$d_2$'};
                \node [place] (eat)   [label=above:Eat]   at (0,4) {'$c_0$', '$c_1$', '$c_2$'};
                \node [place] (beds)  [label=above:Beds]  at (0,6) {'$b_0$', '$b_1$', '$b_2$'};
                \node [place] (sleep) [label=above:Sleep] at (0,8) {};

                \node [transition] (a) at (2,2) {};
                \node [transition] (b) at (2,6) {};
                \node [transition] (c) at (3,7) {};

                % a
                \draw [edge] (eat) to node {$x$} (a);
                \draw [edge] (dolls) to node {$dolls(x)$} (a);
                \draw [edge] (a) to node {$x$} (play);
                % b
                \draw [edge, bend left=45] (sleep) to node {$x$} (b);
                \draw [edge] (b) to node {$bed(x)$} (beds);
                \draw [edge] (b) to node {$x$} (eat);
                % c
                \draw [edge] (beds) to node {$bed(x)$} (c);
                \draw [edge, bend right=45] (play) to node {$x$} (c);
                \draw [edge, bend right] (c) to node {$x$} (sleep);
                \draw [edge, bend left=30] (c) to node {$dolls(x)$} (dolls);
            \end{tikzpicture}
            \caption{Kinder \dots}
            \label{fig:children}
        \end{figure}

    \item
        Variablen:
        \begin{equation}
            dom(x) = children
        \end{equation}
        Farbzuweisungsabbildung:
        \begin{equation}
            \begin{split}
                & cd \colon P \to \mathcal{C} \\
                & Sleep \mapsto \left\{ c_0, c_1, c_2 \right\} \\
                & Beds \mapsto \left\{ b_0, b_1, b_2 \right\} \\
                & Eat \mapsto \left\{ c_0, c_1, c_2 \right\} \\
                & Dolls \mapsto \left\{ d_0, d_1, d_2 \right\} \\
                & Play \mapsto \left\{ c_0, c_1, c_2 \right\}
            \end{split}
        \end{equation}

    \item

\end{enumerate}

\end{document}
