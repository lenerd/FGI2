\documentclass[a4paper]{scrartcl}

% font/encoding packages
\usepackage[utf8]{inputenc}
\usepackage[T1]{fontenc}
\usepackage{lmodern}
\usepackage[ngerman]{babel}

% math
\usepackage{amsmath, amssymb, amsfonts, amsthm, mathtools, MnSymbol}
\allowdisplaybreaks
\newtheorem*{behaupt}{Behauptung}

% tikz
\usepackage{tikz}
\usetikzlibrary{arrows,automata}

% misc
\usepackage{enumerate}

\title{Formale Grundlagen der Informatik II}
\subtitle{Aufgabenblatt 3}
\author{
    Jan-Hendrik Briese (6523408) \\
    Lennart Braun (6523742) \\
    Marc Strothmann (6537646) \\
    Maximilian Knapperzbusch (6535090)
}
\date{zum 3. November 2014}

\begin{document}
\maketitle

\section*{Übungsaufgabe 3.3} 
Schnitt von $\omega$-Sprachen
\begin{enumerate}[1.]
    \item Bestimmen Sie $L(A_1 ), L(A_2 ), L^{\omega}(A_1 ) und L^{\omega}(A_2 )$.
    \item Konstruieren Sie die initiale Zusammenhangskomponente des Produktautomaten 
        $A_3$ im Sinne von Satz 1.8 bzw. Lemma 1.19. \textit{Hinweis:} Sie benötigen 
        3 Zustände.
    \item Bestimmen Sie $L(A_3)$ und $L^{\omega}(A_3)$. Vergleichen Sie $L(A_3)$ mit 
        $L(A_1) \cap L(A_2)$ und $L^{\omega}(A_3)$ mit $L^{\omega}(A_1) \cap L^{\omega}(A_2)$.
    \item Konstruieren Sie die initiale Zusammenhangskomponente des Produktautomaten 
        $A_4$ im Sinne von Satz 1.21. \textit{Hinweis:} Sie benötigen 6 Zustände.
    \item Bestimmen Sie $L(A_4)$ und $L^{\omega}(A 4 )$. Vergleichen Sie $L(A_4)$ mit 
        $L(A_1) \cap L(A_2)$ und $L^{\omega}(A_4)$ mit $L^{\omega}(A_1) \cap L^{\omega}(A_2)$.
\end{enumerate}
\section*{Übungsaufgabe 3.4}
Prüfen Sie für alle Zweierkombinationen der folgenden vier Transitionssysteme, ob 
diese bisimilar sind. Geben Sie für die bisimilaren Kombinationen die 
Bisimulationsrelation explizit an. Zeigen Sie für nichtbisimilaren Kombinationen, 
dass keine Bisimulationsrelation angegeben werden kann. Hinweis: Sie können sich 
Arbeit sparen, wenn sie beachten, dass folgende Symmetrie gilt: 
$TS_1 \underline{\leftrightarrow} TS_2$ impliziert $TS_2 \underline{\leftrightarrow} TS_1$.
\begin{description}
    \item[$TS_1 \underline{\not\leftrightarrow} TS_2$:] \hfill\\ 
        $(P_0, Q_0), (P_1, Q_1), (P_2, Q_1) \lightning$ \\
        Nachdem die Automaten mit eine a-Kante in $(P_1, Q_1)$ bzw. $(P_2, Q_1)$ 
        gewechselt haben, kann $P_1 \stackrel{c}{\rightarrow} P_3$ bzw. $P_2 \stackrel{b}{\rightarrow} P_3$
        nicht ausgeführt werden, allerdings führt eine b- und eine c-Kante von 
        $Q_1 \rightarrow Q_2$.
    \item[$TS_1 \underline{\leftrightarrow} TS_3$:] \hfill\\
        $(P_0, R_0), (P_1, R_1), (P_1, R_2), (P_2, R_1), (P_2, R_2) \lightning$ \\
        $R_1$ und $R_2$ verfügen über b- und c-Kanten, $P_1$ und $P_2$ verfügen
        nur jeweils über eine b- oder c-Kante.
    \item[$TS_1 \underline{\leftrightarrow} TS_4$:] \hfill\\
        $(P_0, S_0), (P_0, S_1), (P_1, S_2), (P_1, S_3), (P_2, S_2), (P_2, S_3), \lightning$ \\
        $P_1$ verfügt über eine b-Kante, $S_3$ nicht. \\
        $P_2$ verfügt über eine c-Kante, $S_2$ nicht. \\
        $S_2$ verfügt über eine b-Kante, $P_2$ nicht. \\
        $S_3$ verfügt über eine b-Kante, $P_1$ nicht. \\
\end{description}
\end{document}
