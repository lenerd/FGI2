\documentclass[a4paper]{scrartcl}

% font/encoding packages
\usepackage[utf8]{inputenc}
\usepackage[T1]{fontenc}
\usepackage{lmodern}
\usepackage[ngerman]{babel}

% math
\usepackage{amsmath, amssymb, amsfonts, amsthm, mathtools, MnSymbol}
\allowdisplaybreaks
\newtheorem*{behaupt}{Behauptung}

% tikz
\usepackage{tikz}
\usetikzlibrary{arrows,automata}

% misc
\usepackage{enumerate}

\title{Formale Grundlagen der Informatik II}
\subtitle{Aufgabenblatt 3}
\author{
    Jan-Hendrik Briese (6523408) \\
    Lennart Braun (6523742) \\
    Marc Strothmann (6537646) \\
    Maximilian Knapperzbusch (6535090)
}
\date{zum 3. November 2014}

\begin{document}
\maketitle

\section*{Übungsaufgabe 3.3} 
Schnitt von $\omega$-Sprachen
\begin{enumerate}[1.]
    \item Bestimmen Sie $L(A_1)$, $L(A_2)$, $L^{\omega}(A_1)$ und $L^{\omega}(A_2)$. \\
        Darstellung der Sprachen in Form von Regulären Ausdrücken
        \begin{align*}
            L(A_1) &= \left( a + b \right)^* \\
            \begin{split}
                L(A_2) &= a^*b \cdot \left( a + ba^*b \right)^* \\
                &= \left( a^*ba^*b \right)^* \cdot a^*ba^*
            \end{split} \\
            L^\omega(A_1) &= \left( a + ba^*b \right)^\omega + \left( a + ba^*b \right)^* \cdot b a^\omega \\
            L^\omega(A_2) &= a^* b \cdot \left( a + ba^*b \right)^\omega
        \end{align*}

    \item Konstruieren Sie die initiale Zusammenhangskomponente des Produktautomaten 
        $A_3$ im Sinne von Satz 1.8 bzw. Lemma 1.19. \textit{Hinweis:} Sie benötigen 
        3 Zustände. \\
        \begin{equation*}
        	\begin{tikzpicture}[->,>=stealth',shorten >=1pt,auto,node distance=2.5cm,
           semithick]
        		\tikzstyle{every state}=[fill=white,draw=black,text=black]
        	
        			\node[initial,state] 	(A)					{$q,s$};
        			\node[state, accepting]	(B)	[right of=A]	{$p,t$};
        			\node[state]			(C)	[below of=A]	{$r,t$};			
        	
        				\path 	(A) 	edge [loop above] 	node {a} (A)
        								edge 				node {b} (B)
        								edge [bend left]	node {b} (C)
        						(B)		edge [loop above]	node {a} (B)
        						(C)		edge [bend left]	node {b} (A)
        								edge [loop below]	node {a} (C);
        	\end{tikzpicture}
        \end{equation*}

    \item Bestimmen Sie $L(A_3)$ und $L^{\omega}(A_3)$. Vergleichen Sie $L(A_3)$ mit 
        $L(A_1) \cap L(A_2)$ und $L^{\omega}(A_3)$ mit $L^{\omega}(A_1) \cap L^{\omega}(A_2)$. \\
        \begin{align*}
            L(A_3) &= \left( a^*ba^*b \right)^* \cdot a^*ba^* \\
            L^\omega(A_3) &= \left( a^*ba^*b \right)^* \cdot a^*ba^\omega
        \end{align*}
        \begin{equation}
            L(A_3) = L(A_2) = L(A_1) \cup L(A_2)
        \end{equation}
        \begin{equation}
            L^\omega(A_3) \subseteq L^\omega(A_1) \cup L^\omega(A_2)
        \end{equation}
        \newline
        Die Sprachen für $A_3$ lauten $L(A_3)=(a+ba^{*}b)^{*}ba^{*}$ und $L^{\omega}(A_3)=(a+ba^{*}b)^{*}b(a)^{\omega}$. Es gilt $L(A_1)\cap L(A_2)= a^{*}b(a+ba^{*}b)^{*} = L(A_2)$. Es gilt weiter $L(A_1)\cap L(A_2) = L(A_3)$. Außerdem gilt $L^{\omega}(A_1)\cap L^{\omega}(A_2) \supseteq (a+ba^{*}b)^{*}b(a)^{\omega}=L^{\omega}(A_3)$ gemäß \textit{Lemma 1.19}.
        \newline
        TODO

    \item Konstruieren Sie die initiale Zusammenhangskomponente des Produktautomaten 
        $A_4$ im Sinne von Satz 1.21. \textit{Hinweis:} Sie benötigen 6 Zustände. \\
        \begin{equation*}
        	\begin{tikzpicture}[->,>=stealth',shorten >=1pt,auto,node distance=3.3cm,
           semithick]
        		\tikzstyle{every state}=[fill=white,draw=black,text=black]
        	
        			\node[initial,accepting, state] 	(A)		{$q,s,1$};
        			\node[state]	(B)	[right of=A]	{$r,t,2$};
        			\node[state]			(C)	[above of=B]	{$r,t,1$};
        			\node[state]			(D)	[below of=A]	{$p,t,2$};
        			\node[state, accepting]			(E)	[below of=D]	{$p,t,1$};
        			\node[state]			(F) [right of=D]	{$q,s,2$};		
        	
        				\path 	(A) 	edge 			 	node {a} (F)
        								edge [bend left]	node {b} (B)
        								edge 				node {b} (D)
        						(B)		edge [bend left]	node {b} (A)
        								edge 				node {a} (C)
        						(C)		edge				node {b} (A)
        								edge [loop right]	node {a} (C)
        						(D)		edge [bend left]	node {a} (E)
        						(E)		edge [bend left]	node {a} (D)
        						(F)		edge [loop right]	node {a} (F)
        								edge 				node {b} (D)
        								edge 				node {b} (B);
        	\end{tikzpicture}
        \end{equation*}

    \item Bestimmen Sie $L(A_4)$ und $L^{\omega}(A_4)$. Vergleichen Sie $L(A_4)$ mit 
        $L(A_1) \cap L(A_2)$ und $L^{\omega}(A_4)$ mit $L^{\omega}(A_1) \cap L^{\omega}(A_2)$. \\
        \newline
        Für $A_4$ ergeben sich die Sprachen $L(A_4) = (a^* b a^* b)^* \cdot (a^* b a (aa)^* + \epsilon)$ und $L^\omega (A_4) = (a^* b a^* b)^\omega + (a^* b a^* b)^* \cdot (a^*b) \cdot (aa)^\omega$. Damit gilt $L(A_1)\cap L(A_2)\neq L(A_4)$, da $baab\in L(A_4)$, jedoch $baab\notin L(A_2)$ und damit $baab\notin L(A_1)\cap L(A_2)$. Für die Omega-Sprachen gilt gemäß der unter 4. verwendeten Konstruktionsvorschrift (\textit{Satz 1.21 im Skript}) $L^{\omega}(A_4)=L^{\omega}(A_1)\cap L^{\omega}(A_2)$

\end{enumerate}
\section*{Übungsaufgabe 3.4}
Prüfen Sie für alle Zweierkombinationen der folgenden vier Transitionssysteme, ob 
diese bisimilar sind. Geben Sie für die bisimilaren Kombinationen die 
Bisimulationsrelation explizit an. Zeigen Sie für nichtbisimilaren Kombinationen, 
dass keine Bisimulationsrelation angegeben werden kann. Hinweis: Sie können sich 
Arbeit sparen, wenn sie beachten, dass folgende Symmetrie gilt: 
$TS_1 \underline{\leftrightarrow} TS_2$ impliziert $TS_2 \underline{\leftrightarrow} TS_1$.
\begin{description}
    \item $TS_1$, $TS_2$
        \begin{itemize}
            \item Aus Bedingung a) folgt $(P_0, Q_0) \in \mathcal{B}$.
            \item Wenn $(P_0, Q_0) \in \mathcal{B}$, dann nach b) auch
                $(P_1, Q_1), (P_2, Q_1) \in \mathcal{B}$.
            \item Da $Q_1 \stackrel{c}{\rightarrow} Q_2$ müsste die Aktion $c$
                auch in $P_1$ möglich sein. $\lightning$
        \end{itemize}
        $TS_1$ und $TS_2$ sind nicht bisimilar
        ($TS_1  \underline{\not\leftrightarrow} TS_2$).

    \item $TS_1$, $TS_3$
        \begin{itemize}
            \item Aus Bedingung a) folgt $(P_0, R_0) \in \mathcal{B}$.
            \item Aus Bedingung b) folgt, dass $(P_1, R_1) \in \mathcal{B}$
                oder $(P_1, R_2) \in \mathcal{B}$ gelten muss.
                \begin{enumerate}
                    \item $(P_1, R_1) \in \mathcal{B}$ \\
                        Da $R_1 \stackrel{c}{\rightarrow} R_4$ müsste die Aktion
                        $c$ auch in $P_1$ möglich sein. $\lightning$
                    \item $(P_1, R_2) \in \mathcal{B}$ \\
                        Da $R_2 \stackrel{c}{\rightarrow} R_4$ müsste die Aktion
                        $c$ auch in $P_1$ möglich sein. $\lightning$
                \end{enumerate}
        \end{itemize}
        $TS_1$ und $TS_3$ sind nicht bisimilar
        ($TS_1  \underline{\not\leftrightarrow} TS_3$).

    \item $TS_1$, $TS_4$
        \begin{itemize}
            \item Aus Bedingung a) folgt $(P_0,S_0), (P_0,S_1) \in \mathcal{B}$.
            \item Wenn $(P_0,S_0), (P_0,S_1) \in \mathcal{B}$, dann nach b) auch
                \\ $(P_1,S_2), (P_1,S_3), (P_2,S_2), (P_2,S_3) \in \mathcal{B}$.
            \item Da $P_1 \stackrel{b}{\rightarrow} P_3$ müsste die Aktion $b$
                auch in $S_3$ möglich sein. $\lightning$
        \end{itemize}
        $TS_1$ und $TS_4$ sind nicht bisimilar
        ($TS_1  \underline{\not\leftrightarrow} TS_4$).

    \item $TS_2$, $TS_3$
        \begin{equation*}
            \mathcal{B} = \left\{ (Q_0,R_0), (Q_1,R_1), (Q_1,R_2), (Q_2,R_3), (Q_2,R_4), (Q_3,R_0) \right\}
        \end{equation*}
        Alle drei Bedingungen sind erfüllt: $TS_2$ und $TS_3$ sind bisimilar
        ($TS_2  \underline{\leftrightarrow} TS_3$).

    \item $TS_2$, $TS_4$
        \begin{itemize}
            \item Aus Bedingung a) folgt $(Q_0,S_0), (Q_0,S_1) \in \mathcal{B}$.
            \item Wenn $(Q_0,S_0), (Q_0,S_1) \in \mathcal{B}$, dann nach b) auch
                $(Q_1,S_2), (Q_1,S_3) \in \mathcal{B}$.
            \item Da $Q_1 \stackrel{b}{\rightarrow} Q_3$ müsste die Aktion $b$
                auch in $S_3$ möglich sein. $\lightning$
        \end{itemize}
        $TS_2$ und $TS_4$ sind nicht bisimilar
        ($TS_2  \underline{\not\leftrightarrow} TS_4$).

    \item $TS_3$, $TS_4$
        \begin{itemize}
            \item Aus Bedingung a) folgt $(R_0,S_0), (R_0,S_1) \in \mathcal{B}$.
            \item Wenn $(R_0,S_0), (R_0,S_1) \in \mathcal{B}$, dann nach b) auch
                \\ $(R_1,S_2), (R_1,S_3), (R_2,S_2), (R_2,S_3) \in \mathcal{B}$.
            \item Da $R_1 \stackrel{c}{\rightarrow} R_4$ müsste die Aktion $b$
                auch in $S_2$ möglich sein. $\lightning$
        \end{itemize}
        $TS_3$ und $TS_4$ sind nicht bisimilar
        ($TS_3  \underline{\not\leftrightarrow} TS_4$).

    % \item[$TS_1 \underline{\not\leftrightarrow} TS_2$:] \hfill\\ 
    %     $(P_0, Q_0), (P_1, Q_1), (P_2, Q_1) \lightning$ \\
    %     Nachdem die Automaten mit eine a-Kante in $(P_1, Q_1)$ bzw. $(P_2, Q_1)$ 
    %     gewechselt haben, kann $P_1 \stackrel{c}{\rightarrow} P_3$ bzw. $P_2 \stackrel{b}{\rightarrow} P_3$
    %     nicht ausgeführt werden, allerdings führt eine b- und eine c-Kante von 
    %     $Q_1 \rightarrow Q_2$.
    % \item[$TS_1 \underline{\leftrightarrow} TS_3$:] \hfill\\
    %     $(P_0, R_0), (P_1, R_1), (P_1, R_2), (P_2, R_1), (P_2, R_2) \lightning$ \\
    %     $R_1$ und $R_2$ verfügen über b- und c-Kanten, $P_1$ und $P_2$ verfügen
    %     nur jeweils über eine b- oder c-Kante.
    % \item[$TS_1 \underline{\leftrightarrow} TS_4$:] \hfill\\
    %     $(P_0, S_0), (P_0, S_1), (P_1, S_2), (P_1, S_3), (P_2, S_2), (P_2, S_3), \lightning$ \\
    %     $P_1$ verfügt über eine b-Kante, $S_3$ nicht. \\
    %     $P_2$ verfügt über eine c-Kante, $S_2$ nicht. \\
    %     $S_2$ verfügt über eine b-Kante, $P_2$ nicht. \\
    %     $S_3$ verfügt über eine b-Kante, $P_1$ nicht. \\
\end{description}
\end{document}
