\documentclass[a4paper]{scrartcl}

% font/encoding packages
\usepackage[utf8]{inputenc}
\usepackage[T1]{fontenc}
\usepackage{lmodern}
\usepackage[ngerman]{babel}

% math
\usepackage{amsmath, amssymb, amsfonts, amsthm, mathtools}
\allowdisplaybreaks
\newtheorem*{behaupt}{Behauptung}

% tikz
\usepackage{tikz}
\usetikzlibrary{arrows,automata}

% misc
\usepackage{enumerate}
\usepackage{geometry}
\usepackage{listings}

\geometry{top=2cm, bottom=2cm, right=2cm, left=2cm}
\parindent0pt

\title{FGI2}
\subtitle{Aufgabenblatt 03 - Lösungen}
\author{}
\parindent0pt

\date{}

\begin{document}
	\maketitle

zu 3.3:
\begin{enumerate}
	\item
	\begin{equation*}
	\begin{split}
		& L(A_1)= (a+b)^{*}\\
		& L_(A_2)= a^{*}b(a+ba^{*}b)^{*}\\
		& L^{\omega} (A_1)= (a+ba^{*}b)^{\omega}+(a+ba^{*}b)^{*}\cdot ba^{\omega}\\
		& L^{\omega}(A_2)=a^{*}b(a+ba^{*}b)^{\omega}\\
	\end{split}
	\end{equation*}
	\item Gemäß \textbf{Satz 1.8} und \textbf{Lemma 1.19} ergibt sich folgender Produktautomat: 
\begin{equation*}
	\begin{tikzpicture}[->,>=stealth',shorten >=1pt,auto,node distance=2.5cm,
   semithick]
		\tikzstyle{every state}=[fill=white,draw=black,text=black]
	
			\node[initial,state] 	(A)					{$q,s$};
			\node[state, accepting]	(B)	[right of=A]	{$p,t$};
			\node[state]			(C)	[below of=A]	{$r,t$};			
	
				\path 	(A) 	edge [loop above] 	node {a} (A)
								edge 				node {b} (B)
								edge [bend left]	node {b} (C)
						(B)		edge [loop above]	node {a} (B)
						(C)		edge [bend left]	node {b} (A)
								edge [loop below]	node {a} (C);
	\end{tikzpicture}
\end{equation*}
\item Die Sprachen für $A_3$ lauten $L(A_3)=(a+ba^{*}b)^{*}ba^{*}$ und $L^{\omega}(A_3)=(a+ba^{*}b)^{*}b(a)^{\omega}$. Es gilt $L(A_1)\cap L(A_2)= a^{*}b(a+ba^{*}b)^{*} = L(A_2)$. Es gilt weiter $L(A_1)\cap L(A_2) \supseteq L(A_3)$. Außerdem gilt $L^{\omega}(A_1)\cap L^{\omega}(A_2)=(a+ba^{*}b)^{*}b(a)^{\omega}=L^{\omega}(L_3)$.


\newpage
\item
\begin{equation*}
	\begin{tikzpicture}[->,>=stealth',shorten >=1pt,auto,node distance=2.5cm,
   semithick]
		\tikzstyle{every state}=[fill=white,draw=black,text=black]
	
			\node[initial,accepting, state] 	(A)		{$q,s,1$};
			\node[state]	(B)	[right of=A]	{$r,t,2$};
			\node[state]			(C)	[above of=B]	{$r,t,1$};
			\node[state]			(D)	[below of=A]	{$p,t,2$};
			\node[state, accepting]			(E)	[below of=D]	{$p,t,1$};
			\node[state]			(F) [right of=D]	{$q,s,2$};		
	
				\path 	(A) 	edge 			 	node {a} (F)
								edge [bend left]	node {b} (B)
								edge 				node {b} (D)
						(B)		edge [bend left]	node {b} (A)
								edge 				node {a} (C)
						(C)		edge				node {b} (A)
								edge [loop right]	node {a} (C)
						(D)		edge [bend left]	node {a} (E)
						(E)		edge [bend left]	node {a} (D)
						(F)		edge [loop right]	node {a} (F)
								edge 				node {b} (D);
	\end{tikzpicture}
\end{equation*}
\item
Für $A_4$ ergeben sich die Sprachen $L(A_4)=(bb)^{*}(baa^{*}b)^{*}+(\epsilon + ba(aa)^{*}+aa^{*}ba(aa)^{*})$ und $L^{\omega}(A_4)=(bb)^{\omega}+(baa^{*}b)^{\omega}+ba^{\omega}+aa^{*}ba^{\omega}$. Damit gilt $L(A_1)\cap L(A_2)\neq L^(A_3)$, da $baab\in L(A_3)$, jedoch $baab\notin L(A_2)$ und damit $baab\notin L(A_1)\cap L(A_2)$.
\end{enumerate}

LENNART:
$$ L^\omega (A_4) = (a^* b a^* b)^\omega + (a^* b a^* b)^* \cdot (a^*b) \cdot (aa)^\omega $$ 

$$ L(A_4) = (a^* b a^* b)^* \cdot (a^* b a (aa)^* + \epsilon) $$ 
\end{document}